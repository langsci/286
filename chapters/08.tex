\documentclass[french,output=paper,colorlinks,citecolor=brown]{../langscibook} 
\author{Sophie Piron\affiliation{Université du Québec à Montréal}\orcid{}} 

\title{Analyse comparative des \textit{Élémens de la grammaire françoise} de Lhomond et de ses \textit{Élémens de la grammaire latine}}

\IfFileExists{../localcommands.tex}{
  % add all extra packages you need to load to this file  

\usepackage{tabularx,multicol} 
\usepackage{url} 
\urlstyle{same}

\usepackage{listings}
\lstset{basicstyle=\ttfamily,tabsize=2,breaklines=true}

\usepackage{./langsci/styles/langsci-optional}
\usepackage{./langsci/styles/langsci-lgr}
\usepackage{./langsci/styles/langsci-gb4e}

  \newcommand*{\orcid}{}
\newcommand*{\markindex}{}

  %% hyphenation points for line breaks
%% Normally, automatic hyphenation in LaTeX is very good
%% If a word is mis-hyphenated, add it to this file
%%
%% add information to TeX file before \begin{document} with:
%% %% hyphenation points for line breaks
%% Normally, automatic hyphenation in LaTeX is very good
%% If a word is mis-hyphenated, add it to this file
%%
%% add information to TeX file before \begin{document} with:
%% %% hyphenation points for line breaks
%% Normally, automatic hyphenation in LaTeX is very good
%% If a word is mis-hyphenated, add it to this file
%%
%% add information to TeX file before \begin{document} with:
%% \include{localhyphenation}
\hyphenation{
affri-ca-te
affri-ca-tes
dis-ci-plin-ary
}

\hyphenation{
affri-ca-te
affri-ca-tes
dis-ci-plin-ary
}

\hyphenation{
affri-ca-te
affri-ca-tes
dis-ci-plin-ary
}

  \bibliography{../localbibliography}
  \togglepaper[8]%%chapternumber
}{}


\abstract{This paper compares two grammars published by Lhomond at the end of the 18th century. One is dedicated to Latin, the other to French. The paper shows how similar both grammars are. They are constructed in a way that the French grammar can be used as an introduction to the Latin grammar. Most importantly, the French grammar has its specific theoretical organization, based on the parts of speech. Such an organization makes concordance syntax fall into an orthographic approach.}

\shorttitlerunninghead{Analyse comparative des \textup{Élémens\ldots} de Lhomond}
\begin{document}
\begin{otherlanguage}{french}
\renewcommand{\chapappifchapterprefix}{Chapter}
\maketitle

\section{Introduction}
\largerpage
 Charles-François Lhomond publie ses \textit{Élémens de la grammaire latine} (EGL) en 1779. Il existe assez peu d’informations sur les éditions parues du vivant de l’auteur. Il semble y en avoir eu neuf, mais la liste complète reste lacunaire, et toutes les éditions ne sont pas disponibles: \textsuperscript{1}1779, \textsuperscript{2}1780, \textsuperscript{3}1781, \textsuperscript{4}1784, \textsuperscript{9}1793. La troisième édition, qui remonte à \citeyear{Lhomond1781}, est la plus récente qui soit consultable à la Bibliothèque nationale de France. Elle servira par conséquent de référence dans le cadre de notre étude. 

Lhomond publie les \textit{Élémens de la grammaire françoise} (EGF) un an après sa grammaire latine. La liste des éditions connues et parues du vivant de l’auteur est, elle aussi, plutôt réduite: \textsuperscript{1}1780, \textsuperscript{5}1786 et \textsuperscript{7}1790. Les différences entre ces éditions sont mineures. Elles relèvent de la mise en page et du style. L’édition de \citeyear{Lhomond1790}, la dernière du vivant de l’auteur, servira ici de référence.

Ces deux publications ont joué un rôle capital dans l’histoire de la grammaire scolaire, tant latine que française. L’importance se jauge à la fois sur le plan éditorial et sur le plan grammatical. Les nombreuses rééditions du vivant de l’auteur, mais surtout tout au long du XIX\textsuperscript{e} siècle (et même encore au début du XX\textsuperscript{e}), ainsi que les adaptations, confirment le succès de librairie sur le marché grammatical. Le catalogue en ligne de la BnF propose ainsi environ 320 ouvrages de Lhomond et de ses adaptateurs pour la grammaire latine et à peu près 480 pour la grammaire française. Pour cette dernière, Chervel a comptabilisé que “La Bibliothèque nationale en possède 760 exemplaires, d’éditions toutes différentes” \citep[63]{Chervel1977}. Quel que soit le chiffre exact des rééditions et adaptations, le succès de ces deux ouvrages est incontestable.

\begin{quote}
    Aucune grammaire française, sans doute, n’a connu un succès aussi durable que celle de Lhomond. \citep[63]{Chervel1977}

    Les \textit{Élémens de la grammaire latine} de \textsc{Lhomond} restent, et de loin, le manuel le plus utilisé au XIX\textsuperscript{e} siècle […]. \citep[16]{Chervel1979}
\end{quote}

Avec le succès commercial désormais attaché au nom de Lhomond croît sa valeur sur le marché symbolique de la grammaire, que ce soit pour les \textit{Élémens} français (\citealt{Chervel1977, Chervel2006, ColombatEtAl2010}) ou les \textit{Élémens} latins (\citealt{Chervel1977, Colombat1999}). Si aucune innovation théorique ne porte son empreinte, l’apport de Lhomond reste incontestable, mais il se situe ailleurs. Il relève, pour les EGF au moins, d’un tour de force pédagogique opéré sur le matériau grammatical et l’ordre de présentation au sein de l’ouvrage \citep{Piron2019}. 

\begin{quote}
La date clé dans l’histoire de la production grammaticale à usage scolaire au XVIII\textsuperscript{e} siècle n’est pas le Restaut de 1732, qui distingue les deux orthographes, ni le Wailly de 1754 qui abandonne la déclinaison, mais le petit Lhomond de 1780. \citep[220]{Chervel2006}

En France, ils [les EGL] ont constitué la grammaire latine de référence pendant près d’un siècle […]. Bien que fondamentalement différents, la méthode de Du Marsais et la grammaire de Lhomond sont sans doute les travaux qui ont le plus marqué l’enseignement du latin au XVIII\textsuperscript{e} siècle. […] la première cédera assez rapidement la place à la seconde qui offrira à l’enseignement du latin un fondement d’autant plus incontesté que son caractère traditionnel ne risquait pas de choquer […]. \citep[106]{Colombat1999}
\end{quote}

L’objectif du présent article est de comparer les grammaires latine et française de Lhomond, telles qu’elles ont été publiées du vivant de l’auteur. La comparaison se trouve justifiée par quatre arguments. Tout d’abord, la grammaire latine constitue le modèle descriptif en usage depuis la naissance de la grammaire française, mais sous une forme évoluée, \textit{la grammaire latine étendue} \citep{Auroux1994}. Par ailleurs, à partir du XVII\textsuperscript{e} siècle (avec la \textit{Nouvelle méthode latine} de Port-Royal, en particulier depuis la réédition de 1650), “la langue latine est désormais décrite du point de vue du francophone” \citep[11]{Colombat1995}, ce qui transforme encore davantage le modèle latin, mais resserre en même temps les liens entre grammaire latine et grammaire française. À cet égard, les EGL de Lhomond inaugurent une certaine forme de modernité en proposant “une méthode d’initiation à la langue latine à l’usage des francophones” \citep[166]{Colombat1999}.

Ensuite, du côté du français cette fois, le XVIII\textsuperscript{e} siècle confère désormais à la grammaire française le rôle de porte d’entrée pour l’étude du latin, et donc de sa grammaire (\citealt{Chevalier2006}, \citealt{Chervel1977}). Ce rôle est d’ailleurs reconnu aux \textit{Élémens} français~de Lhomond: “Par bien des aspects, l’ouvrage [EGF] ne constitue qu’une propédeutique à l’étude du latin.” \citep[165]{Colombat1999}. Lhomond lui-même le précise dans sa préface aux EGF: “C’est par la langue maternelle que doivent commencer les Etudes […] \& cette connoissance leur [aux enfants] sert comme d’introduction aux Langues anciennes qu’on veut leur enseigner.” \citep[3]{Lhomond1790}. Enfin, chose on ne peut plus évidente, la similitude des titres – \textit{Élémens de la grammaire françoise, Élémens de la grammaire latine~} – invite à un parcours comparatif. Nous prendrons les EGF comme point de référence, ce qui ne correspond pas à la chronologie de parution des ouvrages, mais bien à leur articulation.

\section{Structure}

Malgré une structure peu explicite, les EGF se divisent en deux parties. La première est consacrée aux espèces de mots\footnote{\textrm{Lhomond n’utilise pas l’expression} \textrm{\textit{partie du discours}}\textrm{, mais} \textrm{\textit{espèce de mot.}} \textrm{Cet usage se coupe volontairement de la tradition, on le trouve cependant chez d'Olivet (\citeyear{Olivet1767}), à qui Lhomond se réfère par ailleurs (\citeyear[63]{Lhomond1790}). Lhomond ne justifie pas ce choix terminologique, par contre il justifie celui d’autres termes comme} \textrm{\textit{nom}} \textrm{à la place de} \textrm{\textit{substantif}}\textrm{. Comme il l’explique dans la préface des EGL et dans celle des EGF également, il recherche une terminologie plus claire, une terminologie que les enfants puissent entendre.}} et occupe presque la totalité de l’ouvrage (76 pages). La seconde traite de l’orthographe (10 pages). Les EGL présentent, quant à eux, trois parties: les espèces de mots (130 pages), la syntaxe latine (59 pages) et la méthode latine pour traduire les gallicismes en latin (88 pages). Les plans rédactionnels des deux ouvrages semblent donc assez différents au premier abord et ne suivent pas les mêmes proportions pour le traitement des parties du discours. La présence, dans les EGL, d’un chapitre consacré à la syntaxe est pratique courante~dans les grammaires latines et, plus spécifiquement, dans les ouvrages de type \textit{rudiments} \citep{Colombat1999}: on relèvera surtout le texte phare de Lancelot (\citeyear{Lancelot1644}), mais aussi par exemple Chompré (\citeyear{Chompré1751}). Il existe également des ouvrages de type \textit{méthode}, qui s’inscrivent dans la pratique du thème. Ces techniques de traduction peuvent suivre des notions de syntaxe, ce que propose déjà \citet{Lancelot1644}.

Il est normal de ne pas trouver de méthode de traduction dans les EGF. Par contre, l’absence d’une section explicitement consacrée à la syntaxe dans cette petite grammaire française est ce qui frappe le plus quand on compare les deux grammaires de Lhomond. Cela étant dit, si la syntaxe constitue un incontournable dans une grammaire latine, elle se définit moins comme un passage obligé pour les grammaires élémentaires du français. En effet, sur neuf publications similaires aux EGF et parues au XVIII\textsuperscript{e} siècle, seules quatre proposent une partie consacrée à la syntaxe: Gaullyer (\citeyear{Gaullyer1722}), Wailly (\citeyear{Wailly1759}), Bertera (\citeyear{Bertera1773}) et Royon (\citeyear{Royon-professeur1777})\footnote{\textrm{Sans syntaxe: \citet{Restaut1732Abrege}, Panckoucke (\citeyear{Panckoucke1749}), Viard (\citeyear{Viard1763}), \citet{ChompréEtAl1778} et Domergue (\citeyear{Domerque1778}).}}. La présence d’une partie explicitement consacrée à l’\is{orthographe}orthographe dans les EGF relève, en revanche, d’une pratique bien implantée dans les petites grammaires du français. De fait, dans les grammaires que nous avons consultées à ce sujet, seuls \citet{Wailly1759} et \citet{Viard1763} ne proposent pas de chapitre intitulé \textit{orthographe.} Cela ne signifie pas pour autant que ces notions n’y soient pas abordées. 

Au-delà de ces différences entre EGL et EGF, le traitement des parties du discours y est rédigé selon le même principe: d’abord, une série de chapitres proposant des définitions et observations générales pour chaque partie du discours; ensuite, un chapitre (assez long) apportant des précisions sur chaque espèce de mot. Ce chapitre s’intitule \textit{suppléments} dans les EGL, \textit{remarques particulières} dans les EGF. Un tel découpage de la matière grammaticale en une section plus générale et une autre plus complexe est assez fréquent, à l’époque de Lhomond, dans les grammaires consacrées au latin \citep{Colombat1999}. Cette pratique de dédoublement semble également exploitée dans les grammaires françaises, bien qu’elle soit parfois difficile à déceler~ou qu’elle consiste à présenter de nouveau la matière sous un angle un peu différent: on pensera notamment à \citet{Wailly1759}, chez qui des remarques sur les parties du discours prennent place après une partie consacrée aux différentes classes de mots et à la syntaxe; on pensera aussi à Restaut (\citeyear{Restaut1732Abrege}), qui propose des observations sur les parties du discours après les avoir toutes passées en revue au sein de différents chapitres.

La liste des parties du discours est identique, sauf pour l’article (qui n’est présent que dans les EGF) et pour le \is{participe}participe (qui est, en latin, mis en équivalence avec le gérondif et le supin). L’ordre est, lui aussi, le même si ce n’est que la préposition est vue avant l’adverbe dans les EGF et après celui-ci dans les EGL. Le point fréquemment relevé (notamment \citealt{Chervel1977}, \citealt{Colombat1999}) concerne l’\is{adjectif}adjectif: il est hissé au rang de partie du discours, et ce dès 1779 dans les EGL. Comme le signale \citet{Colombat1999}, cette position a déjà été adoptée par la grammaire latine de \citet{Goulier1773}. Pour le français, les prédécesseurs de Lhomond sont Girard (\citeyear{Girard1747}) et Beauzée (\citeyear{Beauzée1767}), mais aussi les grammaires élémentaires de \citet{Royon-professeur1777} et de \citet{Domerque1778}. 

\section{Traitement des parties du discours}

Les EGL et les EGF adoptent un traitement identique des parties du discours. Le développement se déploie en cinq temps, plus ou moins perceptibles dans la rédaction de ces deux grammaires et dans leur mise en page: la définition de la partie du discours traitée; les catégories qui affectent la classe (traditionnellement nommées \textit{accidents}, mais ce terme n’est pas utilisé), à savoir le genre, le nombre, la personne, etc.; la morphologie du mot lorsqu’il varie (féminin, pluriel, conjugaison); la \is{syntaxe d'accord}syntaxe d’accord et la \is{syntaxe de régime}syntaxe de régime. 

\subsection{Définition}

La comparaison entre les EGL et les EGF pour ce qui est de la définition des espèces de mots fait clairement apparaître une volonté rédactionnelle d’uniformiser les deux publications. Il s’agit de proposer deux ouvrages qui puissent être utilisés en séquence, sans coupure théorique.

Le nom, l’\is{adjectif}adjectif, le pronom et le verbe sont définis de manière pratiquement identique, voire identique entre les deux textes, Lhomond ayant seulement supprimé des EGF tout ce qui avait rapport au latin. Ainsi, aucune mention n’est faite des cas et des déclinaisons dans les définitions du nom et de l’\is{adjectif}adjectif. 

\begin{quote}
    L’Adjectif est un mot que l’on ajoute au nom pour marquer la qualité d’une personne ou d’une chose, comme \textit{bon} pere, \textit{bonne} mere; \textit{beau} livre, \textit{belle} image: ces mots, \textit{bon}, \textit{bonne}, \textit{beau}, \textit{belle}, sont des adjectifs joints au nom \textit{pere}, \textit{mere}, \& c. (\citealt[12]{Lhomond1790})
\end{quote}

\begin{quote}
    \textit{L'Adjectif} est un mot que l'on ajoute au nom pour marquer la qualité d'une personne ou d'une chose, comme \textit{bon} pere, \textit{bonne} mere; \textit{beau} livre, \textit{belle} image. \textit{Bon}, \textit{bonne}, \textit{beau}, \textit{belle}, sont des adjectifs: ils se déclinent en latin, \& ils ont les trois genres, masculin, féminin \& neutre. (\citealt[11]{Lhomond1781}\footnote{ \textrm{Les exemples suivants d’adjectifs seront en latin, seuls ceux qui permettent d'introduire la notion d'adjectif sont en français.}})
\end{quote}

Le \is{participe}participe est la partie du discours qui reçoit le traitement le plus différencié entre les deux grammaires. En latin, il est défini au premier chef comme un \is{adjectif}adjectif, même s’il tient à la fois du verbe et de l’\is{adjectif}adjectif. Il est surtout défini par le fait qu’il s’accorde, comme un adjectif. Or la définition de l’adjectif n’avait pas présenté cette catégorie de mot sous cet angle. Dans la version française de la grammaire, les caractéristiques du \is{participe}participe ressortissant à l’adjectif en reproduisent la définition, entièrement fondée, cette fois, sur la qualification.

\begin{quote}
    Le \textit{Participe} est un mot qui tient du verbe \& de l’adjectif, comme \textit{amant, aimé}: il tient du verbe, en ce qu’il en a la signification \& le régime: \textit{aimant Dieu, aimé de Dieu}: il tient aussi de l’adjectif, en ce qu’il qualifie une personne ou une chose, c’est-à-dire, qu’il en marque la qualité. (\citealt[60]{Lhomond1790})
\end{quote}

\begin{quote}
    Les \textit{Participes} sont des Adjectifs qui viennent des verbes; ils s’accordent en genre, en nombre \& en cas avec le nom auquel ils sont joints, \& de plus ils gouvernent le même cas que le Verbe d’où ils viennent; c’est pour cela qu’on les nomme \textit{Participes}, parce qu’ils tiennent de l’adjectif et du verbe. (\citealt[90]{Lhomond1781})
\end{quote}

Les autres parties du discours – adverbe, préposition, conjonction et interjection – sont définies avant tout comme des mots indéclinables dans les \textit{Élémens} latins. 

\begin{quote}
    L’\textit{Adverbe} est un mot indéclinable, qui se joint le plus souvent à un Verbe, \& en détermine la signification. (\citealt[92]{Lhomond1781})
\end{quote}

\begin{quote}
    La \textit{Préposition} est un mot indéclinable, qui joint à un Nom, ou à un Pronom, veut ce Nom ou Pronom à l’Accusatif ou à l’Ablatif. (\citealt[94]{Lhomond1781})
\end{quote}

\begin{quote}    
    La \textit{Conjonction} est un mot indéclinable qui sert à lier les parties du discours. (\citealt[96]{Lhomond1781})
\end{quote}

\begin{quote}
    L’\textit{Interjection} est un mot indéclinable qui sert à marquer les différens mouvemens de l’ame. (\citealt[97]{Lhomond1781})
\end{quote}

Lhomond a évacué cet angle morphologique de ses \textit{Élémens} français et les inscrit ainsi dans un projet délatinisant de la grammaire. Ce n’est que plus tard que les adaptateurs de Lhomond adopteront un angle orthographique et introduiront le concept de \textit{mot invariable} en remplacement de \textit{mot indéclinable} (par exemple, Le Tellier \citeyear{Le-Tellier1811}).

On décèle quelques différences d’approche dans les définitions des mots invariables, excepté pour l’interjection. Si la préposition sert à joindre dans les deux ouvrages, un éclairage différent prend place dans cette partie du discours. En latin, la préposition s’inscrit dans une relation à deux termes: la préposition et le mot dont elle régit le cas. En français, elle s’inscrit dans une relation à trois termes: les deux mots reliés par la préposition, et la préposition elle-même. Cette fois, la préposition ne régit pas de cas (cependant elle régit bien le mot qui suit), mais marque un rapport d’ordre sémantique (le lieu, l’ordre, l’union, etc.). 

L’adverbe, dans les EGL, se joint habituellement à un verbe, mais rien n’est dit sur les autres mots qu’il pourrait déterminer, tandis qu’il est précisé dans les EGF qu’il peut également se joindre à un \is{adjectif}adjectif. À l’inverse, la conjonction présente un comportement plus restreint dans les EGF: au lieu de pouvoir lier n’importe quelles parties du discours, elle ne joint que des phrases ou propositions.

\subsection{Accidents}

Après chaque définition d’une espèce de mots variables, Lhomond propose habituellement les accidents\footnote{ \textrm{Lhomond n’utilise pas le terme} \textrm{\textit{accident}} \textrm{et n’en propose aucun autre en remplacement.}} qui l’affectent, en tout cas ceux qui ont une incidence sur l’apprentissage élémentaire de la langue: genre, nombre, personne, mode et temps. Ces attributs sont identiques en latin et en français, mais ne prennent pas toujours les mêmes valeurs. Ainsi, en latin, outre le masculin et le féminin, la valeur du genre peut être le neutre. Autre exemple, le conditionnel est un mode en français, alors qu’il est absent de la liste des modes en latin.

La présentation des accidents manque quelquefois de systématicité. Ainsi, l’\is{adjectif}ad\-jec\-tif possède un genre, mais rien n’est dit sur le nombre alors même que le chapitre propose une règle pour la formation du pluriel des \is{adjectifs}adjectifs. Relevons, par ailleurs, que le nom ne possède pas la catégorie de la personne, alors que celui-ci est mis en œuvre dans la règle d’accord du verbe avec son sujet: ``Tout verbe doit être du même nombre \& de la même personne que son nominatif ou sujet'' \citep[47]{Lhomond1790}; ``Tout Verbe s'accorde en nombre \& en personne avec son nominatif'' \citep[31]{Lhomond1781}. Le chapitre consacré au verbe ne présente pas non plus très clairement la personne comme un accident de cette espèce de mot, mais plutôt comme une information qui provient de l'emploi des pronoms \textit{je, tu,} etc. Quant à la classe des pronoms, elle n'est pas mise en relation avec la catégorie de la personne. Les pronoms personnels sont toutefois présentés sous cet angle\footnote{ \textrm{L’attribut y est exploité comme un élément de désignation d’une véritable personne (celle qui parle, à qui on parle ou de qui on parle) plutôt que comme un trait grammatical.}}. Il ne faut pas imputer cette incohérence à Lhomond lui-même, qui ne fait ici que s'inscrire dans la tradition. À l'époque, la notion de personne, centrale dans l'accord du verbe, n’est pas attachée au substantif puisque celui-ci ne présente pas de marque morphologique de personne \citep{Colombat1999}.

\subsection{Morphologie}

Les aspects morphologiques nécessitent de nombreux tableaux dans les EGL puisqu’il faut présenter les différents modèles de déclinaison du nom\footnote{ \textrm{Le nom est la classe de mots la plus complexe de ce point de vue avec cinq déclinaisons et dix modèles puisque certaines déclinaisons présentent plusieurs comportements.}} et de l’\is{adjectif}adjectif, et également les déclinaisons de chaque type de pronom. Il faut aussi présenter les modèles de conjugaison\footnote{\textrm{Douze tableaux: quatre conjugaisons pour chacune des trois sortes de verbes (actif, passif, déponent).}}. À part pour les pronoms, les EGF ne requièrent de tableaux que dans le traitement du verbe\footnote{ \textrm{Douze tableaux:} \textrm{\textit{avoir}}\textrm{;} \textrm{\textit{être}}\textrm{; verbes en -}\textrm{\textit{er}}\textrm{, en -}\textrm{\textit{ir}}\textrm{, en -}\textrm{\textit{oir}}\textrm{, en -}\textrm{\textit{re}}\textrm{; temps primitifs réguliers, puis irréguliers; verbes passifs, neutres, réfléchis et impersonnels.}}. En effet, Lhomond a évacué de la grammaire française la déclinaison du nom, choix déjà répandu à l’époque, mais pas encore entériné par tous les grammairiens. Plusieurs d’entre eux qui publient justement des grammaires élémentaires à la même époque que Lhomond proposent encore des tableaux de déclinaison associés au nom: notamment \citet{Viard1763}, \citet{Bertera1773}, \citet{Royon-professeur1777}\footnote{ \textrm{\citet{Royon-professeur1777} propose des tableaux de déclinaison, très complets, mais lors du traitement de l’article.}} et \citet{ChompréEtAl1778}.

Lhomond procède à une \is{délatinisation}délatinisation de la grammaire française en excluant ainsi la déclinaison de ses EGF. C’est un choix qui mérite encore d’être souligné à la fin du XVIII\textsuperscript{e} siècle, d’autant que cette option théorique établit une différence non négligeable entre les EGL et les EGF. En effet, soulignons-le une fois de plus, malgré certaines apparences, les deux ouvrages sont construits de manière similaire et de façon à pouvoir être utilisés en séquence. Se soustraire à la tradition de la déclinaison en français, c’est se couper d’une passerelle pédagogique aisée vers le latin. Néanmoins, cette \is{délatinisation}délatinisation laisse place, en filigrane, à une conception très latine de la morphologie nominale et adjectivale. Dans les EGL, la déclinaison est rapidement introduite : “En latin le Nom change sa derniere syllabe” \citep[2]{Lhomond1781}. Dans les EGF, les règles morphologiques sont présentées sous le même angle, comme une modification de la finale du mot: “Pour former le pluriel, ajoutez \textit{s} à la fin du nom” \citep[9]{Lhomond1790}; “Quand un adjectif ne finit point par un \textit{e}, on y ajoute un \textit{e} muet pour former le féminin” \citep[12]{Lhomond1790}. Bien sûr, le lien entre la déclinaison latine et les modifications morphologiques du français reste ténu, mais il doit être souligné, nous semble-t-il. Enfin, tout en établissant en français une conception morphologique prête à être réinvestie dans l’étude du latin, Lhomond propose un traitement orthographique de la formation du pluriel et du féminin. Il en expose les principales règles: -\textit{x} (\textit{caillou, cailloux}), consonne double (\textit{cruel, cruelle}), -\textit{que} (\textit{public, publique}), etc. Fait non négligeable, il positionne cette information très tôt lorsqu’il aborde une partie du discours. Ce choix montre une volonté d’associer l’\is{orthographe}orthographe à la présentation des parties du discours. Celles-ci sont donc conceptualisées sous cet angle. Il s’agit, plus précisément, de l’\is{orthographe}orthographe de principe de Restaut, celle des ``différentes terminaisons des noms par rapport aux genres ou aux nombres, \& des verbes par rapport aux tems \& aux personnes'' (\citealt[48]{Restaut1732Principes}), et non de l'\is{orthographe}orthographe d’usage. 

\section{Syntaxe d’accord}

Lhomond établit clairement la différence entre \is{syntaxe d'accord}\textit{syntaxe d’accord} et \is{syntaxe de régime}\textit{syntaxe de régime}. Il le fait au commencement de la partie consacrée à la syntaxe dans les EGL, soit à peu près à la moitié de l’ouvrage, mais beaucoup plus tôt dans les EGF, plus précisément dans une note de bas de page au sein du chapitre consacré aux \is{adjectifs}adjectifs. Il définit la \is{syntaxe d'accord}syntaxe d’accord comme la syntaxe “par laquelle on fait accorder deux mots en genre, en nombre, \& c.” (\citealt[15]{Lhomond1790}; \citealt[131]{Lhomond1781}). Ce n’est autre que la syntaxe de concordance ou de convenance, généralisée au XVI\textsuperscript{e} siècle et bien en place à l’âge classique, dans les grammaires latines en particulier \citep{Colombat1999}. 

Fait majeur: les aspects syntaxiques que Lhomond aborde dans la partie \textit{syntaxe} des EGL sont classés par espèce de mots (syntaxe des noms, des \is{adjectifs}adjectifs, des verbes, des pronoms, des \is{participe}participes, des prépositions, des adverbes et des conjonctions) et ensuite par type de syntaxe. Ils dédoublent à l’occasion la règle principale qui a déjà été présentée dans les chapitres consacrés aux différentes espèces de mots, mais c’est véritablement dans cette 2\textsuperscript{e} partie du livre que l’exposé des règles est fait rigoureusement. La terminologie utilisée mérite d’être relevée. Dans la 1\textsuperscript{re} partie de l’ouvrage (les espèces de mots), Lhomond fait mention de règles (\textit{règle des adjectifs, des pronoms adjectifs, du} qui \textit{et du} que \textit{relatifs} et \textit{de la règle générale pour les verbes}). Dans la 2\textsuperscript{e} partie (la syntaxe), il utilise cette fois le terme \textit{accord} (\textit{accord de deux noms, de l’adjectif avec le nom, du verbe avec le nominatif ou sujet, du pronom avec l’antécédent}), excepté pour le participe, dont la règle est exposée en tant que \textit{syntaxe des} \is{participe}\textit{participes.} Lhomond a ainsi ajouté d’autres règles à celles sur lesquelles s’entendaient les grammaires latines humanistes: le verbe avec son nominatif, l’\is{adjectif}adjectif avec le substantif et le relatif avec son antécédent \citep{Colombat1999}. Ne consacrant aucun chapitre explicite à la syntaxe dans ses EGF, Lhomond a intégré les règles d’accord dans la première partie de l’ouvrage, plus précisément dans chaque partie du discours concernée. Des titres détachent clairement les sections consacrées à la \is{syntaxe d'accord}syntaxe d’accord et c’est bien ce terme qui est employé: \textit{accord des adjectifs avec les noms, accord des verbes avec leur nominatif, accord du} \is{participe}\textit{participe passé.} Il arrive toutefois que le titre n’utilise pas le terme \textit{accord}, mais use simplement du mot \textit{règle}: \textit{règle des pronoms, règle du} qui \textit{ou} que \textit{relatif.}

\section{Syntaxe de régime}

Lhomond définit la \is{Syntaxe de régime}syntaxe de régime comme la syntaxe, en français, “par laquelle un mot régit \textit{de} ou \textit{à} devant un autre mot” \citep[15]{Lhomond1790}, en latin, “par laquelle un mot régit un autre mot à tel cas, à tel mode, \& c.” \citep[131]{Lhomond1781}. Les aspects syntaxiques présentés dans les EGF couvrent toutefois plus que ce que ne prévoit la définition étroite qui y est donnée. La ventilation de la matière reproduit le choix rédactionnel de ce qui a été fait pour la \is{syntaxe d'accord}syntaxe d’accord, tant dans les EGL que dans les EGF. De nouveau, le plan rédactionnel des EGF intègre la syntaxe – de régime, cette fois~– dans le traitement de chaque espèce de mot, à la suite de la \is{syntaxe d'accord}syntaxe d’accord lorsque les deux syntaxes sont à l’œuvre. Si le terme \textit{régime} n’est pas systématique, la présence de la \is{syntaxe de régime}syntaxe de régime est indéniable: pour joindre un nom à un mot précédent; régime des \is{adjectif}adjectifs, régime des verbes (actifs, passifs, neutres, etc.) et des conjonctions.

\section{Conclusion}

La comparaison des deux grammaires élémentaires de Lhomond a permis de constater que seule la partie consacrée aux espèces de mots se retrouvait dans la structure des deux publications. Au-delà de cette similitude très sommaire, plusieurs parallèles ont été relevés. Les définitions des parties du discours sont très proches, voire parfois identiques. Le traitement en deux niveaux de difficulté est reproduit dans les EGF. Le choix de considérer l’\is{adjectif}adjectif comme une espèce de mot à part entière prend place dès 1779 dans les EGL. Chaque partie du discours est abordée selon un mouvement en cinq temps, lorsque cela est possible: définition, accidents, formes, \is{syntaxe d'accord}syntaxe d’accord, \is{syntaxe de régime}syntaxe de régime.

Le véritable point commun entre les deux grammaires de Lhomond est la charpente grammaticale. Celle-ci est constituée des espèces de mots: elle soutient les suppléments, la syntaxe, l’\is{orthographe}orthographe. Lhomond opère toutefois un changement majeur lors de la rédaction des \textit{Élémens} français. En effet, il renforce le rôle de cette structure sous-jacente~en intégrant complètement la syntaxe dans les espèces de mots. Le plan rédactionnel était en germe dans les EGL, où la syntaxe s’immisçait dans les espèces de mots. 

Lhomond procède à la fois à une \is{délatinisation}délatinisation et à une latinisation en filigrane de la grammaire française. Tout en supprimant les déclinaisons et le concept de mot indéclinable, il parvient à traiter la variation sous l’angle de la morphologie flexionnelle, mais par le fait même il l’envisage sur le plan de l’\is{orthographe}orthographe. Les \textit{Élémens} français proposent ainsi une évolution majeure sur deux points. La grammaire est désormais une grammaire des mots, qui déploie tout ce qui s’y rapporte: \is{orthographe}orthographe de principe, \is{syntaxe d'accord}syntaxe d’accord, \is{syntaxe de régime}syntaxe de régime, mais aussi \is{orthographe}orthographe d’usage. La nature d’un mot est désormais la porte d’entrée dans la grammaire, ce qui la rend beaucoup plus aisée à consulter et à comprendre. Autre point d’évolution: le positionnement de la \is{syntaxe d'accord}syntaxe d’accord fait basculer cette partie de la syntaxe dans le domaine de l’\is{orthographe}orthographe. Ainsi transformés, les \textit{Élémens de la grammaire françoise} constituent le point de départ de la grammaire scolaire. 

{\sloppy\printbibliography[heading=subbibliography,notkeyword=this]}
\end{otherlanguage}
\end{document}
