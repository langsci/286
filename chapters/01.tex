\documentclass[output=paper]{langsci/langscibook} 
\title{Le terme \textit{accidentia} chez les grammairiens romains} 
\author{Vladimir I. Mazhuga\affiliation{Saint Petersburg Institute of History, RAS}} 

\abstract{This chapter deals with the custom of Latin grammarians of introducing the attributes of the parts of speech with the verb \emph{accident} (3rd person plural), as well as with the use of the nominalized participle plural \is{accidentia}\textit{accidentia}. Contrary to the most widespread view, this chapter demonstrates an early development of the usage in question from Greek rhetorical doctrines of the 1st century BC. The teachings and writings of Theodore of Gadara, Cornelius Celsus and Pliny the Elder had a crucial impact on the development of this usage. One does not need to search for its origin in the hypothetical grammatical vocabulary of the Stoics or in treatises on Greek grammar. Similar customs also appeared in response to the new needs of school teaching in Greek grammatical doctrine, although somewhat later. The Greek analogy to the Latin terms were the plural noun \is{τὰ παρεπόμενα}\textit{τὰ παρεπόμενα} and the verb \textit{παρέπομαι}, which were borrowed from the sophistic dialectic at the time of the flourishing Second \is{Sophistic}Sophistic. Apollonius Dyscolus contributed much to the initial introduction of the terms in question into school textbooks on Greek grammar.}

\begin{document}
\begin{otherlanguage}{french}
\maketitle

\section{Introduction} 
La présente contribution pourrait prendre comme point de départ l’article de Maria Teresa Vitale “Per una terminologia grammaticale : \is{parepómena}\textit{parepómena} – \is{accidentia}\textit{accidentia}” \citep{vitale_per_1982}, tant son auteure paraît avoir magistralement traité le sujet que nous allons aborder. Elle nous semble toutefois avoir élaboré des hypothèses sujettes à caution, plutôt que d’examiner scrupuleusement le matériau disponible. Ces hypothèses se réduisent en substance à la thèse, formulée jadis par Karl Barwick, selon laquelle le terme \is{παρεπόμενα}\textit{παρεπόμενα} des grammairiens grecs et le terme \is{accidentia}\textit{accidentia} des grammairiens latins rendaient compte tous deux de la notion de \is{συμβεβηκότα}\textit{συμβεβηκότα}, qui joua un rôle fondamental dans la philosophie d’Aristote et, dans une moindre mesure, dans celle des \is{Stoics}stoïciens (\citealt{barwick_remmius_1922}: 107--108 ; \citealt{vitale_per_1982}: 212–214). Tenant pour une vérité acquise que les grammairiens latins suivaient assez fidèlement les maîtres grecs, Vitale s’est vue obligée de démontrer d’emblée que dès l’époque d’Aristote le terme \is{συμβεβηκότα}\textit{συμβεβηκότα} pouvait désigner aussi bien les propriétés stables distinctives des êtres que les particularités occasionnelles, celles auxquelles Aristote appliquait habituellement les termes \is{συμβεβηκός}\textit{τὸ συμβεβηκός} ou \is{συμβεβηκότα}\textit{τὰ συμβεβηκότα}. 

Dans sa tentative de réinterprétation de la notion de \is{συμβεβηκός}\textit{συμβεβηκός}, Vitale s’est concentrée avant tout sur la doctrine aristotélicienne de l’\textit{essence} de l’\textit{être} et des attributs de celui-ci. À l’appui de l’idée selon laquelle \is{συμβεβηκός}\textit{συμβεβηκός}, en tant qu’attribut, pouvait exprimer chez Aristote aussi bien l’essence du sujet que ses propriétés occasionnelles, elle évoque une phrase des \textit{Seconds analytiques~}:

\begin{quote}
    Ὑπόκειται δὲ ἓν καθ᾿ ἑνὸς κατηγορεῖσθαι, αὐτὰ δὲ αὐτῶν, ὅσα μὴ τί ἐστι, μὴ κατηγορεῖσθαι. \is{συμβεβηκότα}συμβεβηκότα γάρ ἐστι πάντα, ἀλλὰ \textit{τὰ μὲν καθ᾿ αὑτά, τὰ δὲ καθ᾿ ἕτερον τρόπον·} (\citetitle{aristotle_posterior_1960} 83b.19–83b.21).

    `We have now established that in predication one is asserted of one subject, and that predicates (except those which denote essence) are not predicated of one another. They are all attributes, \textit{some per se and others in a different sense}' (transl. in \citetitle{aristotle_posterior_1960}: 125).
\end{quote}

Il faut entendre ici \textit{τὰ καθ᾿ αὑτά} évidemment selon son sens direct, c’est-à-dire en tant qu’attributs comme tels, et il est inutile, en revanche, d’y chercher le sens d’une qualité distinctive essentielle. Aristote désignait celle-ci par le terme spécifique τὸ ἴδιον, qu’il opposait le plus souvent à τὸ \is{συμβεβηκός}\textit{συμβεβηκός}, bien qu’il concédât parfois à ce dernier le sens d’un \textit{propre relatif et temporaire} (\textit{πρός τι ἴδιον}; cf. \citetitle{aristoteles_topica_1958} 102b22). On peut supposer ce même sens dans l’expression \textit{τὰ καθ᾿ ἕτερον τρόπον}. Vitale déduit néanmoins de la phrase citée une formulation tout à fait inattendue : “I due \is{συμβεβηκότα}\textit{συμβεβηκότα}, distinti in καθ’ αὑτά  = essenziali e καθ’ ἕτερον τρόπον – accidentali, erano già presenti, in un contesto più strettamente logico, negli \textit{Analytica Posteriora}” (\citealt{vitale_per_1982}: 202).

Traitant des notions de \is{συμβεβηκός}\textit{συμβεβηκός} et d’\is{accidens}\textit{accidens}, il semblait naturel de borner l’analyse à la doctrine philosophique de l’\textit{être} individuel. Aristote ne cessait de répéter que la catégorie de \is{συμβεβηκός}\textit{συμβεβηκός} ne s’applique qu’à l’individuel. À partir de la logique des \is{sophistes}\text{sophistes}, la notion de \textit{τὰ} \is{παρεπόμενα}\textit{παρεπόμενα} ne s’appliquait cependant qu’aux qualités propres à tout un genre. Vitale passe sous silence les idées d’Aristote sur ce sujet, exprimées de manière éloquente dans ses \textit{Réfutations sophistiques.} En dévoilant les fausses prémisses des conclusions dont les \is{sophistes}\text{sophistes} usaient dans la réfutation de leurs adversaires, il s’attaque, entre autres, à la fausse déduction à partir de ce qui accompagne généralement une classe de choses (\is{παρεπόμενον}\textit{παρεπόμενον}), où une caractéristique stable est remplacée insidieusement par une propriété occasionnelle (\is{συμβεβηκός}\textit{συμβεβηκός}) :

\begin{quote}
    
(…) \textit{τὸ μὲν \is{συμβεβηκὸς}\text{συμβεβηκὸς} ἔστιν ἐφ’ ἑνὸς μόνου λαβεῖν, (...) τὸ δὲ \is{παρεπόμενον}\text{παρεπόμενον} ἀεὶ ἐν πλείοσιν} τὰ γὰρ ἑνὶ καὶ ταὐτῷ ταὐτὰ καὶ ἀλλήλοις ἀξιοῦμεν εἶναι ταὐτά· διὸ γίνεται παρὰ τὸ ἐπόμενον ἔλεγχος (\citetitle{aristoteles_topica_1958} 168b.29--168b.35).

(…) `\textit{You may secure an admission of the accident in the case of one thing only,} (...) \textit{whereas the consequent always involves more than one thing}; for we claim that things that are the same as one and the same thing are also the same as one another, and this is the ground of a refutation dependent on the consequent' (transl. \citealt{barnes_complete_1984} : 285).

\end{quote}

À la différence de la notion grecque de \is{συμβεβηκότα}\textit{συμβεβηκότα} la notion latine apparentée d’\is{accidentia}\textit{accidentia} a pris progressivement chez les grammairiens romains une signification particulière qui allait bien au-delà des bornes de l’\textit{être} individuel. Mais avant d’examiner ce sujet, essayons de poursuivre le raisonnement de Vitale en ce qui concerne les précédents qu’elle a cru repérer dans la grammaire grecque. À côté du terme grec \is{συμβεβηκός}\textit{συμβεβηκός}, elle tente de reconstituer l’histoire du terme \is{τὰ παρεπόμενα}\textit{τὰ παρεπόμενα} traité comme équivalant de \is{τὰ παρεπόμενα}\textit{τὰ παρεπόμενα}. Vitale ne doutait pas que le terme \is{παρεπόμενα}\textit{παρεπόμενα} fût établi dans la grammaire grecque dès le temps d’Aristophane de Byzance (vers 257–180 av. J.-C.) et de son élève alexandrin Aristarque (ca 216 – ca 144 av. J.-C.), c’est-à-dire dès la fin du III\textsuperscript{e} siècle av. J.-C. Mais les arguments probants lui font défaut. Elle se contente de renvoyer le lecteur aux exemples contenus dans la \textit{Τέχνη} attribuée à Denys le Thrace, grammairien célèbre du début du I\textsuperscript{er} siècle av. J.-C., en négligeant totalement les recherches de Vincenzo Di Benedetto, qui montrait bien, dès les années cinquante, le caractère tardif du corps principal de ce traité scolaire qui semble avoir été composé dans l’Antiquité tardive (cf. \citealt{di_benedetto_dionisio_1958,di_benedetto_dionisio_1959}). Néanmoins, Vitale se voit tenue d’expliquer l’absence ultérieure du terme \is{τὰ παρεπόμενα}\textit{τὰ παρεπόμενα} dans les textes grammaticaux avant Apollonius Dyscole, c’est-à-dire avant le deuxième tiers du II\textsuperscript{e} siècle ap. J.-C., ce en raison de l’emploi prétendument synonymique du terme latin \is{accidentia}\textit{accidentia} chez les grammairiens romains à partir du I\textsuperscript{e}\textsuperscript{r} siècle av. J.-C.

L’absence de témoignages sur l’emploi du terme \is{τὰ παρεπόμενα}\textit{τὰ παρεπόμενα} chez les grammairiens grecs est à relier, selon Vitale, à la prépondérance du \is{vocabulaire}\text{vocabulaire} \is{stoïcien}\text{stoïcien} dans la grammaire grecque de la période car, selon une supposition de Karl Barwick, le terme \textit{συμβεβηκότα} y jouait le rôle qui incombait à \textit{παρεπόμενα} chez les anciens grammairiens alexandrins (\citealt{barwick_remmius_1922}: 97, 107; \citeyear{barwick_probleme_1957}: 48). Barwick a élaboré sa doctrine en s’appuyant sur des raisonnements que l’on pourrait qualifier d’arbitraires. C’est dans les interprétations byzantines des faits de grammaire qu’il a trouvé l’unique exemple d’assimilation du terme \textit{παρεπόμενον} à celui de \textit{συμβεβηκός} : \is{παρεπόμενον}\textit{παρεπόμενον δὲ ἐστι} \is {συμβεγηκός συμβεγηκός} (\textit{Sch. Dion. Thr.} 217.23, cf. \citealt{barwick_probleme_1957} : 47). Sans prendre en compte la différenciation fondamentale que les \is{sophistes}\text{sophistes} anciens et Aristote établissaient entre la notion de \textit{παρεπόμενον} et celle de \textit{συμβεβηκός}, et sans essayer de l’atténuer en quelque sorte, comme Vitale a tenté, mais sans succès, de le faire, Barwick affirmait tout simplement que la signification de quelque chose d’occasionnel convenait bien aux attributs des parties du discours, définis par lui comme \textit{zufällige Merkmale} (\citealt{barwick_probleme_1957} : 48). 

Les chercheurs qui ont travaillé après Vitale ne s’intéressaient plus au rapport sémantique entre les termes \is{παρεπόμενον}\textit{παρεπόμενον} et \is{συμβεβηκός}\textit{συμβεβηκός}. Cependant, comme nous le verrons, certains faits les faisaient tenir à l’idée de Barwick, selon laquelle les \is{stoïciens}\text{stoïciens} avaient coutume de désigner les propriétés des parties du discours par le terme \is{συμβεβηκότα}\textit{συμβεβηκότα}. On se demandait parfois encore si l’emploi du terme \is{παρεπόμενον}\textit{παρεπόμενον} précédait chronologiquement l’emploi de \is{συμβεβηκός}\textit{συμβεβηκός} chez les grammairiens grecs (cf. toutefois de \citealt{jonge_between_2008} : 154–155), mais on ne discutait point la parenté sémantique présumée de ces deux termes, et personne n’a mis en doute jusqu’ici l’essentiel des raisonnements de Vitale. 

Personne n’a mené non plus l’analyse systématique des témoignages disponibles. À y regarder de près, on ne peut parler que d’une analogie dans les emplois du terme \is{τὰ παρεπόμενα}\textit{τὰ παρεπόμενα}, chez les grammairiens grecs, et du terme \is{accidentia}\textit{accidentia}, chez les grammairiens latins. Chacun des deux termes a eu son histoire propre et leur signification exacte témoigne d’une origine différente.

Pour terminer notre analyse du terme \is{τὰ παρεπόμενα}\textit{τὰ παρεπόμενα}, examinons la thèse de Vitale selon laquelle ce terme était bien établi dans la grammaire grecque à l’époque d’Apollonius Dyscole (\citealt{vitale_per_1982}: 212). L’œuvre d’Apollonius Dyscole atteste pourtant plutôt des tout premiers débuts d’un nouvel usage terminologique. 

Le traité d’Apollonius \textit{de Pronomine} est chronologiquement le premier dans la série de ses traités grammaticaux qui nous sont parvenus, et il accuse une forte dépendance vis-à-vis des prédécesseurs d’Apollonius, surtout du grammairien du I\textsuperscript{er} siècle av. J.-C. Tryphon. Dans ce traité, Apollonius emploie habituellement le verbe \textit{παρακολουθέω} au lieu de son synonyme \textit{παρέπομαι}, comme l’avaient fait tant d’autres auteurs de la période précédente (19 exemples apolloniens, y compris le substantif \textit{ἡ παρακολούθησις} de même racine, contre trois exemples seulement du verbe \textit{παρέπομαι} et un exemple du participe \is{παρεπόμενον}\textit{παρεπόμενον}). Dans son œuvre fondamentale \textit{Syntaxis}, le vocabulaire d’Apollonius change d’une manière impressionnante. L’auteur montre ici une prédilection pour les participes substantivés \is{τὰ παρεπόμενα}\textit{τὰ παρεπόμενα} (22 exemples) et \is{τὸ παρεπόμενον}\textit{τὸ παρεπόμενον} (13 exemples), un peu moins pour de simples participes (21 exemples) ; nous trouvons en outre dix cas où le verbe \textit{παρέπομαι} est employé, ce qui donne au total 66 exemples, contre 22 exemples avec le verbe \textit{παρακολουθέω} et trois substantifs de même racine. Le substantif pluriel \is{τὰ παρεπόμενα}\textit{τὰ παρεπόμενα} est donc ici le plus fréquent, et l’on peut le concevoir comme un vrai terme technique. 

Il faut signaler toutefois l’emploi prépondérant de ce terme dans le large domaine du raisonnement logique. Apollonius l’applique aux différentes propriétés sémantiques et morphologiques des parties du discours, aussi bien qu’aux formes particulières de certains mots. Mais dans sa \textit{Syntaxis}, il n’emploie ce verbe et les mots de la même racine que sept fois seulement appliqués aux attributs des parties du discours, en leur donnant un sens général (\citetitle{apollonius_dyscolus_constructione_1910} 320.7 (\textit{παρέπομαι}) ; 118.13 ; 324.3 (\is{παρεπόμενον}\textit{παρεπόμενον}) ; 75.9 ; 145.9 ; 267.1--267.2 (\is{παρεπόμενα}\textit{παρεπόμενα})). Une fois seulement, Apollonius use du verbe \textit{παρέπομαι} pour indiquer les propriétés distinctives concrètes des parties du discours :

\begin{quote}
    \begin{otherlanguage}{greek}
    Καθὼς ἔφαμεν, ἔστιν γενικωτάτη ἡ τῶν ἀπαρεμφάτων ἔγκλισις, ἀναγκαίως λείπουσα τοῖς προδιαπορηθεῖσι, \textit{〈τοῖς προσώποις καὶ〉 τῷ {\is{παρεπομένῳ ἀριθμῷ}\textbf{{παρεπομένῳ\linebreak ἀριθμῷ}}, ὃς οὐ φύσει παρέπεται τῷ ῥήματι}} παρακολούθημα δὲ γίνεται προ\-σώπων τῶν μετειληφότων τοῦ πράγματος
    \end{otherlanguage} (\citetitle{apollonius_dyscolus_constructione_1910} 324.10--325.1).
    
    `Comme nous le disions, le mode infinitif est le plus général, puisque lui font nécessairement défaut les accidents dont on a vu plus haut qu’ils faisaient problème : \textit{〈la personne et〉 le nombre – ce dernier n’étant pas par nature un accident du verbe,} mais une dépendance des personnes qui prennent part à l’acte.' (\citealt{lallot_apollonius_1997} : 327).

\end{quote}

On voit bien que \is{παρεπόμενον}\textit{παρεπόμενον} n’est pas encore chez Apollonius un terme technique du lexique grammatical, mais il est évident qu’à partir d’un certain moment ce grammairien éminent montre une prédilection pour ce terme et ses dérivés. On peut expliquer aisément cette prédilection par l’épanouissement de la Seconde \is{Sophistique}\text{Sophistique}, phénomène culturel dont est fortement empreint l’enseignement scolaire du temps d’Apollonius, et qui a fait renaître entre autres la dialectique des anciens \is{sophistes}\text{sophistes}. 

C’est aussi dans l’histoire de l’\is{école}\text{école} et surtout dans l’enseignement de la \is{rhétorique}rhétorique, étroitement lié aux études de grammaire, qu’il faut chercher les raisons de l’emploi du terme \is{accidentia}\textit{accidentia} et surtout du verbe \textit{accido} (sous la forme de la 3\textsuperscript{e} personne du pluriel \textit{accidunt}) chez les grammairiens latins. Mais un autre phénomène, plus ancien, a déterminé ici le choix du terme. En laissant maintenant de côté les raisonnements de Vitale, examinons de plus près les observations des savants qui ont essayé de retracer l’histoire des termes grecs \is{τὸ συμβεβηκός}\textit{τὸ συμβεβηκός} – \is{τὰ συμβεβηκότα}\textit{τὰ συμβεβηκότα} et leurs équivalents latins \textit{accidens} – \textit{accidentia.} Comme nous allons le voir, tous les chercheurs partaient des exemples que l’on trouve chez Denys d’Halicarnasse (à Rome en 30/29 jusqu’à 7 env. av. J.-C.). En voici un parmi d’autres. En parlant de la maîtrise exceptionnelle des règles de la \is{rhétorique}\text{rhétorique} chez Démosthène, Denys traite en général de l’apprentissage régulier aussi bien des éléments les plus simples de l’\is{art rhétorique}\text{art rhétorique} que de ceux de la prosodie et des catégories grammaticales. 

\begin{quote}
    \begin{otherlanguage}{greek}
    κρατήσαντες δὲ τούτων, τὰ τοῦ λόγου μόρια· ὀνόματα λέγω καὶ ῥήματα καὶ συνδέσμους· καὶ τὰ \is{συμβεβηκότα}\textit{συμβεβηκότα} τούτοις, συστολάς, ἐκτάσεις· ὀξύ\-τη\-τας, βαρύτητας· γένη, πτώσεις, ἀριθμούς, ἐγκλίσεις, τὰ ἄλλα παραπλήσια τούτοις μυρία ὄντα (\citetitle{dionisius_halicarnaseus_opuscula_nodate} 52, 242.20--242.23).
    \end{otherlanguage}
    
    `Quand nous possédons parfaitement cela, nous étudions les parties du langage, je veux dire les noms, les verbes, les mots de liaison, et \textit{tous leurs accidents}, abrégement, allongement, accent aigu, accent grave, genre, cas, nombre, flexion, et mille autres choses du même genre' (\citealt{aujac_denys_1988} : 151).
\end{quote}

Le terme \is{τὰ συμβεβηκότα}\textit{τὰ συμβεβηκότα} est appliqué ici à la fois aux attributs des parties du discours et aux particularités de la prosodie. Il s’agit de l’usage pratique et parfois personnel de différents éléments du langage. On peut penser en outre aisément que le terme \is{τὰ συμβεβηκότα}\textit{τὰ συμβεβηκότα} était employé précisément dans le sens que lui prêtait Barwick de quelque chose d’individuel et d’occasionnel. D’autres exemples de l’emploi semblable du terme \is{τὰ συμβεβηκότα}\textit{τὰ συμβεβηκότα} chez Denys renforcent encore cette impression (\textit{Amm}. II 1–12, 421.5–432.13, spec. 421.17 ; \textit{Comp}. 25, 131.18–132.8 ; cf. de \citealt{jonge_between_2008}: 147–149). Vu l’influence massive de la théorie stoïcienne du langage sur la théorie grammaticale à ses débuts, Barwick a essayé le premier de faire remonter l’emploi du terme en question au vocabulaire des \is{stoïciens}\text{stoïciens} (\citealt{barwick_probleme_1957}: 48, cf. \citealt{barwick_remmius_1922}: 107–108). Mais il s’agit chez Denys généralement d’une combinaison particulière des caractéristiques grammaticales et phonétiques, tandis que considérés en soi, les attributs des parties du discours devraient évoquer principalement une idée des propriétés permanentes d’une espèce. Denys semble avoir appliqué le terme \is{τὰ συμβεβηκότα}\textit{τὰ συμβεβηκότα} plutôt à tout un répertoire de procédés stylistiques qu’aux attributs des parties du discours comme tels. 

Dans les citations des œuvres des \is{stoïciens}\text{stoïciens} chez les auteurs du I\textsuperscript{e}\textsuperscript{r} jusqu’au début du III\textsuperscript{e} siècle ap. J.-C., tels Philon d’Alexandrie, Plutarque, Galien, Clément d’Alexandrie, le terme \textit{τὸ} \is{συμβεβηκός}\textit{συμβεβηκός} désigne cependant soit quelque chose d’occasionnel (\citetitle{hulser_fragmente_1987} 515, 746), soit un événement, distinct d’un corps matériel en tant que sa cause (\citetitle{hulser_fragmente_1987} 762, 695) (cf. de \citealt{jonge_between_2008}: 153). La signification du terme est assez large chez les \is{stoïciens}\text{stoïciens}, mais elle ne diffère pas essentiellement de celle que le terme a chez Aristote. Tout en leur donnant une interprétation nouvelle, les \is{stoïciens}\text{stoïciens} gardaient néanmoins le système primaire des catégories aristotéliciennes, et il est difficile d’imaginer qu’ils pussent appliquer le terme \is{τὰ συμβεβηκότα}\textit{τὰ συμβεβηκότα} aux propriétés des parties du discours. 

Tout en partageant dans une certaine mesure les critiques émises par des savants éminents comme Jan Pinborg et Dirk M. Schenkeveld, de Jonge adhère, bien qu’avec des précautions, à l’hypothèse de Barwick sur les origines stoïciennes de l’emploi du terme \is{τὰ συμβεβηκότα}\textit{τὰ συμβεβηκότα} chez Denys d’Halicarnasse (de \citealt{jonge_between_2008} : 151--154). À la suite de Richard Janko, il indique en outre l’emploi similaire du terme en question dans le traité du \is{philosophe épicurien}philosophe épicurien Philodème (vers 111–40/35 av. J.-C.) \textit{De poematibus.} Les deux savants mentionnent cet usage aussi bien dans le passage où est cité Pausimachus de Milet, qui développait la doctrine selon laquelle l’euphonie constituait en elle-même le trait distinctif (\textit{τὸ ἴδιον}) de la poésie, que dans celui où Philodème réfute le postulat restrictif défendu par Pausimachus (\citealt{janko_philodemus._2000}: 182, 268–269 (fr. 74.1--74.6), cf. 300 – 301 (fr. 94.15--94.21), \citetitle{philodemus__1976} \textit{Poem.} 22–25 (fr. 18.25–19.9) ; de \citealt{jonge_between_2008}: 155). C. de Jonge n’a cité toutefois les exemples trouvés chez Philodème que pour montrer une fois de plus l’ancienneté de l’usage du terme \is{τὰ συμβεβηκότα}\textit{τὰ συμβεβηκότα} dans le domaine des parties du discours.

Mais en fait, le savant hollandais a relevé un phénomène particulier qui est digne d’une analyse spéciale. Le terme \is{τὰ συμβεβηκότα}\textit{τὰ συμβεβηκότα} a été, de toute évidence, ancré dans le vocabulaire des \is{philosophes épicuriens}philosophes épicuriens en tant que désignation commune des propriétés aussi bien permanentes qu’occasionnelles (cf. \citetitle{epicurus_epistola_1887} 40, 6.14). Tout comme dans le cas de l’emploi synonymique du terme plus usuel \textit{τὰ παρακολουθοῦντα}, on peut cependant observer chez Épicure une tendance à désigner par ce terme surtout les propriétés stables – mais il faut noter l’habitude qu’a cet auteur de souligner expressément le caractère de stabilité de telles propriétés, par l’ajout d’une détermination spéciale indiquant leur essence physique constante : \textit{τὴν φύσιν ἀίδιον} (voir \citetitle{epicurus_epistola_1887} 71, 24.5--24.7 ; cf. 68--69, 22.13–23.9).´

On sait bien qu’Épicure (vers 342–270 av. J.-C.) négligeait les doctrines dialectiques des philosophes grecs, pour proposer une épistémologie originale fondée sur l’analyse des impressions. La terminologie d’Épicure et de ses adeptes n’avait pas grande chance de trouver place dans le lexique linguistique de l’Antiquité grecque et latine – à une exception près. Les chercheurs ont bien mis en évidence que les mêmes théories de l’euphonie intéressaient les \is{stoïciens}stoïciens et les \is{épicuriens}épicuriens, qu’il s’agissait là d’un domaine commun de leur réflexion (cf. \citealt{janko_philodemus._2000}: 181–182, 188–189 ; \citealt{campbell_philodenus_2002}: 105–109). Même si Denys d’Halicarnasse parle lui-même de ses emprunts aux théories linguistiques des stoïciens, il est fort probable que les vues de Denys sur la prosodie ainsi que celles sur les particularités des parties du discours reprennent plutôt les idées des \is{épicuriens}épicuriens, telles que nous les connaissons à travers l’œuvre de Philodème.

Barwick a signalé l’emploi du terme \is{τὰ συμβεβηκότα}\textit{τὰ συμβεβηκότα} appliqué aux propriétés des parties du discours chez l’auteur qu’on peut identifier comme le grammairen Polybe de Sardes (II\textsuperscript{e} siècle ap. J.-C. ; \citetitle{polibius_[de_sardis]_barbarismo_1867}, 286.13 ; \citealt{barwick_remmius_1922}: 97; cf. \citealt{jones_polybius_1996}). Dans le traité sur le barbarisme et le solécisme, cet auteur développe l’idée que le solécisme naît du remplacement (\textit{κατὰ ἐναλλαγὴν}) erroné d’une partie du discours par une autre, et surtout de l’interversion entre attributs de parties du discours : \textit{ἢ ὅταν \is{τὰ συμβεβηκότα}τὰ συμβεβηκότα τοῖς τοῦ λόγου μέρεσιν εἰς ἄλληλα ἐναλλάσσηται}. Cette pensée est si proche des idées de Denys d’Halicarnasse sur l’usage souple de différentes parties du discours et de leurs formes grammaticales et acoustiques, que leur fond commun est hors de doute. 

Il faut chercher ailleurs les origines de l’emploi du verbe latin au pluriel \textit{accidunt} et du participe \is{accidentia}\textit{accidentia} comme termes de la grammaire latine. On doit prendre en compte aussi le cas d’un processus long des transformations sémantiques de ces vocables dans le domaine de la \is{rhétorique}rhétorique et dans celui de la grammaire latine, autrement dit, d’un processus à peu près parallèle à l’élaboration des termes de la grammaire grecque \textit{παρέπεται} et \is{παρεπόμενα}\textit{παρεπόμενα}, que nous avons esquissé ci-dessus.

Signalons quelques faits dans l’histoire de la doctrine \is{rhétorique}rhétorique au début du Principat, qui pouvaient influencer d’une manière décisive la pensée des grammairiens latins. Il faut prendre en considération surtout la doctrine de Théodore de Gadara sur la répartition de la cause civile selon les questions principales qu’on y posait. Comme l’on sait, Théodore enseignait la \is{rhétorique}rhétorique à Rome au futur empereur Tibère dans les années 33–32 av. J.-C. Le manuel de \is{rhétorique}rhétorique qu’il a composé a été traduit en latin, comme l’atteste Quintilien (\textit{Inst. or.} II 15, 21). Or, suivant toujours le témoignage de Quintilien (III 6, 36), parmi les nombreuses questions qui pouvaient être posées dans la cause civile, Théodore mettait en avant deux questions générales, à savoir, «~le fait existe-t-il ?~» et, «~le fait étant certain, quelles en sont les particularités concomitantes ?~» : «~\textit{idem Theodorus, qui de eo an sit et de accidentibus ei, quod esse constat, id est} \textbf{\textit{περὶ οὐσίας καὶ συμβεβηκότων} existimat quaeri}.~» (\textit{Inst. or.} III 6, 36). Autrement dit, sous la notion de \is{συμβεβηκότα}\textit{συμβεβηκότα} ou d’\is{accidentia}\textit{accidentia} en traduction latine, Théodore a réuni tout ce qui était à disputer et à définir sitôt après l’établissement du fait.

D’autres maîtres grecs de la \is{rhétorique}rhétorique ont développé à leur tour une répartition de la cause civile en deux parties principales. Mais le cas du célèbre érudit romain Cornelius Celsus (première moitié du I\textsuperscript{er} siècle ap. J.-C.) est particulièrement intéressant pour nous. Quintilien nous apprend que, dans la répartition de la cause civile, Celsus partait de deux questions semblables à celles que Théodore posait jadis : \textit{Celsus Cornelius duos et ipse fecit status generales}, an sit? quale sit? \textit{Priori subiecit finitionem} (\textit{Ibid.} 38). Quant à la deuxième partie de la cause, dont parle Celsus, il la nomme tout simplement \textit{qualitas,} si l’on croit Quintilien, et la subdivise à nouveau en deux catégories : \textit{Qualitatem in rem et scriptum dividit.} Les sujets qui constituaient le contenu de la deuxième partie de la cause civile selon la répartition générale de Théodore de Gadara et Cornelius Celsus, occupent une place prépondérante dans les traités de \is{rhétorique}rhétorique que nous connaissons. Les termes \is{συμβεβηκότα}\textit{συμβεβηκότα} ou \is{accidentia}\textit{accidentia}, aussi bien que le verbe latin au pluriel \textit{accidunt}, pouvaient donc être appliqués par les \is{rhétoriciens}rhétoriciens érudits à différents sujets à traiter, y compris aux propriétés morphologiques et phonétiques des parties du discours.

Les œuvres de Denys d’Halicarnasse, contemporain de Théodore de Gadara, venu lui aussi à Rome vers 29 av. J.-C., ont donc pu subir aussi l’influence de cet usage. Mais ce sont les écrits de Pline l’Ancien (23/24–79 ap. J.-C.) qui présentent, semble-t-il, un cas particulièrement suggestif. On sait que, dans sa \textit{Naturalis historia,} Pline l’Ancien a puisé son matériau maintes fois chez Aurelius Celsus (cf. \citealt{munzer_beitrage_1897}: 41–45, 56–70). Il est vraisemblable que Pline connaissait aussi la partie de l’œuvre encyclopédique de Celsus consacrée à la \is{rhétorique}rhétorique, et qu’il s’en est inspiré dans son traité de \is{rhétorique}rhétorique \textit{Studiosus} qui n’est pas parvenu jusqu’à nous. Il est particulièrement significatif qu’on trouve un reflet possible de la terminologie de Celsus dans les fragments du traité grammatical de Pline \textit{De dubio sermone}.

Il s’agit de l’exposé des idées de Pline l’Ancien sur la forme personnelle des pronoms définis, exposé qu’on trouve chez Clédonius, éminent grammairien du V\textsuperscript{e} siècle (\citetitle{keil_grammatici_1855}, V 49.27), aussi bien que dans l’\textit{Ars Bernensis} (\citetitle{keil_grammatici_1855} Suppl. 135.1--135.8), dont l’auteur se réfère au grammairien Sergius, actif à la fin du V\textsuperscript{e} siècle et dans la première moitié du VI\textsuperscript{e} siècle. Nonobstant leur caractère tardif, ces témoignages semblent offrir un reflet assez fidèle du lexique de Pline ainsi que de sa pensée même. Suivant ces témoignages, ainsi que celui du grammairien carolingien Clément, Pline considérait la forme personnelle desdits pronoms comme inhérente à eux et non comme quelque chose d’adjacent.

\begin{quote}
    Plinius artigraphos dicentes \textbf{pronominibus} \textbf{finitis} \textbf{accidere} \textbf{personas} reprehendit. tunc enim bene diceretur, \textbf{si} \textbf{aliud} \textbf{esset} \textbf{pronomen} \textbf{finitum,} \textbf{aliud} \textbf{persona,} \textbf{non} \textbf{enim} \textbf{una} \textbf{res} \textbf{potest} \textbf{esse} \textbf{quae} \textbf{accidit} \textbf{et} \textbf{cui} \textbf{accidit, ergo melius ita dicendum est}, ait, \textit{eadem esse finita pronomina, quae sunt etiam personae} (Cledonii \textit{Ars}, \citetitle{keil_grammatici_1855} V 49.27--49.32 ; cf. \citetitle{clemens_scotus_ars_1928}, 61.8 ; \citealt{della_casa_il_1969}: 301).
    
    Sed tamen Plinius Secundus grammaticus, sicut Sergius ostendit, reprehendit eos, qui dicunt \textbf{personas} \textbf{accidere} \textbf{finitis} \textbf{pronominibus}, ut \textbf{ipsud} \textbf{\is{accidens}accidens} \textbf{aliud} \textbf{sit} \textbf{atque} \textbf{illud,} \textbf{cui} \textbf{accidit}. Hinc Plinius ipse dixit: \textit{Sed scire debemus huiusmodi definitores non tam in ratione errare quam in ordine verborum, ut dicerent} \textbf{\textit{personas pronominibus accidere, cum dicere debuissent finita pronomina non recipere quasi aliunde personas}} (\textit{Ars Bernensis}, \citetitle{keil_grammatici_1855} 135.1--135.8 ; cf. \citealt{della_casa_il_1969}: 301–302).
\end{quote}

Le verbe \textit{accido}, aux différentes formes où il est employé, conserve ici encore sa signification primitive d’une connexion occasionnelle. Il est significatif toutefois que le mot soit employé précisément à propos de l’attribut d’une partie du discours (\citealt{della_casa_il_1969}: 301). L’autorité de Pline l’Ancien a été assez forte pour donner une première impulsion aux innovations lexicales relatives aux attributs des parties du discours.

Flavius Caper, \textit{Magister Augusti Caesaris} (selon le témoignage de Pompeius : \citetitle{keil_grammatici_1855} V, 153.13), grammairien de la deuxième moitié du II\textsuperscript{e} siècle ap. J.-C., semble avoir récapitulé ces essais. Il a composé, entre autres, un traité intitulé \textit{De dubio genere}, dont le thème est très proche du traité plinien. Le traité ne nous est pas parvenu, excepté quelques citations tardives, mais l’on peut aisément présumer qu’il dépend du traité de Pline. Le fait est que tous les auteurs dont les manuels de grammaire accusent une dépendance par rapport à Caper emploient déjà le verbe au pluriel \textit{accidunt} appliqué aux attributs des parties du discours. C’est le cas surtout des grammairiens d’origine africaine, comme Palladius – connu communément sous le nom de Pseudo-Probus et actif à Rome au début du IV\textsuperscript{e} siècle –, ou comme le Pseudo-Augustin, (\citetitle{keil_grammatici_1855} IV, 522.4 \textit{et passim~}; V, 517.35; 519.27 ; 520.33 ; cf. \citealt{basset_ptota_2007}: 274 ; \citeyear{mazhuga_uber_2011}: 104). Il en va de même des \textit{Artes grammaticae} de Sacerdos, qui enseignait la grammaire à Rome à la fin du III\textsuperscript{e} siècle (\citetitle{keil_grammatici_1855} VI, 429.16 ; 442.16 ; 444.22).

Résumons nos observations. Contrairement à l’opinion reçue, il n’y a pas de raisons suffisantes pour mettre l’emploi des termes \textit{accidunt} et \is{accidentia}\textit{accidentia} dans la grammaire latine en dépendance directe de l’usage des grammairiens grecs qui référaient aux attributs des parties du discours par le verbe \textit{παρέπομαι} et le participe \is{παρεπόμενα}\textit{παρεπόμενα} avec ses formes substantivées. Il n’y a pas besoin, non plus, d’expliquer l’emploi desdits termes latins ni par l’influence de l’emploi présumé du terme \is{τὰ συμβεβηκότα}\textit{τὰ συμβεβηκότα} dans la \is{philosophie stoïcienne}philosophie stoïcienne du langage, ni par son emploi bien attesté dans l’\is{épistémologie épicurienne}épistémologie épicurienne. On doit seulement suivre la transformation du terme latin \is{accidentia}\textit{accidentia}, en partant de la signification du terme grec apparenté \is{τὰ συμβεβηκότα}\textit{τὰ συμβεβηκότα}, que celui-ci a acquise dans les théories rhétoriques au début du Principat.
%HERE

\section*{Acknowledgements}
L'auteur exprime sa profonde gratitude à Mme Sylvie Archaimbault (CNRS/Univ. Paris-Sorbonne) qui a pris en charge de relire et de corriger le texte français de cet article. Il tient également à remercier M. Jean Lallot (École Normale Supérieure, Paris) de ses suggestions importantes et de sa relecture du texte.

{\sloppy\printbibliography[heading=subbibliography,notkeyword=this]}
\end{otherlanguage}
\end{document}
