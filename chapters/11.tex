\documentclass[french,output=paper,colorlinks,citecolor=brown]{../langscibook} 
\author{Silvia Frigeni\affiliation{Sapienza Università di Roma, Université Sorbonne Nouvelle -- Paris 3}\orcid{}} 

\title{“Mithra aux vastes pâturages”: L’antropologia di Émile Benveniste}

 

\abstract{In the first part of this article we present two different ways of approaching ethnological concerns: the so-called “anthropology” of Émile Benveniste, the meaning of which will be further explained, and Antoine Meillet’s sociolinguistic point of view. In the second part, these approaches are illustrated by two comparative studies by Meillet and by Benveniste respectively. Both studies happen to bear on the same subject (the Indo-Iranian god Mitra) and both are intended to provide a perspective on the culture and life of the people whose language they scrutinize.}

\IfFileExists{../localcommands.tex}{
  % add all extra packages you need to load to this file  

\usepackage{tabularx,multicol} 
\usepackage{url} 
\urlstyle{same}

\usepackage{listings}
\lstset{basicstyle=\ttfamily,tabsize=2,breaklines=true}

\usepackage{./langsci/styles/langsci-optional}
\usepackage{./langsci/styles/langsci-lgr}
\usepackage{./langsci/styles/langsci-gb4e}

  \newcommand*{\orcid}{}
\newcommand*{\markindex}{}

  %% hyphenation points for line breaks
%% Normally, automatic hyphenation in LaTeX is very good
%% If a word is mis-hyphenated, add it to this file
%%
%% add information to TeX file before \begin{document} with:
%% %% hyphenation points for line breaks
%% Normally, automatic hyphenation in LaTeX is very good
%% If a word is mis-hyphenated, add it to this file
%%
%% add information to TeX file before \begin{document} with:
%% %% hyphenation points for line breaks
%% Normally, automatic hyphenation in LaTeX is very good
%% If a word is mis-hyphenated, add it to this file
%%
%% add information to TeX file before \begin{document} with:
%% \include{localhyphenation}
\hyphenation{
affri-ca-te
affri-ca-tes
dis-ci-plin-ary
}

\hyphenation{
affri-ca-te
affri-ca-tes
dis-ci-plin-ary
}

\hyphenation{
affri-ca-te
affri-ca-tes
dis-ci-plin-ary
}

  \bibliography{../localbibliography}
  \togglepaper[11]%%chapternumber
}{}



\begin{document}
\begin{otherlanguage}{italian}
\maketitle

\section{Due diverse prospettive: antropologia e sociolinguistica} 

\subsection{L’\textit{antropologia} di Émile Benveniste: qualche precisazione}

La necessità di presentare un articolo tramite un titolo che riassuma e illustri il contenuto, magari riuscendo a interessare il lettore, si scontra a volte con la necessità di utilizzare termini il più possibile onnicomprensivi e vaghi, magari suggestivi ma bisognosi di una precisazione. Nel caso di questo articolo, il termine “\is{antropologia}antropologia” accostato a un linguista e filologo quale fu Émile Benveniste (1902--1976) richiede senz’altro qualche spiegazione in più, che illustri tanto il senso in cui lo si intende quanto il ruolo che avrà nella trattazione. 

A partire dalla pubblicazione nel 1966 del primo volume dei \textit{Problèmes} \textit{de} \textit{linguistique} \textit{générale}, la cui quinta sezione porta il suggestivo titolo “L’homme dans la langue”, Benveniste passò da una notorietà circoscritta agli studiosi di lingue iraniche e ai filologi indoeuropeisti a un riconoscimento da parte di un pubblico ben più vasto, che comprendeva filosofi, psicanalisti, teorici del linguaggio, letterati e antropologi. La ragione era da cercare nell’interesse interdisciplinare mostrato da Benveniste negli articoli che compongono i due volumi dei \textit{Problèmes} (il secondo sarebbe uscito nel 1974, a cura di collaboratori di Benveniste che mantennero la struttura voluta dall’autore per il primo volume). Da un’altra prospettiva, legata agli studi di linguistica storica, la pubblicazione del \textit{Vocabulaire} \textit{des} \textit{institutions} \textit{indoeuropéennes} nel 1969 fu accolta con favore dagli antropologi, interessati dall’uso fatto della ricostruzione etimologica come mezzo per ricostruire i rapporti sociali, economici, religiosi e politici espressi dai termini delle istituzioni presenti nelle lingue indoeuropee.\footnote{ \textrm{Si veda il giudizio dato da Claude Lévi-Strauss in occasione del necrologio scritto per Benveniste e pubblicato sulla rivista d'antropologia} \textrm{\textit{L’Homme}}\textrm{, che lui e Benveniste avevano fondato insieme:} \textrm{“}\textrm{il n’est pas excessif de dire que son ouvrage en deux volumes sur le} \textrm{\textit{Vocabulaire} \textit{des} \textit{institutions} \textit{indoeuropéennes}} \textrm{apporte à l’anthropologie sociale une contribution d’importance majeure.}\textrm{”} \textrm{(\citealt[5]{Lévi-Strauss1976}).}}

\todo[inline]{\citet{Lévi-Strauss1976} is not included in the bibliography.}

Si intenderà quindi con “antropologia” di Benveniste la definizione che ne ha dato di recente Charles Malamoud, riferendosi al \textit{Vocabulaire} come luogo in cui i confini tra indagine linguistica e ricerca antropologica diventano più sfumati. Il riferimento è alla distinzione tra “\is{signification}signification” e “\is{désignation}désignation” che Benveniste pone nell’~“avant-propos” dell’opera, con cui l’autore vorrebbe delimitare il suo campo di ricerca rispetto a quello degli studiosi di storia e di sociologia:

\begin{quote}
    [...] étudiant des vocables ou des structures linguistiques plus complexes, Benveniste est amené à parler de ce que \textit{désignent} ces vocables et de ce que révèlent ces structures. C’est-à-dire qu’il y a ce problème qu’il a lui-même maintes fois traité et qui est un des chapitres des études benvenistiennes – des études de Benveniste et des études sur Benveniste –, à savoir le rapport entre signification et désignation.  En ce qui concerne la désignation, il y a des données, des faits, des institutions, des manières d’être, des gestes qui caractérisent à tel ou tel moment, dans tel ou tel domaine de la civilisation humaine. 

    Quand il s’agit de la désignation, c’est-à-dire ce que désignent les termes dont le linguiste s’efforce d’élucider la signification, nous sommes, me semble-t-il, dans l’anthropologie dès lors que le corpus des textes considérés parle de ce dont parlent les anthropologues: des termes de parenté, relatifs à la vie sociale, à la vie économique, à la vie religieuse. (\citealt[246]{Malamoud2016}, in corsivo nel testo)
\end{quote}

Assumendo questa propensione di Benveniste a interessarsi di ciò che effettivamente viene denotato dai termini di cui egli studia la significazione, che seguendo Malamoud si indicherà qui con il nome di “antropologia”, il presente articolo vorrebbe mostrare come questa prospettiva fosse già presente in altri lavori precedenti di \is{linguistica storica}linguistica storica. 

Ci si occuperà in particolare di un articolo in cui Benveniste offre la sua interpretazione di un epiteto riservato al dio avestico Mithra: la scelta di questo lavoro in particolare è dovuta al trattarsi di un tema parzialmente sovrapponibile a quello affrontato dal suo maestro Antoine Meillet (1866--1936), che in un saggio rimasto famoso si occupò del significato del nome del dio indoiranico Mitra. L’analisi di entrambi gli articoli cercherà di mostrare i punti di contatto e le differenze tra le due prospettive.

\subsection{Antoine Meillet e la sociologia linguistica}

L’interesse antropologico che compare nei lavori di Benveniste, così come è stato definito nel paragrafo precedente, si discosta in molti aspetti dalla prospettiva \is{sociolinguistica}sociolinguistica assunta da Meillet. Si è già sostenuto che questo interesse sociologico abbia dei tratti in comune, quando addirittura non abbia influenzato, i lavori di Benveniste e di un altro celebre allievo di Meillet, Georges Dumézil (1898--1986), che pure seguiranno strade metodologicamente diverse dal maestro (\citealt{Monod-Becquelin1988, Lincoln2012}, per citarne alcuni): l’impostazione e gli scopi sono tuttavia divergenti se non opposti. La sociolinguistica di Meillet si rifà alla sociologia di Émile Durkheim (1858--1917), da cui riprende la nozione di \textit{fatto} \textit{sociale}. Nel suo celebre articolo “Comment les mots changent de sens” \citep{Meillet1906a}, pubblicato nella rivista \textit{L’Année} \textit{Sociologique} fondata da Durkheim, afferma:

\todo[inline]{\citet{Meillet1906a} and \citet{Meillet1906b} are not included in the bibliography.}
 
\begin{quote}
    le langage est donc éminemment un fait social. En effet, il entre exactement dans la définition qu’a proposée Durkheim~; une langue existe indépendamment de chacun des individus qui la parlent, et, bien qu’elle n’ait aucune réalité en dehors de la somme de ces individus, elle est cependant, de par sa généralité, extérieure à chacun d’eux~; ce qui le montre, c’est qu’il ne dépend d’aucun d’entre eux de la changer et que toute déviation individuelle de l’usage provoque une réaction~; [...{]} Les caractères d’extériorité à l’individu et de coercition par lesquels Durkheim définit le fait social apparaissent donc dans le langage avec la dernière évidence. \citep[230]{Meillet1906a}]
\end{quote}

Sempre al 1906 risale anche un’altra menzione della lingua come fatto sociale: si tratta della lezione inaugurale del corso di Grammatica comparata del Collège de France, poi riportata nell’articolo “L’état actuel des études de linguistique générale” \citep{Meillet1906b}. Va osservato (come fa \citealt[68]{Koerner1988}) che Meillet non distingue tra \textit{langage} e \textit{langue} al modo di Ferdinand de Saussure, e usa i due termini in maniera pressoché equivalente. Soprattutto, la sua concezione sociale della lingua (o del linguaggio) si differenzia da quella del suo maestro Saussure: in Meillet, ma non in Saussure, è presente l’idea che i cambiamenti linguistici siano condizionati dai mutamenti sociali, e che quindi chi voglia studiare questi cambiamenti debba rifarsi al rapporto tra la grammatica della lingua in esame e lo stato di civilizzazione della società che utilizza quella lingua (\citealt{Koerner1988, PuehRaynski1988, Wald2012}\todo{Not in the bibliography}).

Benveniste citerà esplicitamente Meillet quando criticherà questo progetto, ormai considerato irrealizzabile (\citealt[14--15]{Benveniste1966}). Tuttavia riprende lì dove le ricerche di Meillet avevano dovuto interrompersi: già alla fine degli anni Sessanta, un suo seminario tenuto al Convegno internazionale Olivetti a Milano sarà dedicato a questo tema \citep{Benveniste1974}\todo{Nothing from 1974 in the bibliography}. Se non c’è isomorfismo tra lingua e società, argomenta Benveniste, non si può nemmeno negare il ruolo privilegiato che ha la lingua nell’indicare i cambiamenti sociali. Se “il n’y a de correspondance ni de nature ni de structure entre les éléments constitutifs de la langue et les éléments constitutifs de la société” (\citealt[93]{Benveniste1974}?, questo vuol dire che il rapporto che si cerca tra lingua e società non può essere una correlazione strutturale, ma sarà piuttosto di natura trascendentale:

\begin{quote}
    la langue représente une permanence au sein de la société qui change, une constance qui relie les activités toujours diversifiées [...] de là procède la double nature profondément paradoxale de la langue, à la fois immanente à l’individu et transcendante à la société. \citep[65]{Benveniste1974}.
\end{quote}

Il rapporto che questa dualità ha con la società è sincronico, di tipo semiologico: si tratta del rapporto dell’interpretante con l’interpretato, la lingua interpreta e contiene la società. La scoperta della base comune a lingua e società, che Benveniste aveva posto come unica condizione possibile per realizzare il piano di studi prospettato da Meillet, si risolve nella primazia della prima sulla seconda tramite una via semiotica che non era quella del maestro.\footnotemark{}
\footnotetext{ \textrm{Non si entrerà qui nel merito della distinzione, che pure bisognerebbe fare, tra le elaborazioni teoriche a cui giunge Benveniste tra gli anni Trenta e Cinquanta e quelle degli anni Sessanta, quando diventa centrale la questione della semiologia della lingua. Per quanto riguarda le possibili eredità etnolinguistiche di Meillet in Benveniste, ad es. un punto di vista semantico sulle forme grammaticali viste come traduzioni simboliche di risposte ai problemi propri di ciascuna lingua, e che permetterebbe la comparazione di lingue isolate, cf. \citet{Monod-Becquelin1988}) e \citet[117]{Benveniste1966}).}}

Due lavori pubblicati dai due studiosi a mezzo secolo di distanza sullo stesso argomento, l’interpretazione della figura del dio indoiranico Mitra, possono aiutare a illustrare questa continuità discorde. Come si vedrà, entrambi pongono dei problemi di natura sociologica e religiosa alla materia trattata, in cui lo studio delle etimologie, da cui pure partono, ha un valore marginale. Ma se pure Benveniste si pone un obiettivo che per certi aspetti va a completare il lavoro intrapreso da Meillet (l’interpretazione di una forma grammaticale il cui senso era rimasto oscuro) il metodo e il punto di vista sono necessariamente divergenti.

In questo senso andrebbe approfondito il ruolo che riveste la \is{fraseologia formulare}fraseologia formulare in entrambi i lavori. Si tratta di un compito che non si può portare a termine qui per mancanza di spazio, ma che sarebbe di grande interesse per future ricerche.
Permettendo di non fermarsi all’etimologia della singola parola, ma di comparare le frasi formulari tra diverse lingue all’interno di una stessa famiglia, la fraseologia è stato uno strumento importante per gli indoeuropeisti che però hanno spesso trascurato l’aspetto semantico delle formule considerate, preferendo concentrarsi sul solo aspetto etimologico.  Nel lamentarsi di questa mancanza, Calvert Watkins (1933--2013) ha sottolineato come Benveniste sia stato una delle rare eccezioni, dato che sarebbe riuscito a riconoscere “the function of these formulas as expressions of an underlying semiotic system” \citep[43]{Watkins1995}. L’importanza della semantica negli studi di Meillet, e la differente prospettiva semiotica e interpretativa adottata da Benveniste, potrebbero fornire uno spunto di riflessione da cui esaminare l’impiego della fraseologia così come si presenta nei due autori, e le diverse questioni teoriche e di metodo che questo comporta.

\section{Gli studi sul dio Mitra}

\subsection{“\textit{Le dieu indo-iranien Mitra}”}

L’importanza accordata alla \is{semantica}semantica, l’attenzione rivolta all’aspetto sociale delle lingue storiche, la scrupolosità dell’\is{analisi linguistica}analisi linguistica, sono tutti aspetti che avvicinano Benveniste all’insegnamento di Meillet.

Suo predecessore all’École pratique des Hautes Études e al Collège de France, fu Meillet a indirizzare il giovane Benveniste allo studio delle \il{lingue iraniche}lingue iraniche: un dominio che sarebbe rimasto il principale tra i molti trattati da Benveniste nel corso della sua carriera. Lo stesso Meillet pubblicò alcuni lavori nel settore, sia pure occupandosi quasi esclusivamente di avestico e antico persiano: vanno ricordate la \textit{Grammaire} \textit{du} \textit{vieux{}-perse}, pubblicata nel 1915 e ristampata con sostanziali modifiche a opera di Benveniste nel 1931, e le \textit{Trois} \textit{conférences} \textit{sur} \textit{les} \textit{Gâthâ} \textit{de} \textit{l’Avesta} del 1925. Ma già nel 1907 era stato pubblicato “Le dieu indo-iranien Mitra”, che sarebbe rimasto un punto di riferimento per gli studiosi successivi, e che è stato considerato l’opera in cui “the sociological and philological approaches to the study of religion earlier associated with Fustel and Max Müller first powerfully came together” \citep[13]{Lincoln2012}.

In questo breve saggio Meillet respinge l’ipotesi condivisa da diversi autori, tra cui il pioniere degli studi sulle lingue iraniche Christian Bartholomae (1855--1925), che vedeva nel dio vedico \textit{Mitra-} e nella sua controparte avestica \textit{Mithra-} una divinità solare, invocata insieme al cielo.\textbf{ }Pur riconoscendo il significato del nome comune \textit{mitrá}, codificato come ‘contratto’ nel suo \textit{Altiranisches} \textit{Wörterbuch} (1904, col. 1183), Bartholomae non aveva chiarito il nesso tra questo sostantivo e il nome proprio del dio indoiranico. Le interpretazioni successive avrebbero attribuito una connotazione morale al nome del dio, che con la sua luce protegge la verità e combatte la menzogna: il nome comune sarebbe perciò derivato da una funzione esercitata dalla divinità.\footnotemark{}
\footnotetext{Cf. \citealt{Schmidt2006}}

Rettificando questa visione, Meillet osserva che non c’è differenza tra il nome proprio del dio e i nomi comuni presenti in sanscrito (\textit{mitrá{}-} ‘amico’) e in avestico (\textit{mithra} ‘contratto’). Entrambi sono derivati da un comune termine indoiranico *\textit{mitrá-}, di cui Meillet rintraccia la radice i.e. \textit{*mei-} ‘scambiare’, presente in diverse forme sia nominali che verbali di altre lingue indoeuropee.\footnotemark{} Fanno parte di questa discendenza comune ad es. il verbo sanscrito \textit{máyate} ‘egli scambia’, ma anche termini nominali come l’antico slavo \textit{měna}, che vuol dire ‘scambio’ ma soprattutto ‘contratto’: un altro sostantivo slavo \textit{miră} ‘pace, ordine’ ha portato al russo \textit{mir} ‘comunità’, poi ‘comunità di paesani’ e quindi ‘villaggio’.

\footnotetext{ \textrm{È d’obbligo segnalare (anche se per motivi di spazio non si potrà entrare nel dettaglio) che le posizioni di Meillet sono state successivamente criticate da altri studiosi. La ricostruzione etimologica da lui fornita è solo una delle possibili e quanto all’interpretazione del nome, sono stat}\textrm{e} \textrm{proposte connotazioni che si accordassero meglio ad alcuni dei contesti dell’utilizzo, come} \textrm{‘}\textrm{alleanza, obbligo morale}\textrm{’} \textrm{(cf. ad es. \citealt{Herzferd1947}, \citealt{Brereton1981}\todo{Both not included in the bibliography}), o al carattere compassionevole del dio (come in \citealt{Lentz1964, Lentz1970} e in \citealt{Gonda1972, Gonda1973}\todo{Neither is in the bibliography}): vedi il già citato \citet{Schmidt2006}). Per una panoramica sul dibattito filologico generato dal lavoro di Meillet vedi Manfred Mayrhofer, “mitrá-”, in \citealt[354--355]{Mayrhofer1996}.}} 

Il significato di ‘amico, amicizia’ presente nel termine comune sanscrito non è quello attribuibile al nome proprio della divinità, afferma Meillet, perché “[il] ne se concilie pas avec le caractère général du dieu” (\citealt[145]{Meillet1907}): inoltre non è confermato dal corrispondente termine iranico. \textit{Mitra-} è “la personnification du contrat”, non diversamente dalle dee greche Temi e Dike per la giustizia o della dea romana Venere per la grazia femminile. Qui Meillet si rifà al suo maestro Michel Bréal (1832--1915) che aveva spiegato il ruolo delle Erinni comparandole alle αρ(Ϝ)άι, le maledizioni personificate, sopperendo così alla scarsa chiarezza etimologica del loro nome. L’etimologia non è perciò il metodo principale né il fine dell’indagine: quando rintracciabile, funge da controprova della giustezza dell’interpretazione proposta. “Les personnalités divines dont le nom est étymologiquement clair dans les langues indo-européennes sont toutes ainsi des personnifications de noms communs” \citep[145--146]{Meillet1907}.

Riprendendo il caso del dio del sole Helios così come interpretato nella \textit{Griechische Mythologie} del mitografo tedesco Otto Gruppe, pubblicata l’anno prima, Meillet chiarisce la tripartizione in cui vede definirsi il nome del dio Mitra:

\begin{quote}
    Hélios est d’abord ce que son nom indique, le soleil ; en second lieu, la puissance naturelle mystérieuse qui agit dans le soleil ; en troisième lieu, la personne dont on rapproche cette puissance naturelle, et les trois notions sont interchangeables.” De même l’indo-iranien \textit{Mitra-} est le contrat, la puissance mystique du contrat, et une personne ; et les trois notions s’interchangent constamment. \citep[146]{Meillet1907}
\end{quote}

Una tale definizione richiede un contesto in cui essere valida. Meillet passa qui alla seconda parte della sua analisi, quella in cui deve verificare “si cette doctrine rend compte de ce qu’indiquent les plus anciens documents connus, en l’espèce, les Védas et l’Avesta, sur le caractère du dieu indo-iranien \textit{Mitra-}” \citep[146]{Meillet1907}.

Per quanto riguarda i testi vedici, Mitra ha un solo inno del Rigveda (III, 59) a lui consacrato. Pur nella sua brevità, questo testo è sufficiente per Meillet a mostrare due tratti fondamentali del dio: “d’une part, il ne présente aucun trait qui indique un caractère naturaliste quelconque du dieu, et de l’autre, il est clair que \textit{Mitrá-} surveille sans sommeil les tribus humaines, et qu’on doit demeurer dans le contrat formé avec lui” \citep[147]{Meillet1907}. Assieme a Varuna, la divinità con cui compare spesso accoppiato nel pantheon vedico, Mitra è un Aditya, un guardiano dell’ordine universale: si tratta di personificazioni puramente morali, prive di natura fisica \citep[147]{Meillet1907}. 

Escluso dalle grandi religioni ufficiali, il Mithra iranico sembra appartenere a un culto antico e importante poi introdotto nel sistema del mazdeismo zoroastriano: questo culto doveva essere di origine indoiranica e non attribuibile a un prestito indiano, vista la sua scarsa importanza nei Rigveda. 
Fin dall'inizio dell’inno X dello Yasht a lui dedicato, Mithra viene presentato come colui con il quale non è possibile rompere un contratto, e che non può essere ingannato (Yasht X, 2). Qui Meillet rintraccia la sua parentela col sole: considerato in molte tradizioni indoeuropee simile a un occhio che vede tutto, il sole diventa qui l’occhio della divinità incaricata di punire la menzogna (analogamente a quanto avviene per gli Adityas nella tradizione indiana) e deve sorvegliare le azioni degli uomini. Lo Yasht X, 7, in cui si invoca Mithra “aux mille oreilles, bien fait, aux dix mille yeux, haut, à la connaissance étendue, fort, sans sommeil, éveillé”, è paragonabile al passo del Rigveda (VII, 34, 10) in cui Varuna viene definito “fort, aux mille yeux” \citep[150]{Meillet1907}.

Nel caso dell’Avesta la relazione tra il dio e il sole è meno netta: introdotto solo successivamente nel sistema zoroastriano, Mithra sorveglia in prima persona le infrazioni ai contratti. Luce che penetra ovunque e illumina ogni trasgressione, Mithra non è quindi una divinità solare ma è stata a poco a poco identificata con il sole: “étant le contrat, [il] a tous les moyens de punir les violations du contrat” \citep[153]{Meillet1907}, fino ad arrivare al carattere guerriero che contraddistingue il dio iranico rispetto alla sua controparte indiana.

Chiarito questo aspetto del dio e quindi accantonato il suo carattere naturalista, Meillet dedica la terza parte della sua trattazione a giustificare una simile conclusione. 

\begin{quote}
    On ne doit pas être surpris de voir diviniser le contrat ; car le contrat était dès le principe un acte religieux, entouré des cérémonies définies, fait avec certains rites ; et les paroles qui l’accompagnaient n’étaient pas de simples promesses individuelles ; c’étaient des formules, douées d’une force propre, et qui se retournaient, en vertu de cette force interne, contre le transgresseur éventuel. Le \textit{Mitra-} indo-iranien est à la fois le «~Contrat~» et la puissance immanente du contrat. \citep[156]{Meillet1907}
\end{quote}

Il legame con la religione e con le formule rivela più di ogni altra cosa l’importanza sociale del dio, la sua valenza antropologica, soprattutto per la cultura iranica. Mentre il Mitra vedico è rimasto poco sviluppato, la divinità iranica è diventata talmente importante e potente da imporsi alla tradizione ortodossa mazdeana, e a diffondersi presso le popolazioni che hanno subìto l’influenza iranica, come gli armeni, fino a venire stravolta nel culto mitraico dei Romani. In tutte rimane però la sua caratteristica principale, comune al tipo religioso di epoca indoeuropea: “ce n’est pas un phénomène naturel, c’est un phénomène social divinisé” \citep[159]{Meillet1907}.

\subsection{\textit{“Mithra aux vastes pâturages”}}

A differenza di Meillet, Benveniste si occuperà a più riprese del dio Mithra, privilegiandone la versione iranica. In un articolo pubblicato nel 1960 e dedicato esclusivamente a questa divinità, Benveniste non cita nemmeno il lavoro del maestro di più di cinquant’anni prima. Eppure la sua influenza è evidente fin dal principio della trattazione. Lo scopo dell’articolo è quello di chiarire il significato dell’epiteto \textit{vouru.gaoyaoitiš}, che ricorre costantemente associato a Mithra nell’Avesta e in particolare nello Yasht X a lui dedicato. Nonostante la sua frequenza debba denotare qualche carattere specifico del dio, i commentatori l’hanno tradotto ovunque come “Mithra aux vastes pâturages”, che una vecchia tradizione voleva collegato al suo ruolo del dio del sole che fa fruttificare le campagne. “Mais on ne croit plus à cette image naturiste du dieu. Cependant, le même type d’interprétation persiste dans l’exégèse moderne” è l’unico commento che faccia pensare a un riferimento di Benveniste a Meillet, sia pure velato \citep[277]{Benveniste2015}.

D’altra parte lo stesso epiteto compariva già nell’articolo del 1907. Lì Meillet si era arreso all’impossibilità di definirne la seconda parte, etimologicamente oscura, ma aveva tentato una soluzione:

\begin{quote}
il est sans doute impossible de pénétrer entièrement le sens d’un mot fixé par la tradition et qu’un long usage rituel a usé et obscurci ; mais le rapprochement des passages védiques montre que la \textit{gávyūtih} qui répond à la \textit{gaoyaoitiš} iranienne est un espace où le fidèle demande au dieu, et notamment à Mitra, de le protéger. L’épithète \textit{vourugaoyaoitiš} atteste donc le caractère indo-iranien du dieu et concorde avec le rôle qui lui est attribué ici. \citep[156]{Meillet1907}
\end{quote}

Benveniste non cita questa conclusione di Meillet, che pure va nella sua stessa direzione non naturalista. Si limita a osservare che anche gli studi più recenti persistono nell’interpretare Mithra come colui che fa scendere la pioggia, assicurando così l’acqua ai campi: lo stesso ruolo che doveva avere il dio vedico Soma, il cui titolo è il corrispettivo indiano \textit{urúgávyūti{}-}. Ma questo ruolo non può essere proprio di Mithra, visto che altri dèi mazdeani ben più legati al dominio delle acque non ricevono un tale epiteto. La sola strada da intraprendere è perciò per Benveniste lo studio del termine nella sua forma avestica e in quella vedica “selon sa forme étymologique d'abord, puis dans ses emplois textuels” \citep[278]{Benveniste2015}, la stessa percorsa da Meillet. 

La parte etimologica è liquidata in fretta da Benveniste come non problematica: “tout l’essentiel du problème est hors de l’étymologie” \citep[278]{Benveniste2015}. Si tratta quindi di esaminare i testi vedici e avestici per ottenere il contesto da cui trarre il senso del termine.

Per quanto riguarda i testi avestici, la sola occorrenza di \textit{gaoyaoiti-} non in composizione si trova nello Yasht X, consacrato a Mithra. “On peut dégager de cette strophe – l’unique exemple, rappelons-le, de \textit{gaoyaoiti} en contexte non formulaire – une conception assez précise de la notion” \citep[281]{Benveniste2015}. A chiarirne il senso è l’epiteto presente subito dopo, e occorrente anche in un altro inno, che chiarisce il sentimento di indipendenza che provano bestie e uomini quando Mithra dona loro strade larghe, profonde per la \textit{gaoyaoiti.}

\begin{quote}
    La \textit{gaoyaoiti} est cette zone de sécurité collective dont le dieu trace les accès dans les pays où il reçoit les égards dus. Qu’hommes et bétail y trouvent subsistance peut confirmer que le terme désignait d’abord un «~pâturage~», mais bien plus importante est cette connotation de la \textit{gaoyaoiti} comme lieu d’asile sous la protection de Mithra. \citep[281]{Benveniste2015}
\end{quote}

L’epiteto \textit{vouru.gaoyaoitiš} esprimerebbe quindi il vasto spazio di sicurezza che Mithra accorda a coloro che gli si affidano. Il suggerimento di Meillet viene quindi sostanzialmente confermato da Benveniste: non però tramite l’utilizzo di analisi etimologiche più avanzate o di nuove conoscenze sopraggiunte, ma grazie a un utilizzo più sistematico (e forse più spericolato) del raffronto degli impieghi testuali, fuori dall’etimologia.

Tale significato viene ribadito da Benveniste in un altro composto, l’epiteto \textit{vasō.gaoyaoiti-}, che attribuirebbe a Mithra la facoltà di dispensare la \textit{gaoyaoiti} a suo piacimento. Questo però per Benveniste non può essere tradotto come è stato fatto, attribuendo a Mithra la capacità di dispensare pascoli. Qui l’analisi di Benveniste sconfina in una breve trattazione antropologica del popolo iranico, non certo la più significativa né la più approfondita fra quelle da lui fatte, ma che mostra il possibile sconfinamento dell’analisi nel territorio della \textit{désignation} (per cui cf. \citealt[10]{Benveniste1969}):

\begin{quote}
    un fidèle mazdéen n’a jamais demandé à un dieu – ni surtout à Mithra – des pâturages ; l’espace ne manquait pas à ces tribus iraniennes des premiers âges. L’objet de leurs prières, leur plus constant souci, était double : l’eau et la sécurité. L’eau avait ses dieux, ses mythes, ses rituelles. Mais pour la sécurité, on comptait d’abord sur Mithra. \citep[282]{Benveniste2015}
\end{quote}

L’analisi dei testi vedici conferma sostanzialmente quanto Benveniste ha già affermato. Ma la controparte del Mithra iranico non è il solo Mitra, piuttosto l’entità Mitra-Varuna. Dal punto di vista del corrispettivo formale \textit{urúgávyūti-} ricorre una volta sola, ma è sostituito nella sostanza dall'espressione \textit{urvī gávyūti-}, la vasta \textit{gávyūti}, che nella tradizione vedica può essere concessa anche da altri dèi, come il già citato Soma.

Dalla ricerca del significato di un termine tramite il metodo comparativo si è arrivati così alla definizione di un ente concreto, spazialmente e temporalmente localizzabile. “La \textit{gávyūti} est, adjacent à la localité habitée, un territoire tribal, protegé par un dieu souverain, où les hommes et le bétail sont à l’abri des incursions et des calamités” \citep[284]{Benveniste2015}.

Questa sicurezza può essere concessa solo da un dio: la differenza tra il mondo vedico e quello iranico fa supporre che la sua relazione con Mithra si sia rinsaldata nella protostoria iranica. L’antichità del culto di Mithra in Iran e la sua speciale appartenenza al mondo iranico è la conclusione che accomuna Benveniste e Meillet, sia pure per vie diverse.

Anche la scarsa trasparenza etimologica del termine, ormai non riconducibile alla sua radice, viene spiegato da Benveniste in maniera non diversa da quanto affermato da Meillet: il continuo uso rituale che ne è stato fatto ha legato definitivamente questa parola alla nozione di potere divino, staccandola dalla sua origine ma allo stesso tempo rendendola rivelatrice di antiche credenze e di aspetti poco noti della figura divina di Mithra.

\section{Conclusioni}

Era stato Meillet a negare un’interpretazione naturalistica in favore di una sociale. Tuttavia il lavoro di Meillet non viene mai menzionato, in un lavoro in cui se ne menzionano altri, anche lavori con cui Benveniste non si trova d’accordo.

Una prima spiegazione può essere che in effetti per Meillet l’espressione oggetto dell’articolo rimane oscura, anche se l’intuizione esposta da Meillet (lasciata volutamente incerta perché mancante di prove sufficienti) va nella stessa direzione dell’interpretazione data da Benveniste.

Un’altra possibile spiegazione riguarda il fatto che il lavoro di Meillet sia considerato la base implicita da cui parte l’analisi di Benveniste. Il suo articolo ha segnato una svolta nel modo di concepire la figura di Mithra tale per cui non è necessario menzionarlo, basta far riferimento al fatto che non si consideri più il dio da un punto di vista naturalistico. Ma soprattutto, Benveniste non ha bisogno di menzionarlo perché nei fatti ne è erede e continuatore.

Pur nella diversità del metodo, la lingua storica rimane nell’uno come nell’altro lo strumento attraverso il quale comprendere la società che si esprime attraverso di essa. Semantica e fraseologia sono ciò che spinge Benveniste a trarre le conclusioni finali riguardanti il vero significato di \textit{gavyuti}: il significato di insieme della frase, la considerazione delle altre parole in essa presenti, la valenza sacrale delle espressioni.

Il culto del dio come dio di giustizia, la stipulazione di un contratto, il valore della mancanza di menzogna che sola permette di stipulare patti, può far comprendere perché fosse a esso demandata la creazione della sicurezza.

Dio della repressione ma anche della ricompensa verso chi lo teme, il Mithra avestico visto da Meillet e Benveniste è il dio che sostiene le fondamenta della società, intesa come luogo che si regge sull’accordo tra persone basato sull’integrità e il rispetto degli accordi. Si può perciò comprendere perché questo dio, nella sua valenza tutta particolare a lui riservata nel pantheon iranico e di differenza rispetto alla tradizione vedica, attirasse l’attenzione di Meillet e Benveniste: soprattutto quest’ultimo ci tornerà lungo tutta la sua carriera.

La presenza in Benveniste di un pensiero linguistico in cui la semantica ha un posto sempre più rilevante, dove si dà valenza al concetto antropologico di lingua intesa come espressione di un mondo, e di vita insieme, viene qui espressa in modo stringato e circoscritto a un problema linguistico, come del resto gli è proprio. Allo stesso tempo, è proprio questa coerenza metodologica di Benveniste a permettere di notare la vicinanza di questo lavoro alle sue coeve riflessioni di carattere generale sull’uomo, la lingua e la società. La vicinanza delle sue riflessioni alla lezione di Meillet, qui più che mai evidente, ci consentirà forse allora di comprenderne meglio lo sviluppo e di rintracciare nella grammatica comparata una delle sue possibili origini.


{\sloppy\printbibliography[heading=subbibliography,notkeyword=this]}
\end{otherlanguage}
\end{document}
