\documentclass[output=paper]{langsci/langscibook} 
\usepackage{tabularx}
\author{Michelle Li\affiliation{Caritas Institute of Higher Education, Hong Kong}\orcid{}} 
%\ORCIDs{}

\title{Language history from below: Pidgins and Creoles as examples}

\abstract{This paper argues for pidgins and creoles as examples of a relatively new perspective to historical sociolinguistics called “language history from below,” which focuses on the language use of ordinary people. While recent research on major European languages adopting this perspective has produced some significant results, a “from below” perspective of pidgins and creoles has not yet been explored. Focusing on Chinese Pidgin English data from various historical sources, this paper shows how its evolution, groups of users, and the linguistic variation of the preposition long exemplify the idea of “language history from below.”}

\begin{document}
\maketitle

\section{Introduction} 
One of the greatest challenges to the development of historical linguistics is the problem of bad data which “may be fragmentary, corrupted, or many times removed from the actual productions of native speakers” \citep[100]{labov_principles_1972}. Given this incompleteness of information presented by historical data, \citet[101]{labov_principles_1972} argues that linguistic research should adopt the Uniformitarian Principle, which states that “the linguistic processes taking place around us are the same as those that have operated to produce the historical record.” This principle proposes that linguistic changes that occurred in the past can be explained by observing changes in the present (on the antecedents of this principle, see \citealt{christy_uniformitarianism_1983}). A consequence of this approach is an emphasis on the use of synchronic data and quantitative methods for interpreting linguistic and social variations. In recent decades, historical linguistics has made significant progress due to technologies such as digitalization, which allows for the construction of large historical corpora. The fast-developing field of historical sociolinguistics has opened new avenues for discussion of theoretical and methodological concerns in the field, the problems and challenges of using historical data, and the value of written historical sources (\citealt{hernandez-campoy_handbook_2012}, \citealt{russi_current_2016}). 

In addition to the major languages, pidgin and creole languages also feature significantly in historical research because, as \citet[1696]{romaine_historical_2005} states “language is both a historical and social product, and must therefore be explained with reference to the historical and social forces which have shaped its use.” As the evolution and linguistic structure of \il{pidgins}pidgins and \il{creoles}creoles are products of special sociohistorical contexts, these languages are prime examples for the study of language historiography and historical sociolinguistics. 

This paper is organized as follows. \sectref{sec:7:2} introduces recent developments in historical sociolinguistics and the basic concept of “language history from below.” \sectref{sec:7:3} argues for the relevance of \il{pidgins}pidgins and \il{creoles}creoles within the “from below” approach in language history. Focusing on \il{Chinese Pidgin English}Chinese Pidgin English (CPE), spoken on the China coast, \sectref{sec:7:4} presents the major groups of \is{CPE users}CPE users. In \sectref{sec:7:5}, \is{linguistic variation}linguistic variation in the use of the \is{preposition}preposition \textit{long} in different historical sources will be examined. \sectref{sec:7:6} concludes the paper.

\section{The “from below” approach to language history} \label{sec:7:2}

The linguistic data used in historical linguistics often come from written texts. As mentioned above, there are various problems with the value and quality of such written data. \citet{hernandez-campoy_application_2012} summarize the seven major problems with data in historical sociolinguistic research as representativeness, empirical validity, invariation, authenticity, authorship, social and historical validity, and standard ideology. These problems are all related to the quality of written texts and the data derived from them. However, one should not discard the value of this medium of communication simply because of the problems and limitations it presents. Compared with speech, writing is often viewed as asynchronous and more formal. \citet[1461]{ammon_historical_1988}, however, argues that, from the point of view of a stylistic continuum, writing and reading do not necessarily represent the formal end of the continuum. Moreover, like speech, written data can also show \is{sociolinguistic variation}sociolinguistic variation and can be used to reconstruct diachronic variation, such as in her study of the linguistic variation of Middle Scots \citep{romaine_socio-historical_1982}. The impression that written language represents the educated elite, scholarly literature and standard language is partly due to historical linguists’ traditional bias towards the use of data from the most formal end of writing styles. \citet{milroy_consequences_1999}, for example, shows that, from approximately the 16th century onward, research on the history of English has almost always referred to the \is{standard variety}standard variety. 

The notion of “from below” was first applied in the discipline of history \citep{Thompson1966}, focusing on points of view or accounts offered by ordinary or common people rather than scholars or the higher social classes. A parallel approach called “language history from below” is developing in historical linguistics. This new interest in the language use of the lower social strata is partly due to the historical reliance on the \is{standard variety}standard variety of a language for linguistic data. \citet[198]{bakker_types_1997} points out that the traditional view of language history has favored the elaborate, socially exclusive and distance-orientated variety, usually the standard, which has resulted in only partial descriptions of the history of a language. 
 
 The “from below” approach highlights two sources of linguistic data: (i) language use by ordinary people and (ii) selection of texts showing speech or vernacular features \citep{elspas_everyday_2007}. Although who or what is considered “below” may vary from study to study, the term in general refers to ordinary people \citep{hailwood_who_2013}, who are important to study because while constituting the majority of any population of language users, their language use is often neglected \citep{elspas_everyday_2007}. The types of materials selected for examination are mainly texts with a resemblance to speech, including ego-documents such as personal letters, diaries, autobiographies, and catechisms, rather than literary works or formal documents. In discussing the features of different types of written texts, \citet{hernandez-campoy_use_2012} devises a continuum and argues that at one end lie texts of “distance” such as legal documents, whose language is mainly literate, formal, and planned, and at the other end are texts of “immediacy,” which have an oral, informal, and unplanned nature similar to speech. Other types of ego-documents showing varying degrees of vernacular features include trial proceedings, diaries, travelogues, drama texts, sermons, and proclamations (\citealt{van_der_wal_touching_2013}). Ego-documents have been employed as sources in the study of the grammar and spelling of private emigrant letters written by Germans (\citealt{elspas_everyday_2007}, \citealt{elspas_germanic_2007}) and letters written in seventeenth-century Dutch \citep{nobels_etraordinary_2013}.

\section{Pidgins and creoles as examples of “language history from below”}
\label{sec:7:3}

The formation of \il{pidgins}pidgins and \il{creoles}creoles was mainly a solution to the immediate communication barrier between indigenous people and Europeans under different contact circumstances. While many languages undergo \is{standardization}standardization at different levels of language use, the formation of pidgins and creoles is the result of negotiations and compromises on the spot among the groups of users concerned. Therefore, standardized usage was not the principal concern of the users. So far, works within the “language history from below” approach have mainly focused on major languages like English, German, and Dutch. In the following, I will argue that pidgins and creoles are ideal examples of languages of the common people. 

From the 16th century onward, the European trading and political conquests in Africa and Asia created an environment where intercultural communication was commonplace. These languages of mixed sources are often depicted as marginal, makeshift, baby talk, or corrupted and broken versions of their European superstrates. Their use was mainly restricted to informal domains, and the speakers were mainly those of a lower social class. As such, few have cared about \is{standardizing}standardizing pidgins and creoles, but it is exactly this lack of attention that gives them room for innovation. The “creole continuum” (\citealt{stewart_urban_1965}, \citealt{bickerton_dynamics_1975}), where we find different varieties of a \il{creole}creole, ranging from the least to the most similar to a standard, is a manifestation of the richness of linguistic and social \is{variation}variation of contact languages. In sum, we can see that the concept of “language history from below” can be applied both socially and linguistically to understand the history of \is{pidgins}pidgins and creoles in language contact settings. 

Pidgins are used for restricted communication, such as trade, and thus have limited vocabulary and employ simpler grammatical structure \citep[5]{holm_introduction_2000}. One of the best-known examples of pidgin is Chinese Pidgin English (\il{CPE}CPE), which was initially used as a lingua franca for trading first in Canton and Macau in the 18th century and later in other treaty ports in China after the opening of foreign trade in the 19th; it gradually declined in the first half of the 20th century (\citealt{michaelis_chinese_2013}). The vocabulary of \il{CPE}CPE consists mainly of English words with some borrowings from Portuguese, Malay, and Chinese; grammatically, it shows substantial influence from Cantonese, the local Chinese dialect spoken in Canton (\citealt{ansaldo_china_2010}). Social and political circumstances also played a part in the development of CPE. Officially, foreigners were prohibited from learning the Chinese language, and there was mutual hostility between the Chinese and foreigners \citep{baker_off_1990}. These factors provided a space for \il{CPE}CPE to develop as a lingua franca for interethnic communication. While negative views on pidgins and creoles abound, praises of their usefulness such as the one on CPE given by the old China hand William Hunter are harder to find.

\begin{quote}
    In the Canton book-shops near the Factories was sold a small pamphlet, called ‘Devil’s Talk’…I have often wondered who the man was who first reduced the ‘outlandish tongue’ to a current language. Red candles should be burnt on altars erected to his memory and oblations of tea poured out before his image, placed among the wooden gods which in temples surround the shrine of a deified man of letters. \citep[63]{hunter_fan_1882}  
\end{quote}

Carl Crow, a long-term resident in China, also recognized the effectiveness of CPE as a means of communication:

\begin{quote}
    In no language could the conversation have been any more definite and in none could fewer words have been used. In spite of its barbarous nature pidgin was quite sufficient for all commercial transactions and in default of any other means of communication could cover a wide range of subjects. \citep[31]{crow_foreign_2011}
\end{quote}

As with studying other historical languages, data on CPE are small and fragmentary, though they come from a wide variety of sources. CPE recorded in the English language appears in diverse sources, including travelogues, journals, memoirs, newspaper and magazine articles, guide books, etc.

In Chinese language sources, which are mainly instructional materials, \il{CPE}CPE is transcribed in Chinese characters. Various versions of anonymously written so-called redhaired phrasebooks containing a vocabulary of about several hundred essential words or phrases in pidgin English, like the one mentioned in \citet{hunter_fan_1882}, were sold in Canton \citep{bolton_chinese_2003}. It is clear that these coarsely made, cheap phrasebooks were prepared by and for ordinary people. A notable instructional manual source is called \textit{The Chinese and English Instructor} {\cjkfont 英語集全} (1862). Note that the “Chinese” in the title refers to the local dialect, Cantonese instead of Mandarin. Apart from Cantonese and English, what makes the book a valuable source is the inclusion of pidgin equivalents of Cantonese and English dialogues in some parts of the book. Due to the comprehensiveness of its content and the systematicity of the transcription, this source is of great importance to the grammatical analysis of CPE. The compiler of the \textit{Instructor} was a prominent Cantonese merchant called Tong Ting-kü {\cjkfont 唐廷樞} [1832--1892]. In traditional Chinese society, the merchant class was considered the lowest in the Chinese social hierarchy, and many of them had no or limited formal education. Tong was an exception, having attended the Morrison Mission School in Hong Kong, and was highly praised for his proficiency in English. Most of the CPE examples from the Chinese-language sources are taken from this book.

\section{The \is{users of CPE}users of CPE}

Accounts from historical sources indicate that the users of CPE, Chinese and foreigners alike, came from the non-elite, lower social class, thus were people “from below.” This section will look at the major groups of users constituting the speech community of CPE. 

\subsection{European traders and other foreigners}

The foreign community in the old Canton city was multinational, so the immediate challenge was to find a common language for communication. Economic, social, and political factors contributed to the rise of CPE as the lingua franca. The need to be familiar with CPE was evident in John Robert Morrison’s [17 April 1814--29 August 1843] book \textit{A Chinese Commercial Guide Consisting of a Collection of Details Respecting Foreign Trade in China} (\citeyear{morrison_chinese_1834}) where a “glossary of words and phrases peculiar to the jargon spoken at Canton” was found. The jargon mentioned here referred to the Canton pidgin. Some examples in the glossary are as follows.

\ea 
    \label{ex:7:1}
    \textit{Can do?} ‘Will it do?’\\
    \textit{Chop-chop} ‘quick, fast’, as \textit{too muchy chop-chop} for ‘very quick’
\z

In the 19th century, the function of CPE expanded as families of foreign traders were allowed to reside in China and there was need to hire Chinese servants to help them in their housekeeping. The dialogue in \REF{ex:7:2} is a communication between a European master and a Chinese boy. 

\ea \label{ex:7:2}
Boy: \textit{You makee ling?} ‘Did you ring, sir?’\\
Master: \textit{Yes, sendee catchee one piecee tailor man.} ‘Yes, send for a tailor.’\\
Boy: \textit{Just now have got bottom side.} ‘He is below at present.’\\
Master: \textit{Show he come top side.} ‘Tell him to come up.’\\

\citep[43]{anonymous_englishman_1860}
\z

\subsection{Chinese}

\subsubsection{\textit{Merchants and compradors}}

As Chinese government officials did not trade directly with foreign traders, merchants and compradors functioned as their intermediaries. In order to perform their duties, “[t]he comprador[s] … speak a broken English mixed up with Portuguese, some Dutch, and French, the same as most of the Chinaman who come about the ship. It is rather difficult to understand them at first, but one soon gets used to hearing them”\citep[29]{tyng_before_1999}

Though of the lowest class, a merchant could become rich and influential through his familiarity with Westerners and Western cultures; “through his expertise in pidgin English and his knowledge of the West, he became a middleman between East and West, not only economically but also socially, politically, and culturally” \citep[180]{thao_du_1970}. A good example is the author of the \textit{Chinese and English Instructor}, Tong Ting-kü, who was highly respected by the Chinese as well as Westerners.

\subsubsection{\textit{Interpreters}}

Another group of Chinese who had close connection with the foreign community were interpreters, also referred to as linguists in the 18th and 19th centuries (\citealt[50]{hunter_fan_1882}, \citealt{van_dyke_canton_2005}). Like merchants, interpreters were indispensable partners of foreign traders. Linguists were appointed by the Hoppo (administrator of Canton Customs) to act as interpreters. Apart from providing translation services for the Chinese and Westerners, their duties and functions were various, including monitoring the daily activities of the foreigners \citep{barreto_interpreters_2016}. In the early 18th century, as many interpreters came from Macao, they tended to also speak (pidgin) Portuguese. The example in \REF{ex:7:3} is a linguist’s translation of a speech in Chinese to “China Anglish.” 

\ea \label{ex:7:3}
    \textit{Yeckhing, chin chin the gentleman alla proper. Yeckhing very much oblige to soupcarg, who have wantchee buy him cargo pigeon. He chin chin gentlemen good voyagee, hopee go home his country No.1 good, and catchee many per cent.}\\
    \glt ‘Yeckhing thanks the gentlemen assembled, for their polite attention, and is very much obliged to the supercargo who bought his chop; he wishes him a pleasant return voyage, and hopes that he will derive a handsome profit on the purchase.’ (original translation, \citealt[126]{tiffany_canton_1849})
\z

\subsubsection{\textit{Domestic servants}}

As mentioned above, the growth of the foreign community in China resulted in more Chinese being employed in foreign households. \citet[309]{zhang_language_2009} shows that, in early Hong Kong, typical European families employed from three to more than ten Chinese servants. This means that not only was there an increase in the number of \is{users of CPE}users of CPE, more importantly, the expansion in domains of use resulted in an elaboration of the pidgin’s vocabulary and grammar.

\subsubsection{\textit{Other users}}

The foreign community’s activities were not restricted to the office and home; they were also in contact with people like tailors, barbers, shopkeepers, prostitutes, etc. who provided various services and entertainments to them. Shops could be found along the Old China Street and Hog Lane adjoining the Thirteen Factories in Canton. Example \REF{ex:7:4} shows a typical exchange between a shopkeeper and a European.

\ea
    \label{ex:7:4}
European:  How you do, Hipqua?

Chinese:    Welly wen, tankee; how you do?

European:  I well. What have got?

Chinese:    Anyting have got. What ting wantyee?

European:  I no sabee; lettee my see something. How muchee this 

cigar-boxee?

Chinese:    Oh! dat cigar-boxee! dat tree quart dollar.

European:  Too muchee. More cheap have got?

Chinese:    No; more cheap no got.

(\citealt[301]{duer_pigeon-english_1860})
\z

\section{Linguistic variation in the preposition \textit{long}}

As speakers of \il{pidgins}pidgins and \il{creoles}creoles come from different cultural and linguistic backgrounds, linguistic variation is common. The \is{preposition}preposition \textit{long} in \il{CPE}CPE demonstrates linguistic variation at different levels. \citet{li_origins_2011} shows that the functions and meanings of \textit{long} are a conflation of its source languages: \textit{along (with)} in English and \textit{tung4} in Cantonese. The double etymology of \textit{long} is evident in the following linguistic variation:

\begin{enumerate}
\renewcommand{\labelenumi}{\alph{enumi})}
\item Syntactic variation: The word order of the \textit{long-}prepositional phrase shows both preverbal and postverbal positions; and 
\item Semantic variation: The range of meaning covered by \textit{long}, namely ‘with, for, from’, is attributable to both English and Cantonese, and there is a close relationship between the meaning of \textit{long} and its syntactic position in the sentence.
\end{enumerate}

In Cantonese, the preposition \textit{tung4} is placed before the verb. To illustrate how the word order of \textit{long} varies in the sources, take the most productive meaning ‘with’ as an example. In the CPE data, \textit{long-}prepositional phrases can be found after the verb phrase \REF{ex:7:5}, as in English, or before it \REF{ex:7:6}, as in Cantonese \textit{tung4}. Compare (5’) and (6’), which are the Cantonese translations of \REF{ex:7:5} and \REF{ex:7:6}, respectively. It is clear that the postverbal placement of \textit{long} does not come from Cantonese, but is modelled on English. 

\ea%5
    \label{ex:7:5}
\textit{I like werry much, do littee pidgeon [long you].} (\citealt{downing_fan-qui_1838}: I. 279)\\
\glt ‘I would very much like do some business with you.’ 

\ex%6
    \label{ex:7:6}
\textit{he [long one gentleman] talkee} (\citealt{tong_chinese_1862}: VI.39)\\
\glt ‘He is talking with a gentleman’ 

\ex (Cantonese)\\
\gll ngo5  hou2  soeng2   [tung4   nei5]  zou6  saang1ji3\\
1    very   want    with      2    do   business\\
\ex (Cantonese)\\
keoi5  [tung4    haak3jan4]  gong2   gan2   je5\\
3  with        guest    talk   \textsc{asp}  thing
\z

The meanings of \textit{long} also recombine the meanings in the source languages. In addition to the comitative usage shown above, other meanings of \textit{long} are used to indicate source, benefactive, and malefactive. The occurrences of each meaning in Chinese- and English-language sources are summarized in \tabref{tab:7:1}.

\begin{table} 
\begin{tabular}{llccc}
\lsptoprule
Meaning & Lang. of source & \multicolumn{2}{c}{Word order of \textit{long}} & Subtotal\\\cmidrule(lr){3-4}
&  & Preverbal \textit{long} & Postverbal \textit{long} & \\\midrule
Comitative & English & 2 & 18 & 20\\
& Chinese & 10 & 8 & 18\\
Source & English & 0 & 0 & 0\\
& Chinese & 7 & 0 & 7\\
Benefactive & English & 0 & 0 & 0\\
& Chinese & 10 & 0 & 10\\
Malefactive & English & 0 & 5 & 5\\
& Chinese & 0 & 0 & 0\\
\midrule
Total &  & 29 & 31 & 60\\
\lspbottomrule
\end{tabular}
\caption{\label{tab:7:1}Distribution of the meaning and syntax of the \is{preposition}preposition \textit{long} in Chinese Pidgin English \citep{li_origins_2011}.}
\end{table}

The comitative use of \textit{long} conflates English \textit{along} (\textit{with}) and Cantonese \textit{tung4}. As both share this meaning, this may explain the high frequency of this use. Other uses of \textit{long} contributed by Cantonese \textit{tung4} include source and benefactive, as shown in examples \REF{ex:7:7} and \REF{ex:7:8}, respectively. The malefactive meaning in \REF{ex:7:9} seems to come from one use of \textit{for} in English \citep{li_origins_2011}. 

\ea%7
    \label{ex:7:7}
Source\\
my [long you] takee some (\citealt{tong_chinese_1862}: VI.12)\\
\glt ‘I will buy some from you’

\ex%8
    \label{ex:7:8}
Benefactive\\
my [long you] catchee one piecee (\citealt{tong_chinese_1862}: VI.26)\\
\glt ‘I will get one for you’ 

\ex Malefactive\\
my too much fear some war ship mans want for make bobbily [long china mans]. (\citealt[968]{tilden_journal._1834})\\
\glt ‘I fear very much that the sailors want to make troubles for the Chinese.’
\z
    
A closer look at \tabref{tab:7:1} suggests a close relationship between the semantics, syntax, and source of attestation of \textit{long}. For example, though both Cantonese and English contribute to the comitative meaning, there is an overwhelming preference (90\%) for the postverbal word order in the English language sources, which suggests that the writers’ language might have come into play. The two instances of a preverbal \textit{long} in English-language sources and the more or less balanced distribution of different word orders of \textit{long} in Chinese-language sources may be due to varying degrees of acculturation. The source and benefactive meanings are exclusively attested in Chinese-language sources and syntactically in the preverbal position, demonstrating a clear case of the influence of the substrate language, Cantonese. 

In sum, the \is{preposition}preposition \textit{long} in \is{CPE}CPE shows the following features: first, the form and function of \textit{long} show different levels of recombination of features from English and Cantonese grammar (see \citealt{li_origins_2011} for a detailed account of this conflation of features); second, some uses of \textit{long} show a close relationship between semantics and syntax; and third, the presence or absence of certain uses of \textit{long} in certain types of sources may be related to the linguistic backgrounds or acculturation of the writers.

\section{Conclusion}

Recent studies on the historiography of major languages have laid an important foundation for the development of theories and methods for investigating language history. The study of \il{pidgins}pidgins and \il{creoles}creoles offers language historians additional perspectives for understanding languages evolution in intense and rapid contact situations. Using Chinese Pidgin English as an example, this paper shows that contact languages are good examples of “language history from below” because, unlike \is{standard varieties}standard varieties, pidgins and creoles developed without much human intervention and were subject to little pressure of \is{standardization}standardization. As a result, \is{language variation}language variation instead of homogenization seems to be the norm. An example of such \is{variation}variation in language use has been demonstrated with the \is{preposition}preposition \textit{long} in \is{CPE}CPE. Recent developments in historical linguistics, especially historical sociolinguistics, have given rise to significant advances in theories and methodologies. A shift of focus from the study of higher register and \is{standard varieties}standard varieties to that of the language use of ordinary people, such as speakers of pidgins and creoles, can provide new insights for the “from below” approach to language history.


% % \section*{Acknowledgements}


{\sloppy\printbibliography[heading=subbibliography,notkeyword=this]}
\end{document}
