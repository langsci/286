\documentclass[french,output=paper,colorlinks,citecolor=brown]{../langscibook} 
\author{Gabriel Bergounioux\affiliation{Université d’Orléans/LLL – UMR 7270}\orcid{}} 
\title{La \textit{Revue des Patois Gallo-Romans} (1887--1892) et la représentation de l’oral}


\abstract{La \textit{Revue des Patois Gallo-Romans}, apparue cinq ans après les lois Jules Ferry, a été confrontée aux questions que pose la description de l’oral dans un pays de culture écrite. Le choix d’un système de transcription, la définition des qualités attendues de l’enquêteur ou le mode de recrutement des locuteurs ont donné lieu à des interprétations variées. L’étude de l’ensemble des contributeurs permet d’observer, dans le contexte scientifique et social de l’époque, la façon dont les dialectologues ont concilié les conceptions qui sous-tendaient le programme de francisation scolaire et la réévaluation du statut des dialectes qu’a réalisée le comparatisme.}

\IfFileExists{../localcommands.tex}{
  % add all extra packages you need to load to this file  

\usepackage{tabularx,multicol} 
\usepackage{url} 
\urlstyle{same}

\usepackage{listings}
\lstset{basicstyle=\ttfamily,tabsize=2,breaklines=true}

\usepackage{./langsci/styles/langsci-optional}
\usepackage{./langsci/styles/langsci-lgr}
\usepackage{./langsci/styles/langsci-gb4e}

  \newcommand*{\orcid}{}
\newcommand*{\markindex}{}

  %% hyphenation points for line breaks
%% Normally, automatic hyphenation in LaTeX is very good
%% If a word is mis-hyphenated, add it to this file
%%
%% add information to TeX file before \begin{document} with:
%% %% hyphenation points for line breaks
%% Normally, automatic hyphenation in LaTeX is very good
%% If a word is mis-hyphenated, add it to this file
%%
%% add information to TeX file before \begin{document} with:
%% %% hyphenation points for line breaks
%% Normally, automatic hyphenation in LaTeX is very good
%% If a word is mis-hyphenated, add it to this file
%%
%% add information to TeX file before \begin{document} with:
%% \include{localhyphenation}
\hyphenation{
affri-ca-te
affri-ca-tes
dis-ci-plin-ary
}

\hyphenation{
affri-ca-te
affri-ca-tes
dis-ci-plin-ary
}

\hyphenation{
affri-ca-te
affri-ca-tes
dis-ci-plin-ary
}

  \bibliography{../localbibliography}
  \togglepaper[10]%%chapternumber
}{}


\begin{document}
\begin{otherlanguage}{french}
\maketitle

\section*{Introduction} 

En France, les revues ont constitué, de 1859 à 1970 environ, un observatoire privilégié pour comprendre la façon dont s’est effectué le déploiement scientifique de la linguistique (\citealt{Bergounioux1984, ChevalierEncrevé2006}). Dans la deuxième moitié du XIX\textsuperscript{e} siècle, elles ont constitué une préfiguration \citep{Bergounioux1997} ou une alternative, souvent conflictuelle, à l’organisation de l’enseignement supérieur dans ce domaine (\citealt{Desmet1996}, \citealt{Bergounioux2002}).

La \textit{Revue des Patois Gallo-Romans} (désormais \textit{RPGR}) constitue cependant un cas particulier car la dialectologie, dans son état contemporain, ne s’était pas implantée dans les universités. Les cours ouverts dans les facultés des lettres de province étaient consacrés aux états anciens \citep{Bergounioux1984}. La seule reconnaissance institutionnelle, au sein de la IV\textsuperscript{e} section de l’École Pratique des Hautes Études, était la direction d’études confiée à Jules Gilliéron en 1887. Après avoir rappelé le contexte politique et l’historique des travaux sur les «~patois~», on considérera la \textit{RPGR} en tant qu’elle a fourni des éléments de réponse à une question centrale pour la recherche : quelle doit être la représentation de l’\is{oral}oral et comment doit être comprise la solution proposée par celui qui en a conçu le système de \is{transcription}transcription, l’abbé P.-J. Rousselot?

\section{Une situation particulière}

Si l’on observe la situation il y a cent cinquante ans, la France, dans les frontières qui étaient les siennes et qui n’ont guère changé, présente un cas de figure unique en Europe. Pas de revendication d’indépendance de ses territoires comme, à la même époque, en Norvège (par rapport à la Suède) ou en Irlande (face au Royaume-Uni) ; pas d’objectif de rattachement des francophones de Belgique ou de Suisse alors que l’Allemagne, l’Italie ou la Grèce affirmaient leur volonté de fédérer en une seule entité toutes les contrées où une majorité de la population se référait, pour la représentation officielle de son parler, à la même langue écrite. Comme en Autriche-Hongrie, dans l’Empire Ottoman et en Russie, le pays comptait un très grand nombre de locuteurs dont la langue maternelle n’était pas celle qui avait le monopole de l’enseignement et de l’administration mais, à la différence de ces trois états, la France est demeurée unitaire et monolingue.

Une explication s’impose : à défaut d’être francophone, la France a été francisée. Le processus a duré des siècles (\citealt{Brunot1905, BoyerGardy2001, KremnitzBroudic2013},) sans soulever de résistances significatives fondées sur l’identité linguistique. Les lois scolaires de Jules Ferry en 1881--1882 ont précipité un processus qui avait commencé dès le Moyen-Age dans un contexte que la guerre de 1870 avait définitivement transformé. 

Opérant un renversement symbolique, le romantisme avait conféré une dignité aux «~arts et traditions populaires~». La reconnaissance par les sciences historiques de strates de peuplement et de langues littéraires différenciées supplantait la condescendance pour des parlers frustes et désuets. Ceux-ci représentaient l’expression authentique des populations rurales, le réceptacle des traditions, du \textit{folklore} (un mot emprunté à l’anglais en 1885). Témoignages de la permanence d’une population d’origine celtique sur le même territoire, ils appartenaient de plein droit au patrimoine de la nation. Pourtant, ils restaient associés à l’obscurantisme et apparaissaient comme de possibles ferments de fédéralisme voire de séparatisme après la signature du traité de Francfort le 10 mai 1871. 

La réponse des romanistes parisiens passait par la confusion de l’ensemble des parlers métropolitains issus du latin en une seule et même langue dont la forme littéraire proviendrait de la transposition écrite du «~francien~» \citep{AurouxEtAl1996}. Les variétés écartées de l’usage officiel se trouvaient par suite ravalées au rang de «~patois~» – un terme en concurrence avec \textit{jargon} et autres vocables dépréciatifs –, une dénomination étendue au breton, au basque et au flamand, assimilés par leur statut aux \is{dialectes}dialectes romans dont ils partageaient le décri.

\section{La dialectologie : rétrospective et méthodologie}
 
 Dans le prolongement d’un engouement plus empathique que savant qui recoupait le travail entrepris à l’École des Chartes (créée en 1821) à partir des archives et des manuscrits, les premières études avaient abouti à la confection de dictionnaires comme le \textit{Lexique roman} \citeyear{Raynouard1838} de F. Raynouard et le \textit{Glossaire} de H. \citet{Jaubert1856} qui préfiguraient un intérêt renouvelé pour les langues régionales et les formes vernaculaires. Après l’inventaire des sources scripturales sous l’autorité de la commission des travaux historiques de l’Académie des Inscriptions et Belles-Lettres en 1834, les missions de terrain du ministère de l’instruction publique avaient assuré la collecte de données ethnographiques (chansons, proverbes…) dans les années 1850. Parallèlement, en Suisse et en Belgique, des monographies affirmaient la spécificité de la Wallonie et du pays romand. Les enquêtes d’O. Bringuier et de Ch. de Tourtoulon \citeyear{TourtoulonBringuier1876} et la contre-enquête d’A. \citet{Thomas1879} ont relancé la discussion sur les frontières linguistiques internes de la France romane (\citealt{Brun-Trigaud1990}).

La dialectologie s’est constituée méthodologiquement au point de rencontre entre trois techniques : l’enquête de terrain, la projection cartographique et la \is{transcription}transcription. La constitution des données au moyen d’enquêtes est empruntée aux sciences sociales et, du fait des populations étudiées, la recherche s’est avérée plus proche de l’ethnologie que de la sociologie, une orientation déterminée par le peu de sources écrites disponibles sur le monde rural. La cartographie, reprise à la géographie (Gilliéron présentera une partie de ses travaux sous l’intitulé de «~géographie linguistique~»), accompagne le développement de cette discipline et figure en synchronie des variations que la grammaire comparée traitait dans leur profondeur diachronique. 

La représentation des matériaux sonores pose la question de la notation des langues. Il y a un lien entre l’étude sur le terrain et la phonétique de laboratoire, deux disciplines antinomiques par leurs pratiques mais qui partagent une même défiance à l’encontre de la philologie, de la lettre et des textes même si le programme de la \textit{RPGR} aboutissait à restituer des usages parlés sous la forme de documents écrits. 

\section{Les auteurs de la RPGR} 

La \textit{RPGR} paraît la même année que la \textit{Revue des Patois} de Léon Clédat (1851--1930). Elle a pour directeurs Rousselot et Gilliéron. Elle comporte cinq tomes, publiés de 1887 à 1892 (pas de parution en 1889) et totalise 1666 pages. Dans le titre, l’idée d’un substrat gaulois – «~\il{gallo-roman}gallo-roman~» –, est secondaire, seule comptant la réunion en un seul ensemble linguistique des patois «~(…) appartenant comme elle [notre langue littéraire] au latin vulgaire qui est parlé dans les Gaules depuis la conquête romaine~» (\citealt[1]{Rousselot1887}). 

Pierre-Jean Rousselot (1846--1924) est prêtre. Il a commencé par enseigner la phonétique à l’École des Carmes (l’Institut Catholique de Paris) où il a installé un laboratoire transféré au Collège de France en 1897 auprès de Michel Bréal. «~Préparateur~» trente années durant, il est élu à soixante-dix-sept ans dans une chaire de phonétique. Il a soutenu son doctorat en 1891 et l’édition de sa thèse occupe une part importante des livraisons de la \textit{RPGR}. 

Jules Gilliéron (1854--1926) a soutenu son doctorat en 1880 sur le \textit{Patois de la commune de Vionnaz (Bas-Valais)} complété l’année suivante par un atlas \citep{Gilliéron1881}. Il est nommé chargé de conférences à l’EPHE en 1883 et, trente ans après sa naturalisation prononcée en 1886, directeur d’études en 1916. Ses contributions à la \textit{RPGR}, une soixantaine de pages, sont consacrées pour l’essentiel à l’exploitation d’informations tirées d’enquêtes de terrain menées en Savoie. 

Au Collège de France, les soutiennent Henri d’Arbois de Jubainville (1827--1910), premier titulaire de la chaire de «~langue et littérature celtiques~» qui confie à la revue son étude sur «~La langue latine en Gaule~» \citeyear[161--171]{Arbois-de-Jubainville1887} et Gaston Paris (1839--1903) à qui la \textit{RPGR} est dédicacée : «~A M. Gaston Paris, hommage respectueux et reconnaissant de ses élèves (Gilliéron et Rousselot) ». Paris apparaît une seule fois au sommaire, à l’occasion de son allocution solennelle devant le congrès des sociétés savantes en 1888 sur «~Les parlers de France~» \citeyear[161--175]{Paris1888}. Sa présence ne fait que mieux ressortir l’absence de Paul Meyer (1840--1917), le titulaire de la chaire de «~Langues et littératures de l’Europe méridionale~» (comprenant l’occitan) de 1876 à 1906.

Interviennent aussi sur des questions de portée générale Jean Psichari (1854--1929) (à propos de la dialectalisation du grec moderne), Louis Gauchat (1866--1942) (pour une appréciation critique du livre de D. Schindler sur le vocalisme du patois de Sornetan) et Eduard Koschwitz (1851--1904) (sur l’interprétation phonétique de graphies médiévales), trois personnalités étrangères (ou d’origine étrangère) pour des questions qui se posent en dehors des frontières nationales et de l’état actuel des langues.

Au nombre des enquêteurs, le premier, Edmond Edmont (1849--1926), est le futur collaborateur de Gilliéron pour l’\textit{Atlas Linguistique de la France} (1902--1910). Son \textit{Lexique saint-polois} (\citeyear{Edmont1887, Edmont1897}) est publié en fascicules qui représentent près d’un quart de la collection complète de la revue (385 p.). A côté de cette monumentale étude sur une localité du Pas-de-Calais, on relève~les interventions de :\todo[inline]{You can't have colons before tables.}

% \begin{table}
\begin{tabular}{lrl}
% \lsptoprule
Abbé Rabiet  &   137 p.   &  Côte-d’Or \\
Abbé Fourgeaud  &   61 p.  &  Charente \\
A. Doutrepont  &     61 p.  &  Wallonie \\
Ch. Roussey   &   28 p.  &  Doubs \\
M. Camélat  &    25 p.  &  Basses-Pyrénées \\
P. Marchot  &    24 p.  &  Wallonie \\
J. Passy    &  24 p.  &  Hautes-Pyrénées \\
Abbé Devaux    &   11 p.  &  Isère  \\
G. Doncieux   &   11 p.  &  Isère \\
  \lspbottomrule
 \end{tabular} 
%  \caption{\color{red}Please provide a caption}
%  \end{table}
 
Seul Jean Passy traite d’un parler qui n’est pas celui de la région où il a grandi.

D’autres participants ont rédigé entre une et dix pages, tels G. Dottin, A. Jeanroy, P. Lejay, F. Nougaret, P. Passy et une douzaine d’abbés. Les autres auteurs étrangers sont, pour la Belgique, A. Horning et M. Wilmotte et, en Suisse, H. Morf (sur les Grisons).

\section{Terrains et objets}

Cent huit contributions correspondent à des études territoriales sur l’aire \il{gallo-romane}gallo-romane. La Wallonie a fourni une soixantaine de pages (dix articles de Wilmotte, Marchot, Doutrepont) et la Suisse romande, la patrie de Gilliéron, seulement trois pages rédigées par l’enfant du pays. L’aire de la langue d’oc correspond à une trentaine de départements dont seuls une douzaine sont représentés. Aucun ne se trouve à l’est du Rhône, le berceau des provençalistes. En comparaison, une trentaine de départements d’oïl ou franco-provençaux sur une cinquantaine sont mentionnés avec des témoins en grande partie recrutés par Rousselot parmi ses élèves. L’image de la France est anamorphosée. On ne saurait considérer que le Boulonnais (étude d’Edmont sur Saint-Pol) et le Confolentais comptent pour la moitié du territoire parce qu’ils occupent la moitié de la pagination. 

Sur les vingt-et-un articles consacrés aux pays occitans, quatorze sont des éditions de textes, des \is{transcriptions}transcriptions (chansons, contes…) accompagnées d’un commentaire strictement philologique et deux des études phonétiques (Nougaret et Jean Passy). A part Rousselot, seul Camélat a entrepris la description systématique d’un parler dans l’Hérault. Le peu de réalisations de monographies locales exhaustives recoupe l’absence d’études sur des isoglosses en dépit d’un intérêt affirmé pour les frontières du traitement des palatales ou du /a/. Le postulat d’une variation progressive, sans véritable frontière, induisait une attention concentrée sur le niveau communal. 

En terre d’oïl, trois auteurs ont entrepris de rédiger un lexique ou des études étymologiques (Devaux, Edmont, Marchot) et deux ont conçu la description complète d’un parler : Fourgeaud en Charente et Rabiet en Côte d’Or. En tout, 25 articles font une part à la phonétique.

\section{Phonétique et dialectologie : la question de l’alphabet}

Le lien consubstantiel de la dialectologie à la phonétique s’explique par le caractère \is{oral}oral de ces parlers et la subtilité des réalisations qui les différencient. À l’opposé de l’étude du français (ou des grandes langues littéraires), les analyses se centrent sur l’enquête, le recueil des paroles. Les textes qui ne sont pas transcrits en alphabet phonétique sont donnés comme des figurations approchées dont l’intérêt premier est d’offrir une représentation des réalisations sonores. Très peu de documents avaient été consignés à l’écrit avant d’être reproduits dans la \textit{RPGR}, surtout des chants traditionnels, moins encore avaient été publiés (un article de journal en patois). Sur 48 contributeurs, 17 ont consacré tout ou partie de leur production à des questions phonétiques mais aucun n’a redoublé sa description par l’archivage d’un enregistrement. Il faudra attendre l’initiative des Archives de la parole de Ferdinand Brunot (1860--1938) en 1911 \citep{Cordereix2001}. A côté d’études à portée générale (Rousselot, Psichari, Koschwitz), on relève une analyse sur la phonotaxe de Nougaret et deux sur la morphophonologie par Marchot et par d’Arbois de Jubainville. Les autres travaux concernent :

\begin{itemize}
    \item  soit une analyse monographique sur les réalisations sonores d’un parler (J. Passy, P. Passy, Rabiet, Rousselot, Wilmotte), 
    \item  soit une analyse des réalisations modernes à partir des formes latines (d’Arbois de Jubainville, Devaux, Dottin, Gilliéron, Girardot, Horning, Wilmotte), 
    \item  soit une tentative de définition des formes idéal-typiques d’une région (Fleury, Gilliéron, Gauchat, Rousselot).
\end{itemize}

Dans tous les cas, une même question se posait concernant la façon dont doivent être restituées les productions \is{orales}orales dans une notation scripturale dérivée des caractères latins. D’un côté, il y avait les limites imposées par les polices et les fontes des imprimeurs. De l’autre, des solutions avaient déjà été mises en pratique dans l’adaptation aux différentes langues romanes (digraphes, diacritiques, ponctuation…). Les romanistes ont divergé dans la correspondance entre ressources typographiques et variations phonétiques, les deux principaux alphabets en concurrence à la fin du XIX\textsuperscript{e} siècle étant celui de \citet{Böhmer1871} révisé par \citet{Ascoli1873} et celui de Rousselot.

La principale innovation de Rousselot pour décider d’une écriture en 1887 est de partir du signal et non des notations écrites afin de choisir celles qui seraient le mieux ajustées aux productions sonores. Il n’a fondé son système ni sur les oppositions, ni sur les perceptions mais sur les qualités articulatoires que ses instruments lui permettaient d’observer. La description ne commence pas par la séparation en consonnes et voyelles mais par celle des «~résonnances~» (pharyngale ou nasale) et des «~sons~». L’approche est fondée sur des unités que définit leur contenu et non leur fonction. Le «~Système graphique~» est présenté juste après l’introduction générale, dès la page 3 du tome I. Il n’y a pas de tableau récapitulatif et la description des signes préconisés est exposée de façon strictement linéaire.

Après les résonances (consignées sous forme d’indices ou de suscriptions) sont énumérés les «~sons fondamentaux~» subdivisés en «~consonnes~», «~résonnantes~» (= sonantes), semi-voyelles et voyelles. La liste s’établit à trente unités, soit toutes celles présentes en français auxquelles sont ajoutées la glottale /h/, les deux fricatives dentales /θ/ et /ð/, une dorsale palatale /ç/, vélaire /γ/ et uvulaire /χ/, une nasale /ŋ/, une lambdaïque /λ/ et quatre rhotiques notées [R \textlatin{ɾ}\textlatin{̪} r\textlatin{̥} \textlatin{ɾ}]. Afin d’augmenter le nombre de variantes sont prévus des diacritiques correspondant à la mouillure, à la fricativisation, à une réalisation gutturale et aux deux positions ATR et RTR. Pour les sonantes, le passage d’une valeur consonantique à une valeur syllabique est indiqué par un point souscrit. A chacune des trois semi-voyelles du français est affecté un symbole distinct, comme dans l’API.

Concernant les voyelles, Rousselot part de sept timbres qui correspondent à /a/ /e/ /i/ /o/ /u/ /œ/ /y/ auxquels est ajouté le schwa /\textlatin{ə}/. Les timbres sont dédoublés par l’indication de leur quantité (un accent circonflexe à l’envers ou à l’endroit), leur degré d’aperture (accent grave et accent aigu) et la nasalisation (tilde) ou la demi-nasalisation (un tilde aplati). Comme les degrés d’aperture peuvent excéder les quatre pertinents en français, une même voyelle moyenne peut recevoir un indice qui permet d’en démultiplier les réalisations. De la combinaison des diacritiques résultent d’inévitables difficultés d’impression qui ne seront pas résolues.

Les «~sons intermédiaires~» sont marqués par la suscription de l’un des deux. Un son entre sourde et sonore, laissant dans l’indécision le choix d’un /t/ ou d’un /d/ par exemple, sera écrit soit avec un t surmonté d’un d plus petit, soit l’inverse, sans qu’à aucun moment il ne soit fait de distinction entre ces deux notations. Le système est compliqué par la taille des caractères : une gradation est établie en notant les sons en cours d’apparition ou de disparition par une police de moindre taille.

Rousselot conclut sa présentation par le rappel de quelques principes :

\begin{quote}
    Tel est l’ensemble du système graphique que nous proposons. On voit :

    1° Que nous empruntons à l’alphabet et aux usages typographiques français la plupart de nos signes.

    2° Que nous conservons à ces signes la valeur qu’ils ont en français, et que nous modifions la forme de ceux dont nous sommes obligés de modifier la valeur.

    3° que chaque signe a toujours la même valeur et que chaque son est toujours représenté par le même signe. (…)

    4° que chaque son est figuré par un seul caractère ; \textit{ch} est devenu pour nous \textit{c} ; \textit{ou}, \textit{u~}; \textit{eu}, \textit{œ~}; \textit{gu}, \textit{g} ; \textit{ss}, \textit{s}, etc.

    5° enfin que nous n’employons aucun signe qui ne serve à figurer la prononciation. Nous ne faisons donc usage ni de l’apostrophe, ni du trait d’union. (\citealt[6--7]{Rousselot1887})
\end{quote}

Suivent dix pages consacrées aux difficultés d’application, en particulier pour la reconnaissance de certaines distinctions (mouillure, /h/ buccal ou pharyngal, l’opposition /c/ \textit{vs} /k/, les diverses réalisations des rhotiques, les affriquées…). Entre /α/ et /i/, Rousselot découpe huit degrés en comprenant les extrêmes, sept entre /α/ et /u/. Comme à aucun moment il n’a récapitulé la liste de tous les sons que son système permettrait de représenter, on peut en livrer une approximation en intégrant l’ensemble des traits retenus pour distinguer les voyelles entre elles :

\begin{itemize}
    \item différence de timbres : 7 + 1 (schwa)
    \item différence d’aperture : 6 entre /a/ et /i/, 5 entre /a/ et /u/, un nombre indéterminé entre /a/ et /y/, soit une quinzaine de possibilités en plus des trois voyelles fondamentales = 18
    \item différence de quantité : 18 x 2 = 36
    \item différence de nasalisation ou de demi-nasalisation
    \item 36 x 2 (nasalisation) + 36 x 2 (semi-nasalisation) = 144 possibilités
\end{itemize}

Enfin, une voyelle peut toujours être accentuée (rien n’est dit de l’accent secondaire), soit

\begin{itemize}
      \item 144 x 2 (accentué vs non accentué) = 288 combinaisons possibles 
\end{itemize}

à quoi il convient d’ajouter le schwa et, avec la taille des polices, l’indice d’un amuïssement ou d’une épenthèse en cours. La puissance générative des traits dépasse l’inventaire de n’importe quelle langue. Comme l’analyse part de la phonétique articulatoire et que toutes ces voyelles sont effectivement réalisables, Rousselot s’arrête à ce constat. Au contraire, confronté aux données de terrain, Gilliéron formule des réserves :

\begin{quote}
    La voyelle finale qui est notée \textit{a} sans distinction de timbre ni de quantité est un son qui, dans un seul et même patois, peut avoir une existence de la même plénitude qu’une voyelle ordinaire non accentuée, ou être un de ces sons que nous rendons par des caractères plus petits, ou même totalement disparaître dans certaines conditions de son existence. Il en est de même de \textit{è} et de \textit{é}. Il ne faut rien conclure de l’absence de ce son dans certains de nos patois. (\citealt[33--34n]{Gilliéron1888})
\end{quote}

Les limites d’une approche instrumentale sont patentes dans l’absence de notation spécifique pour les diphtongues, si fréquentes dans les parlers d’oïl et d’oc. Identifiées par leur timbre, elles ne sont pas considérées comme des phonèmes spécifiques. D’autres difficultés affleurent constamment. E. Rabiet fait part de ses scrupules dans l’établissement d’une graphie phonétique normalisée du patois de Bourberain en Bourgogne \citeyear[243]{Rabiet1887} et F. Nougaret, étudiant le parler de Bédarieux, hésite à attribuer aux segments des caractères qui sont décidés \textit{in fine} par la phonotaxe \citeyear[216]{Nougaret1890}.

C’est par sa dimension critique que la contribution de Rousselot a marqué son époque. A défaut d’élaborer une méthode qui, au-delà de la description du matériel sonore, établirait les principes définissant la liste des symboles pertinents pour la \is{transcription}transcription d’une langue donnée, il a mis en évidence en quoi la tradition scripturale déformait la représentation des langues, qu’il s’agisse de l’orthographe félibréenne (\citealt[II-158]{Rousselot1887}), de l’orthographe du français (\citealt[III-239]{Rousselot1887}) et même de la \is{transcription}transcription en API proposée quelques années auparavant par Paul Passy (\citealt[III-238]{Rousselot1887}).

\section{Une tâche impossible ?}

La \textit{RPGR} a construit son projet suivant une démarche contradictoire : scientifiquement, les parlers «~\il{gallo-romans}gallo-romans~» sont appréhendés comme des langues dignes d’être étudiées pour elles-mêmes, rompant avec le préjugé d’idiomes barbares ou d’un français déformé. Conçus comme la transmission populaire du latin parlé en Gaule sur un territoire rural donné, pour reprendre la définition de l’époque, ils ne le cèdent en rien par leur antiquité et leur authenticité au français malgré la disparité des destins. Politiquement, à la différence de la \textit{Revue des Langues Romanes}, proche des Félibres, ou des travaux conduits en Allemagne et en Italie, la \textit{RPGR} ne s’assignait pas pour fin de relever le statut des «~patois~» dont le refoulement était entériné par la scolarisation obligatoire mise en place quelques années auparavant. 

Le compromis entre les principes du monde savant et les exigences de l’état s’est résolu par une atomisation des parlers. Au lieu d’une forme écrite de reconnaissance transdialectale qui unifierait de vastes ensembles, les directeurs sollicitaient des enquêtes de terrain restreintes à l’échelle d’une commune, voire en deçà (Rousselot à Cellefrouin, Edmont à Saint-Pol). La moindre nuance phonétique relevée d’une localité à l’autre était consignée, requérant un alphabet phonétique d’autant plus complexe qu’il ne restitue pas des différences à l’intérieur d’un système mais des variations entre villages, voire entre locuteurs.

La méthodologie de la revue excluait les contributions directes des patoisants, sinon à titre de témoins. Les auteurs ne pouvaient être que des savants informés des avancées de la dialectologie ou des lettrés qui, par leur trajectoire, maîtrisaient le patois comme langue maternelle et le français (et le latin) par une fréquentation prolongée du système scolaire. A leurs côtés, des amateurs auraient pu proposer à la revue des productions littéraires régionales ou des études de folklore. Ceux-ci étaient plus attirés par l’exposé des mœurs et des objets que par la description phonétique ou grammaticale – à l’exception notable~d’E. Edmont. Quant aux écrivains, ils étaient d’avance rebutés par Rousselot qui stipulait dans une «~chronique~» :

\begin{quote}
    Des correspondants nous proposent des compositions littéraires en patois. (…) Mais ces compositions ne sont à leur place dans notre \textit{Revue} que si elles réalisent les deux conditions suivantes : 1° Représenter exactement le patois d’un lieu déterminé (…). 2° Etre transcrits suivant le système graphique de la \textit{Revue}. (\citealt[III-159]{Rousselot1887})
\end{quote}

\todo[inline]{Rousselot (1890) has been changed to \citet{Rousselot1887}, since this was the RPGR paper.}

Les enfants issus des campagnes dont l’ascension sociale résultait d’une prise en charge des frais de scolarité par leur futur employeur~étaient soit des prêtres, soit des instituteurs. La mission de l’école laïque – substituer le français aux patois dans les départements allophones – s’avérait peu compatible avec un intérêt pour la description des parlers régionaux. Quant aux prêtres, sollicités directement par Rousselot à l’École des Carmes, sauf Fourgeaud et Rabiet, ils ne sont guère allés au-delà de leur collaboration à l’établissement d’une \is{transcription}transcription. 

\section{Rousselot et Gilliéron : deux approches discordantes}

Si, en couverture, les directeurs de la \textit{RPGR} sont présentés par ordre alphabétique, Rousselot signe seul l’introduction programmatique de la revue. L’implication rédactionnelle des deux directeurs a évolué au fil des livraisons (voir \ref{table: Tab.1}).

\begin{table}
\begin{tabular}{lrr}
\lsptoprule
          &    Rousselot    &    Gilliéron \\\midrule
Tome I   &   25  (+ 37)     &         36 \\
Tome II  &  3 (+ 16)           &    10 \\
Tome III  &  13~       &         13 \\
Tome IV  &   149~  & \\
Tome V  &  175~      &    \\\midrule
Total   &   365 (+ 53)     &         59\\
\lspbottomrule 
\end{tabular}
\caption{Nombre de pages rédigées par les directeurs de la RPGR (entre parenthèses les documents établis avec les prêtres de l’École des Carmes)\label{table: Tab.1}}
\end{table}

E. Edmont est le seul collaborateur attitré de Gilliéron à qui a été confié un nombre significatif de cahiers pour qu’il puisse publier une partie de son lexique. A l’inverse, l’entourage de Rousselot bruit de nombreuses soutanes. L’accroissement spectaculaire des interventions de Rousselot dans les tomes IV et V, alors que Gilliéron s’est retiré, s’explique par l’impression de sa thèse. Le tome V édité pour solde de tout compte est partagé entre une liste des «~mots français usités en saint-polois~» (p. 7 à 144) et la deuxième partie de la thèse de Rousselot : «~Modifications historiques de l’ancien fond du patois~» qui, afin de préparer la publication en volume sans retouche des placards, fait sauter la numérotation de la page 144 à la page 208.

Gilliéron s’est orienté vers~la géographie linguistique qui l’a conduit à développer une étude fondée sur l’onomasiologie (\citeyear{GilliéronMongin1905},\citeyear{Gilliéron1918}) tandis que Rousselot, abandonnant la dialectologie, s’est consacré de façon exclusive à la phonétique instrumentale \citeyear{Rousselot1897}. En dépit de leurs divergences, ils partageaient un point commun : une indifférence à une considération anthropologique qui a contribué à creuser le fossé entre l’école française et les écoles allemande, américaine ou russe. Le table (\ref{table: Tab 2.}) relève les points de divergence entre les deux directeurs.

\begin{table}    
\begin{tabularx}{\textwidth}{lQQ}
\lsptoprule
& Gilliéron        &       Rousselot \\\midrule

Dialecte natif  & Vionnaz (Suisse)    &   Cellefrouin (Charente) \\

Collecte & Enquête de terrain   &   Travail en laboratoire \\

Technique & Cartographie (points d’enquête)  &  Monographie (famille)\footnote{Par opposition à une étude aréale qui compare des échantillons de lexique à l’intérieur d’une zone donnée, la recherche monographique concentre l’enquête sur une seule localité en essayant de donner une représentation exhaustive du parler et de la culture populaire.}\\

Données &  Liste de mots   &     Récits, contes, dialogues\\

Unités  &   Lexique (\textit{Wörter})  &    Sons (\textit{Lautlehre})\\

Discipline & Sémantique (Ethnologie) &   Phonétique (Physique)\\

   & (vs linguistique historique)  &  (vs philologie)\\

Variation &  Diachronique    &    Synchronique \\
\\
  &  (étymologie)     &   (réalisations sonores) \\

Réseau   social & Limité\footnote{D’origine suisse, Gilliéron n’a pas suivi les formations classiques en France (classes littéraires, ENS, agrégation) et ne s’est intégré ni aux réseaux politiques de ce temps, ni aux salons intellectuels, ni aux revues à forte visibilité (\textit{Revue des Deux Mondes}, \textit{Revue de Paris}, \textit{Revue Politique et Littéraire)}. L’E.P.H.E. n’était pas au centre d’un réseau de relations comparables à celui que pouvait ouvrir une chaire en Sorbonne ou un accès aux ministères.}    &    Clergé \\

Enseignement & EPHE      &    École des Carmes \\

Parallaxe & Effacement du témoin  &    Absence du symbolique\\

 &   (vs sociolinguistique)  &    (vs phonologie)\\
\lspbottomrule
\end{tabularx}
\caption{Orientations des directeurs de la RPGR\label{table: Tab 2.}}
\end{table}

Si les raisons de l’échec de la \textit{RPGR} sont à chercher avant tout dans la situation faite aux patois en France et dans l’absence d’un public suffisant de producteurs et de lecteurs, l’antagonisme des directeurs, qui ne s’accordaient pas sur la finalité du projet, expliquerait aussi l’interruption rapide d’une publication qui dès la fin de sa deuxième année avait dû suspendre temporairement sa parution. Les difficultés de la \textit{Revue des Patois} de Clédat, rebaptisée en 1889 \textit{Revue de Philologie française et provençale} (en distinguant les deux langues) et le projet inabouti d’une Société des Parlers de France dont le premier numéro du \textit{Bulletin} en 1893 n’a pas eu de suite confirment les résistances collectives qui contrevenaient à la reconnaissance des \is{dialectes}dialectes. 

\section{Conclusion} 

La dialectologie représentait, en France, un point de rupture avec la priorité que les romanistes accordaient aux documents écrits. Ce que les écoles russe et américaine effectuaient au contact des peuples premiers présents sur leur territoire, les linguistes français le réalisaient dans un cadre radicalement différent : les populations rurales auprès de qui sont conduites les enquêtes sont de même origine, linguistiquement et anthropologiquement, que les savants qui viennent recueillir leur témoignage. La distinction, en termes de géographie, de niveau social et de culture est affaire de degré, non d’extériorité. Dans le même temps, pour les linguistes, reste omniprésente en arrière-plan la référence au latin et au français.

La \textit{RPGR} est bien l’un des lieux où s’est accompli le partage entre la linguistique et la philologie d’une part, l’anthropologie d’autre part, sans que jamais ne parvienne à s’établir en France une revue pérenne consacrée à un domaine et à des langues qui s’imposaient pourtant par leur importance numérique et politique. Le refus de leur reconnaissance symbolique a entravé la recherche et la contradiction entre le soutien scientifique dont a bénéficié la \textit{RPGR} et l’absence de réponse de l’état comme de la société est emblématique d’une contradiction jamais résolue.


{\sloppy\printbibliography[heading=subbibliography,notkeyword=this]}
\end{otherlanguage}
\end{document}
