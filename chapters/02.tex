\documentclass[output=paper]{langscibook}
\usepackage{tabularx}

\author{Jean-Michel Fortis\affiliation{Research team Histoire des Théories Linguistiques, CNRS, Université de Paris}\orcid{}}
\title{From localism to neolocalism} 

\abstract{Localism is the hypothesis that spatial relations play a fundamental role in the semantics of languages. Localism has a long history. The first instance of a localist account can be found in Aristotle’s \textit{Physics}. Later, localist ideas surface time and again for the purpose of analyzing prepositions, cases and transitivity. The first part of this paper will be devoted to a short account of past localist ideas. Remarkably, new forms of localism have reappeared in the past decades. This neolocalism involves two main lines of investigation: thematic roles and lexical semantics, especially the semantic analysis of prepositional meanings. In this paper, our next task will be to contextualize the development of these two strands by placing them in their theoretical environment. Both begin to flourish at a significant juncture marked by the rise of cognitive science and by the semantic turn observable in linguistics in the 1960s. This global context is the subject of our second part and sets the stage for a discussion of neolocalist accounts in the third part. Lastly, since this paper makes no pretense at being exhaustive, we draw attention to questions that had to be left out: the existence of more “abstract” forms of localism, the connection of localism with “grounded cognition” and, finally, diachronic studies.}

\begin{document}

\maketitle

\section{Introduction}
In this paper, “neolocalism” refers to localist accounts which have flourished since, approximately, the 1960s, in lexical semantics as well as in analyses of thematic roles. Let us first circumscribe our subject matter, \textit{localism}, a little more closely.\footnote{\textrm{~As far as can be ascertained, the term “localism” first circulated among German-speaking authors. Unfortunately, I have been unable to identify the place of its first occurrence. Neither Bopp, Wüllner nor Hartung, who are generally regarded as the first localists by succeeding authors, describe their own theory as a brand of “localism”. On the other hand, the term }\textrm{\textit{Lokalisten}}\textrm{ (‘localists’) is found in contexts in which the defenders of various forms of localism are lumped together and subjected to a critical examination, e.g. \citet{michelsen_philosophie_1843} and \citet{rumpel_casuslehre_1845}. The latter authors refer back to \citet{grotefend_data_1835} as an early opponent of localist claims, but I was not able to consult Grotefend’s essay and check whether he was the first to coin the terms localist/localism. At any rate, the terms seem to have been first used by critics, in order to characterize a doctrine that was either repudiated (Grotefend, Rumpel) or found to be too one-sided (Michelsen).}} 
 
John Lyons defines localism as “the hypothesis that spatial expressions are more basic, grammatically and semantically, than various kinds of non spatial expressions (…). Spatial expressions are linguistically more basic, according to the localists, in that they serve as structural templates, as it were, for other expressions; and the reason why this should be so, it is plausibly suggested by psychologists, is that spatial organization is of central importance in human cognition” (\citealt{lyons_semantics_1977}: 718). A paramount example is furnished by those semantic analyses which strive to identify a spatial core meaning for at least a subset of prepositions and cases. Neolocalist descriptions of adpositions in cognitive semantics squarely fall within this tradition.
 
Lyons’ definition captures the essence of traditional localism, which may be conveniently summed up in two basic claims: (1) a “morphogenetic” thesis: some nonspatial expressions (e.g. temporal) are derived, or extended, from spatial ones, and this transfer is expected to be one-way; (2) the transfer is unidirectional for cognitive reasons: the palpability and ubiquity of spatial relations in phenomenal experience, the range of conceptual distinctions they afford (location, motion to, via and from, and their further determinations) provide an experientially grounded framework for the construal of relations and properties of a more abstract nature. In spite of Lyons’ allusion to the work of “psychologists” (perhaps an allusion to the \citeyear{miller_language_1976} book of \citeauthor{miller_language_1976}), this two-pronged version of localism has first and foremost been defended by linguists and without any other support than the methods of their discipline. This has overwhelmingly been the case up to a very recent time. 

The definition above, however, is not ideally suited to an important strand of neolocalism, that is, to those theories in which local relations impart structure to what is today designated as thematic roles. Thematic roles need not be transparently expressed by means of spatial forms, hence may at least partly deviate from the morphogenetic claim; likewise, claiming that changes of state rest, for their linguistic structuring, on the kind of conceptualization that gives rise to the expression of motion events is a step that an analysis of surface forms does not warrant straightforwardly. Further, some authors may be reluctant to endorse the claim of a cognitive primacy of spatial relations; we shall see examples of this stance in the course of the discussion. In a nutshell, the meaning to form orientation of some modern theories is at variance with the semasiological perspective of traditional localism, while the cognitive primacy of space may be, in theory, if not in practice, disputed by some authors.

In what follows, we shall see how these two forms of neolocalism have come to coexist through a revival, rediscovery and reelaboration of traditional localism. Our discussion of neolocalism will be divided in two parts. The first part will be devoted to the environment which favored this new surge of localist ideas. The second part will deal more specifically with neolocalist accounts in semantics and in the field of thematic roles. Finally, in the fourth part, a short discussion reviews some aspects of today’s research which could not be explored in sufficient detail in the preceding sections. But first of all, since allusion has been made to the localist tradition, a brief overview of past localist ideas is in order. 

\section{A retrospect}
The best-known source on the history of localism remains Hjelmslev’s doxography in his \textit{Catégorie des Cas} (1935--1937). His purview, however, is not purely historical: some precursors were singled out not for their historical role but in proportion of their value in Hjelmslev’s eyes. Further, his description, including that of theories he praised most, can be faulted on several counts \citep{Fortis2018}. 

To my knowledge, the first localist analysis can be found in Aristotle’s \textit{Physics}, in the context of a semantic clarification of the relation of containment as expressed in the Greek preposition \textit{ἐν} (‘in’). In the \textit{Physics}, this clarification is motivated by Aristotle’s definition of space as the (minimal) place containing a body. As is the case for ‘being’ in \textit{Metaphysics}, the notion of spatial containment is circumscribed and semantic anarchy staved off by distinguishing the various meanings of \textit{ἐν} in a principled way (in particular according to the categories), and showing that they all presuppose a primary meaning, which Aristotle describes as local (namely ‘being in a place’). This primacy, says Aristotle in the \textit{De anima}, rests on the fact that our objects of thought reside in sensible forms, and place is a necessary condition of the existence of these forms (423a3--423a8). The justification, therefore, is both ontological and cognitive.

The most direct localist legacy of Aristotle’s \textit{Physics} dates from the rediscovery of the text in the Middle Ages, and is embodied in the corpus of studies assignable to \textit{speculative (or modistic) grammar} (13\textsuperscript{th}--14\textsuperscript{th} centuries), a synthesis of grammar and logic which in effect promotes an autonomous syntactic analysis aiming at universal validity, though obviously taking its data from Latin (\citealt{bursill-hall_speculative_1971}: 29). The universal scope of the analysis is ensured by moving to a level of description that is based on metaphysical and physical notions. For example, for the purpose of defining verbhood, tense is no longer an ultimate feature. Rather, the ability to carry tense is grounded in the fundamental semantic import of verbs, which is to convey motion, hence time. From this \textit{physicalist} conception of verbs sometimes follows a redefinition of cases in \textit{localist} terms, i.e. as expressing the origin\slash source or the goal of motion. We see, for example, Simon of Dacia, defining the accusative as a “casus dicens terminum motus” (\citealt{joly_physique_1977}). After the Modists, localist statements can be found in various places, especially in authors with a philosophical inclination, such as J.C. Scaliger who, in his \textit{De Causis}, characterizes the class of prepositions with the Aristotelian category of place. As for non spatial uses, they are related to spatial uses, he says, by \textit{analogy} (\citeyear{scaliger_causis_2018}: 152), a post-Aristotelian term which, in this context, refers to the relation to a primary sense.

In the period which approximately spans the 17\textsuperscript{th} and 18\textsuperscript{th} centuries, localist statements concern nearly exclusively prepositions. Although the pedagogical practice of glossing Latin cases with vernacular prepositions (and conversely) had alerted grammarians to the functional equivalence of prepositions and cases (\citealt[26]{ColombatEtAl2010}), they were manifestly more reluctant to analyze cases than prepositions in localist terms. Arnauld and Lancelot, for instance, while recognizing this functional equivalence (\citealt{arnauld_grammaire_1969}: 62) provide no extensive systematic account of cases; in all likelihood, their arbitrariness was for them a challenge to the very possibility of such an account. In addition, cases were associated with grammatical relations, for example in discussions on the natural word order. This must have made them appear to be of a degree of abstraction not amenable to a reduction to a local primary sense. Not until the time of \citet{harris_hermes_1773}, it seems, are cases treated in localist terms on a par with prepositions. Doeleke may therefore be right when he praises the British Neoplatonist for having been the first to explain the meaning of cases in terms of spatial relations (\citeyear{doeleke_versuche_1814}: 7).

An obstacle to localism was that the whole class of so-called \textit{particles}, including prepositions, had been associated with acts of thought, not with conceptual content (\citealt{nuchelmans_historical_1986}). When Leibniz, stimulated in particular by Locke, turned his attention to particles, his localist analysis of prepositions was a way of undoing the act\slash content distinction by providing them with a conceptual substance, with the ultimate purpose of paraphrasing them in a universal language. In the general epistemic context of the time, linguistic analysis took on a new importance, because of the status of language for the theory of knowledge: language revealed operations and concepts of the mind (for Leibniz, cf. \citealt{dascal_leibniz_1990}) but also had a potential of obfuscation (especially for Locke, \citealt{dawson_locke_2007}), aspects which both attested to its cognitive power. The nominalist proclivity of empiricism, quite perceptible in Locke’s treatment of mixed modes (“it is the name that seems to preserve those essences, and give them their lasting duration”, \citealt[434]{LockeHumanUnderstanding}), by conferring to language a capacity to form universals, could only reinforce its cognitive relevance (see an important discussion of this point in \citealt{formigari_language_1988}).

In this broad context, forms with a spatial meaning are particularly central for they can lay bare a fundamental aspect of cognition, the apprehension of spatial relations. In the diachronical perspective of the time, this apprehension was approached in two different, sometimes coexistent, ways: in view of the importance of this apprehension in the cognitive development of mankind, in a priori genealogies of mind and language (cf. \citealt{condillac_cours_1775}, esp. II.13 on the primary spatial meaning of French \textit{de}), or in the first attempts of modern historical linguistics, as justifying hypotheses on the origin of forms, especially of cases. Localist hypotheses of an a priori nature could coexist with “technical” considerations on the evolution of forms: this was the case in Doeleke’s study, and Bopp himself entertained localist ideas, hence his claim that some endings of Sanskrit, Latin and Greek declensions originated from prepositions and demonstratives with a primitive spatial meaning (\citeyear{bopp_vergleichende_1826}).

All throughout the 19\textsuperscript{th} century and well into the 20\textsuperscript{th} century (cf. \citealt{kurylowicz_inflectional_1964} ), the primacy or importance of spatial meaning for cases is a claim many authors endorse or feel called upon to discuss or criticize, mostly in the perspective of historical linguistics, but in psychological linguistics and philosophy as well (cf. \citealt{marty_logische_1910}). The 19th century is the centrepiece of Hjelmslev’s overview, which is still worth consulting for more information (but again, caution has to be exerted; \citealt{Fortis2018}). The most developed localist account of cases, with an emphasis on Latin, Greek and Sanskrit, is put forward by Wüllner (a student of Bopp) in two books, of which the first (\citealt{wullner_bedeutung_1827}) proposes a semantic analysis focusing on Greek and Latin cases, while the second (\citeyear{wullner_uber_1831}) introduces Sanskrit data and is more historical. In both studies, Wüllner defends the radical view that the fundamental meanings (\textit{Grundbedeutungen}) of all cases (except the nominative and vocative) are local. This “pan-localism” is achieved, obviously, at the price of pushing under the rug the nominative case, thus dodging the problem of analyzing it in localist terms (cf. also Hjelmslev’s own convoluted localist account). Otherwise, Wüllner’s basic tenet is simple: the \textit{Grundbedeutungen} of the genitive, accusative and dative cases are spatial intuitions (\textit{Anschauungen}), respectively of a starting point (\textit{Anfangspunkt}, \textit{woher }‘where from’), of a goal (\textit{wohin }‘where to’) and a localization (\textit{wo }‘where’) (the ablative is considered as a secondary differentiation of the dative). Other senses are derived from these intuitions, forming semantic networks akin to those found in present day cognitive linguistics (without, however, the now usual diagrammatic representations).

The testimony of \citet{rumpel_casuslehre_1845} tells us that in the domain of cases localist accounts enjoyed a supremacy, especially for pedagogical reasons. This supremacy, for Rumpel, had to be contested and overcome. Together with \citet{curtius_uber_1864}, \citet{rumpel_casuslehre_1845,rumpel_zur_1866} championed an anti-localist reaction which betrayed a certain weariness of philosophical grammar. Localism was for him an offshoot of an outdated conception of grammar, inherited from Enlightenment, and according to which language was the creation of the human mind reflecting on its own operations, in a bootstrapping process taking its origin from sensory features and embodied cognition. The description could fit Wüllner’s perspective pretty accurately and, beyond him, might have had Herder and his \textit{Besonnenheit} (‘reflection’) in mind. What Rumpel aimed at was a more formal definition of grammatical relations, redirecting grammar toward surface forms and taking as basic the most fundamental fact of human thinking, the subject-predicate structure, for which localists had no explanation. He also inveighed against the proliferation of senses entailed by the semantic network approach of localists, a proliferation, he said, which in effect transferred to cases the semantic features which had traditionally been used to classify verbs in cooccurence with the different cases.\footnote{A strategy found e.g. in \citet{despautere_sintaxis_1527} and \citet{lancelot_nouvelle_1653}.}

Authors who were less philosophically committed than Wüllner and Hartung and proceeded more matter-of-factly showed some reluctance to accept that the genitive was more than the default adnominal case, or that the accusative had a fundamental spatial value, since this use was marginal with the bare case. Further, with the evolution of historical grammar, Latin and Greek cases were more and more seen as syncretic with respect to Sanskrit, which made it more difficult to confer a unitary value on them. The consequences are best appreciated by considering Holzweissig’s semilocalist account (\citealt{holzweissig_wahrheit_1877}). In Holzweissig’s system, cases with one fundamental spatial value are restricted to a subset of Sanskrit cases and distinguished from the grammatical cases (nominative, vocative, accusative, genitive). Latin and Greek have reshuffled the values of Sanskrit local cases; the Latin ablative, and the Greek genitive and dative are described as \textit{Mischcasus}.

\begin{table}
\begin{tabularx}{\textwidth}{QQll}
\lsptoprule
Latin & Greek & Sanskrit  & Value\\\midrule
abl. separat. & gen. & abl. & \textsc{from} \textit{(Wohercasus)}\\ 
abl. loci\slash temp. & dat. loci\slash temp. & loc. & \textsc{at} \textit{(Wocasus)}\\
abl. comit.\slash mod.\slash instr. & dat. comit.\slash mod.\slash instr. & instr.-sociat. & \textsc{with} \textit{(Mitcasus)}\\
dat. & dat. & dat.  & \textsc{to} \textit{(Wohincasus)}\\
\lspbottomrule
\end{tabularx}
\caption{The shift from spatial to non-spatial values is relativized to a historical stage. Note that Sanskrit lays bare the fundamental meanings present in a more primitive stage of Indo-European.}
\end{table}

The assumption that the concepts underlying cases (\textit{Grundbegriffe}) must be found in a primitive stage is precisely what Wundt (\citealt{wundt_volkerpsychologie._1912}) rejects as “mythological”. Being cognitively (and affectively) motivated through and through, expressions of relations must reflect, for any language at whatever stage, the linguistic decomposition of thought into attribution and predication, as well as an open-ended set of phenomenal properties (\textit{Gegenstandsbegriffe}), or “external determinations”, which may involve causality, comitativity, similarity…, in addition to spatial relations. On this view, localism unduly restricts “external determinations” to spatial values, a mistake justifiable by the fact that external determinations are maximally distinctive in the spatial domain (\citealt{wundt_volkerpsychologie._1912}: 115).

\section{The global context of neolocalist views}
We shall now try to describe in broad strokes the environment in which neolocalist views were incubated, with the goal of understanding the conditions which favored or legitimized them. For convenience, these conditions have been sorted along disciplinary lines. 

\subsection{The global context: The rise of semantics}
Neolocalist descriptions may be conveniently categorized into two classes. One class comprises semantic analyses of lexical items.\footnote{\textrm{ In this paper, we will designate as “lexical semantics” the semasiological analysis of forms like }\textrm{\textit{out, up, over}}\textrm{ etc. and the onomasiological study of the notion of verticality in \citet{nagy_figurative_1974}. Let us observe that the onomasiological orientation of Nagy remains rather marginal (as we shall see). }} The other includes localist accounts of thematic roles. Both kinds of description appear to emerge at a juncture which corresponds to the so-called “cognitive revolution” (\textit{revolution} is a term we would not endorse for reasons we cannot go into here). What should most arrest us in this “revolution” is a number of theoretical changes directly relevant to the new rise of localism. On the side of linguistics, especially in the United States, semantic concerns gain in importance, after a relative eclipse among the post-Bloomfieldians; on the other hand, thematic roles come to the fore of syntactic analysis, especially with Fillmore, as we shall see later. Both trends, as can be gathered from testimonies of the time and from the turn of events itself, notably the advent of generative semantics, were significantly encouraged by generative grammar, and in particular, the notion of deep structure invoked in the \textit{Aspects} model. In this paper, I shall not delve further into this evolution and the semantic turn prompted by transformational grammar (TG); this has been documented and discussed elsewhere \citep{newmeyer_linguistic_1986,harris_linguistic_1993,huck_ideology_1995,Fortis2015a}. However, the important role played by transformational grammar should not mislead us into thinking that semantics would have remained a blind spot if generative grammar had not entered the scene. Hymes and Fought’s observation that, over the long term, American structuralism progressively expands into syntactic and semantic territories does not exclusively rest on the success of generativism (\citealt{hymes_american_1981}). Their judgment can be confirmed by the fact that forays into semantic issues are accomplished by practicioners of a “late structuralist” bent, i.e. by linguists who were extending up to a semantic layer the stratal organization of forms into allophones\slash phonemes\slash morphophonemes-allomorphs and morphemes. We cannot go into the details of this evolution here, but suffice it to say that it was a short step to conceiving of the morpheme as an abstract semantic unit as soon as one had analyzed, e.g., the /u/ of \textit{took} as an allomorph of /ed/; for /ed/ could then be glossed as /past/ (\citealt{hockett_two_1954}, \citealt{lamb_sememic_1964}; for a discussion on this history of the notion of morpheme, cf. \citealt{matthews_grammatical_1993}).

\citet{chafe_phonetics_1962} was an early proponent of this extension to semantics: in this paper, he proposed to consider morphemes as arrangements of semantic features, on the analogy of phonetic features and, in all likelihood, on the model of componential descriptions put forward in “ethno-semantics”, especially for the purpose of analyzing kinship terms (e.g. \citealt{lounsbury_semantic_1956}; on this short-lived strand, cf. \citealt{murray_dissolution_1982}). The examples of generative grammar and generative semantics would encourage Chafe (\citealt{chafe_meaning_1970}) to develop his theory in the direction of a stratal model in which surface forms are generated from a considerably enriched semantic stratum. In this elaborate model, the semantic stratum was in charge of inventorying forms along semantic parameters, and of stating selectional restrictions, semantic changes effected by derivations and inflections, and even pragmatic aspects (such as intonational variations and information structure). Note this conception would lead Chafe to abandon the notion of morpheme altogether, hence to divorce the semantic stratum from the segmentation of linear sequences of forms.

The ambiguity of the notion of morpheme, i.e. its being a class of forms or something \textit{represented} by phonemes, is quite explicitly one of Lamb’s starting points in developing his own version of a stratal grammar (\citealt{lamb_sememic_1964}). To put some order into this confusion, Lamb recommends distinguishing carefully what a unit is composed of from what it is taken to represent. For example, he insists that \textsuperscript{M}/good/ and \textsuperscript{M}/bett-/ (in \textit{better}) are morphemes which cannot be put on the same level as what they represent, namely a superior unit called “lexon” and glossed as \textsuperscript{L}/good/. The latter, in turn, is separated from the semantic unit it represents, its \textit{sememe}. This separation of a semantic plane, as it was for Chafe, and apart from considerations having to do with the inner logic of the system, is motivated by the wish to account for the properties which he thinks can only be stated on this level, and between this level and lower ones, for instance properties such as the synonymy and polysemy of forms, e.g. the fact that the sememe \textsuperscript{S}/also/ is represented by \textsuperscript{L}/also/ and \textsuperscript{L}/too/ and, in a negative environment, by \textsuperscript{L}/either/. In short, in Lamb’s framework, a stratificational grammar (SG) should provide an account of the “tactical” pattern (= combinations) proper to each stratum and of their interrelations. To this purpose, Lamb resorts to a formalism of his own which consists of networks presented in diagrammatic fashion. Units are connected via various types of \textsc{and} and \textsc{or} nodes which account, respectively, for the composition and alternations of units (or classes). The first developed presentation of this formalism was submitted in his short opus of \citeyear{lamb_outline_1966}.

It is all the more important to mention Lamb’s version of stratificational grammar since it was perceived at the time as a rival to TG, and one that could favorably compete with it insofar, e.g., as it afforded an explicit measure of the complexity of grammars (roughly, as a function of the number of links between items). However, a major drawback of Lamb’s theory was that it was not fully expounded until Lockwood’s textbook (\citealt{lockwood_introduction_1972}). This inconvenience, together with other adverse circumstances (\citealt{nielsen_private_2010}), ensured that stratificational grammar would never win a support in any way comparable to the success enjoyed by generative grammar.

On the whole, both TG and stratal grammars contributed to the rise of semantics, and this rise manifested itself, among other things, in semantically-oriented studies of prepositions and in inchoate localist analyses of verbs. White’s analysis of English prepositions (\citealt{white_methodology_1964}), based on a corpus, offered an early example of the former kind of study. It was framed in Lamb’s formalism and exploited the potential of Lamb’s systemic approach. That is, White considered a \textit{system} made up of 11 non-compounded prepositions which he analyzed into their sememes (senses), based on the commutation with other prepositions and on distributional evidence. For example, the commutability of \textit{about} with \textit{around} in certain contexts was taken to justify positing a separate sense corresponding to this use of \textit{about}; in other cases, distribution, that is co-occurrence with a set of semantically cohesive verbs, provided evidence for a distinct sememe. A residue of uses was characterized as idiomatic. Lamb’s network notation served to represent the semantic interconnections of prepositions in the system. Worthy of note was the fact his method was conducive to a proliferation of senses, and that no localist hypothesis was formulated, and understandably so, since his tests conflated \textit{at home} and \textit{at noon}, and, on the other hand, distinguished subtle shades of meaning as a function of cooccurrences.

An example of an early localist attempt coming from the transformational circle is provided by \citet{Lakoff1976}: to simplify, in this paper, some selectional restrictions on verbs were associated with a classification of these verbs by semantic features, and this classification was partly localist. There was for instance a class of verb of “directed change” which was subdivided into goal- vs source-oriented items (resp. \textit{I became insane} vs \textit{I lost my sanity}). Talmy’s dissertation (\citeyear{talmy_semantic_1972}) could also be seen as an (unorthodox) emanation of the transformational approach, close to the generative semantics movement. His deep syntactico-semantic structure was tailored to the analysis of structures referring to motion events, and in this initial stage of his theory, this structure could be interpreted as a linguistic template transferrable to non-spatial events. This was hypothesized to follow from the cognitive centrality of the structuring of motion events (cf. \citealt{kimball_semantics_1975}: 234). There is in this respect a seamless evolution leading to Talmy’s more direct concerns with cognitive matters and his future affiliation with cognitive linguistics (\citealt{fortis_morpho-syntax_2016}).

The most elaborate study adopting the SG framework was due to Bennett (\citeyear{bennett_spatial_1975}) and chose a strategy opposite to that of White, i.e. Bennett strove for the maximal reduction of polysemy. Bennett’s plea for monosemy was facilitated by his methodology (neither based on corpus nor on distribution) and the fact his scope was confined to spatial and temporal uses, which implied that no attempt was made at deriving or explaining more abstract uses. From his testimony, we learn that his inclination to monosemy came from Jakobson’s description of cases as expressing a Gesamtbedeutung, Fillmore’s attempt (\citeyear{bach_case_1968}) at identifying a list of universal deep cases (in modern parlance, thematic roles), and finally componential analyses of the kind promoted by SG. In short and to simplify somewhat, his semantic descriptions were structured sets comprised of five local cases (locative, source, goal, path, extent) combined with various specifications such as, e.g., ‘interior’ (for \textit{in}), ‘proximity’ (for \textit{by}) or ‘visibility’ (for implicitly viewer-centered uses, that is uses in which a viewer is a reference point). The semantic structuring of prepositions was represented in the form of tree diagrams, which thus served to express scope relations: In \textit{The passenger fell asleep and went past his station}, ‘past his station’ was glossed as ‘at a goal at the end of a path via the proximity of his station with respect to a reference point’, with ‘at —’ having scope over ‘goal’, ‘goal’ over ‘path’ etc. As for the stratificational formalism, it was used, in Lamb’s parlance, on the semotactic and semolexemic levels. To semotactics belonged the task of expounding the various kinds of locative propositions available in English, the range of choices at the disposal of speakers (such as the distinction between extent and locative expressions, and the inventories of their components) and selectional restrictions, for instance the cooccurrences between aspectual classes of verbs and prepositions. The semolexemic diagrams encoded the lexemic realization of semantic structures.

Bennett’s position is not localist, insofar as he never declares that space has any kind of primacy and he does not venture into diachronical and cognitive considerations. He claims, however, that local cases have spatial and temporal uses in common; in addition, he proposes an analysis of tenses in terms of local cases. If Bennett were to be assigned to a family of theorists, he might be deemed closest to those who have always denied the primacy of spatial meanings or, in Anderson’s words, who have insisted on the \textit{neutrality} of the principles of organization with respect to the domain in which they are instantiated (be it spatial, temporal, or abstract). In this group we may include Beauzée (\citeyear{beauzee_preposition_1786}) and Pottier (\citeyear{pottier_systematique_1962} and later studies), who both stand for a notion of locative meaning encompassing domains other than space. For Pottier, e.g., the locative layer of cases encompasses space, time and “abstract” meanings he calls “notional” (e.g. \citealt{pottier_linguistique_1974}: 53--55). Whoever would summon diachronical evidence for asserting the primacy of space would be, according to Pottier, misguided: synchronically, relators have a permanent potential for expressing spatial, temporal and notional relations (\citealt{pottier_systematique_1962}: 126). According to Anderson (\citeyear{anderson_grammar_1971}; \citeyear{anderson_localism_1994}), Hjelmslev also adheres to the neutrality stance, which he suggests distinguishing from bona fide localism and for which he reserves the special name of “localistic”. Localistic views, therefore, either are agnostic on the cognitive grounding of local cases, like Bennett, or explicitly reject the primacy of space, like Beauzée, Hjelmslev and Pottier, yet see commonalities between spatial and non-spatial relations.

A hallmark of stratificational approaches was their full autonomization of a semantic plane. In this they differed from generativists, be they adherents of the interpretative or generative version of TG, who regarded semantics, respectively, as a matter of providing interpretations of syntactic structures, especially to disambiguate them, or as a very deep level in charge of the generation of surface forms. There were for sure semantic representations and specific semantic rules in analyses affiliated to TG, for example in Katz and Fodor’s famous attempt at specifying what an interpretative component should be like (\citealt{katz_structure_1963}). By contrast, some studies, including some not affiliated to SG, also undertook to state properly semantic rules not indexed to the generation of surface structures, and to this end had developed a semantic notation apt at representing what they regarded as the “quasi-deductive system” underlying semantic interpretation. “Quasi-deductive system” was the term employed by \citet[163]{Weinreich1972}, whose blueprint for a semantic theory incorporated a notation for representing semantic interactions between cooccurring terms and sketched a generative account directly mapping semantic-categorial structures (such as ‘verb + circumstance’) to morphosyntactic forms. In the same spirit was Leech’s attempt at devising a formulaic (and rather cumbersome) notation for semantics (\citeyear{Leech1969}). Leech made use of two basic relations (predication and attribution), various symbols for representing definiteness, quantifiers, negation, inchoativity etc. and, like Weinreich, analyzed lexical contents into bundles of features (or “clusters”). His field of inquiry, he declared (\citeyear[28--30]{Leech1969}), was properly the semantic plane viewed as an \textit{autonomous} level, in contradistinction to generative grammar. Most significant for us was the fact he applied his apparatus to the domains of place, time and modality, which betrayed the philosophical background of his seemingly purely linguistic essay: in effect, the fundamental concepts and domains he was working with were the Kantian forms of intuition and categories. Presumably, then, place was chosen as a domain of application in view of its being a received category of Western epistemology (for a discussion of this point, see Chalozin-Dovrat, this volume). It was chosen too because spatial markers (reduced to prepositions and names of compass points and object parts) appeared to form a \textit{system} whose features were amenable to a complete inventory. In short, place expressions were both basic and manageable.

\subsection{The global context: Cognitive linguistics}

It is far beyond the scope of the present paper to explore the origins of cognitive linguistics. The reader may consult other texts in which I have attempted to narrate this history (e.g. \citealt{arigne_generative_2015}). Of immediate relevance for our subject is the fact that, after the schism which caused cognitive linguists to split from generative grammar, semantics offered itself as a promising niche. These linguists (notably Lakoff, Langacker and Talmy) were all the more inclined to engage in semantic issues since they had all been involved in the semantically-oriented dissident movement known as generative semantics. In this favorable environment, localist ideas surfaced again. We have seen the case of Talmy above; unfortunately, his dissertation containing no reference to previous work, the possible inspiration of his early localism remains inscrutable. Lakoff and Johnson, with their conceptual metaphor theory, had embarked on an empiricist program which defended as a corollary rather extensive localist views, thus claiming that “most of our fundamental concepts are organized in terms of one or more spatialization metaphors” \citep[17]{lakoff_metaphors_1980}. In fact, spatial metaphors such as ‘more is up’ (e.g. \textit{prices have gone up}) were cited by Johnson (\citeyear{johnson_introduction:_1981}) as a counter-example to the view that metaphors rest on similarities between a source and a target concept. They played a role, therefore, in establishing the very notion of conceptual metaphor, that is, the idea that a metaphor is not based on pre-conceived similarities but “serves as a device for reorganizing our perceptual and\slash or conceptual structures” (\citealt{johnson_introduction:_1981}: 31). It should be noted that this about-turn in Lakoff’s theoretical concerns, from generative semantics to metaphor theory, was not coming out of the blue. Metaphor was a much discussed subject in the 1970s, a “hot topic” as Honeck (\citeyear{honeck_historical_1980}) puts it, partly because of the redirection of psycholinguistics from transformational grammar to semantic issues.

Just as Leibniz, among others (\citealt{leibniz_lingua_1923}), had claimed that the non-spatial senses of particles were connected through tropes with their primary spatial senses, figures (and especially metaphors) were employed for deriving “abstract” senses from concrete ones. Further, through the adoption of prototype theory, imported from psychology and legitimated by it, though simplified in the process, linguists had found a convenient tool for handling polysemy (\citealt{kleiber_semantique_1990}; \citealt{arigne_prototype_2018}). The combination of empiricist views and of prototype theory obviously favored the reintroduction of localist semantics, especially in the traditional area of particles and adpositions, as we shall see in the next section. In the present case, speaking of a “rediscovery” might not be quite appropriate. Although cognitive linguists had been raised in the lap of generative grammar, hence were relatively cut off from the tradition of semantics, we cannot exclude that transmission did take place. Langacker, for instance, was acquainted with Nagy’s dissertation, had read Anderson’s \textit{Case grammar} (\citeyear{anderson_grammar_1971}), and had declared himself to be “basically sympathetic” with its localist orientation (\citeyear{anderson_essay_1973}). Likewise, the work of Nunberg (\citeyear{nunberg_pragmatics_1978}) was known to Lakoff (as is testified in the acknowledgements), and in this text reference was made to historical linguistics, with an eye toward the application of principles of semantic change to the synchronic treatment of polysemy. In particular, Nunberg hinted at the work of Darmesteter (\citeyear{darmesteter_vie_1887}) who, had he been consulted by Lakoff, may have inspired to him the idea of representing a semantic network in diagrammatic form. The junction of empiricism, prototype theory and semantic networks at least partly arose out of these historical circumstances.

\subsection{The global context: Psychology and cognitive science}

The rise of semantic concerns in linguistics coincides with significant changes brought about in psychology by the demise of behaviorism, although, as we shall see with Osgood and Nagy, the divide between pre- and post-behaviorist psychology was not such that no continuity nor conciliation can be witnessed.

The changes hinted at here can be conveniently placed under the banner of “cognitive psychology”, first so named in Neisser’s book of \citeyear{Neisser1967}. Opposition to stimulus-response psychology had gathered momentum in the preceding years (mostly, from the 1950s on) and a series of studies had converged toward the idea that subjects actively (re)construct the stimuli they are exposed to. For example, Bousfield (\citeyear{bousfield_occurrence_1953}) had observed that subjects tended to recall words in clusters corresponding to semantic categories, that is, they tended to reorganize the material presented. In a similar vein, \citet{BransfordFranks1971} observed that subjects recognized as old a sentence that was in fact new, but semantically coherent with the sentences that had been presented during the acquisition phase. Attneave (\citeyear{attneave_transfer_1957}) had demonstrated that subjects found familiar a shape that had not been part of the experimental items but had served to model them by systematic deviations, and which he called for that reason the \textit{prototype} of the series. Neisser (\citeyear{neisser_cognitive_1967}) laid much emphasis on these constructive processes, which he regarded as ubiquitous and spanning all the range of human abilities, from perception (cf. his notion of iconic memory as a buffer storing items for constructive processing) to the hierarchical verbal structures posited by TG, strangely likened to superimposed Gestalten. On the whole, constructive processes and structuring went hand in hand with the relevance devolved to semantic factors.

In some quarters, the mind-as-computer metaphor and the comparison of cognitive processes with states of a Turing machine (the so-called functionalist view of \citealt{putnam_minds_1960}) were enthusiastically seized upon as offering a free hand to speculate on mental representations, including in linguistics (\citealt{katz_mentalism_1964}), without having to worry too much about their ontology. This new freedom was a favorable environment for the reintroduction of notions which had been repressed, though not entirely banned, during the behaviorist era, such as mental images, voluntary attention, or teleological behavior (“will”). For our subject, the fact that mental images were rehabilitated is of especial importance, since spatial or diagrammatic representations would later flourish in cognitive linguistics, and would be identified with the meaning of spatial morphemes. This rehabilitation of imagery was progressive, anticipated in some late behaviorist work \citep[281ff]{Mowrer 1962} and in the margins of mainstream psychology (e.g. in research on hallucinations, \citealt{holt_imagery:_1964}). It was legitimated with experiments which had a strong persuasive power because they exhibited striking linear relationships between processing time and variables hypothesized to be proper to visual images (\citealt{baars_cognitive_1986}: 161, cf. \citealt{shepard_mental_1971} and other studies).There was however some resistance to accepting that phenomenal properties of representations (such as the visual properties of images) may play a functional role in cognition. This resistance may be seen both as a prolonged aversion to anything smacking of introspection, and as a consequence of the computational view of the mind, according to which mental representations having an effective computational role must have a propositional structure (on this debate cf. \citealt{fortis_image_1994}). This resistance, however, was largely overcome, and the endorsement of imagery opened up avenues of research for psychologists and neuropsychologists, who devoted a considerable number of publications to the role of imagery in recall, to the relations between images and words (e.g. in studies on picture naming), or to the relative share of modality-specific representations (e.g. visual) in various categories of “concepts”, a topic hotly debated in neuropsychology (\citealt{fortis_signification_1997}).

The importance laid upon the visuo-spatial\slash linguistic interface was enhanced by the prospect of making the sciences of the mind/brain converge into a unifying approach, cognitive science. It is therefore no coincidence that George Miller, a psychologist who was a staunch advocate of this unified science of cognition (see e.g. his contribution in \citealt{walker_report_1978}), launched into an ambitious study entitled \textit{Language and perception} (\citeyear{miller_language_1976}, written in collaboration with Philip Johnson-Laird). While the first part of the book focused on psychological matters, the second part more directly addressed the psychological underpinnings of the linguistic representation of the perceptual world. Unsurprisingly, the expression of spatial properties (shape, location, and motion) was considered as a fundamental semantic domain, and the authors justified this privilege by appealing to the central role played by spatial relations in cognition, citing in this regard the localist declaration of the philosopher Urban: “our intellect is primarily fitted to deal with space and moves most easily in this medium. Thus language itself becomes spatialized, and in so far as reality is represented by language, reality tends to be spatialized” (\citealt{urban_language_1939}: 186, in \citealt{miller_language_1976}: 375). Urban’s localism echoed British empiricism, Bergson’s reflection on spatializing thought, and also Cassirer’s views on the spiritualization of concrete determinations effected through language as a medium of representation (\citealt{cassirer_philosophie_1923}). These also formed the background, it seems, of Miller and Johnson-Laird’s conception, but most of their discussion was confined to a semantic analysis of spatial terms (paraphrased with basic concepts expressed in first order predicate calculus) and linguistic coordinate systems.

A surge of universalist ideas accompanied the quest for the unification of the sciences of the mind which was proclaimed to be the goal of cognitive science. From different quarters, universalist hypotheses were boldly put forth: Chomsky’s universal grammar (from \textit{Aspects} on), Berlin \& Kay’s theory of a universal path of color terms differentiation (\citealt{berlin_basic_1969}), Ekman’s theory of universal emotions (\citealt{ekman_universals_1971}), Lenneberg’s late work on the biological determinants of language acquisition and processing (\citealt{lenneberg_biological_1967}). In a universalist perspective, the analysis of locative expressions could be framed in terms of universal cognitive constraints on their acquisition and use. Such was for example the way the psychologist Herbert Clark described his own endeavor: demonstrate that the child’s a priori knowledge of space, e.g. the “vectorization” of space determined by the asymmetry of the human perceptual apparatus, constrains the acquisition of spatial relators. From a developmental angle, a localist hypothesis implies that spatial markers are acquired before their metaphorical extensions. This was indeed how Clark dealt with temporal expressions, which he considered to be grounded in metaphors based on the experience of motion and lexicalized by primarily spatial markers (\citealt{moore_space_1973}). This aspect of Clark’s work would later be appropriated by Lakoff in support of his own localism. 

\section{Neolocalist views}

The stage is now set for the emergence of neolocalist views. What follows is an exposition of the views themselves. As was said in the introduction, they address two main issues: lexical semantics and the nature and functioning of thematic roles. 

\subsection{From cross-domain associations to figurative patterns: Osgood and Nagy}

In American lexical semantics of the post-war period, the first clearly localist endeavor can be traced back to a neo-behavioristic framework antedating the semantic turn promoted by transformational grammar. Neo-behaviorism accommodated inner responses which mediated the production of overt behavior, and it was therefore receptive to hypotheses positing unobservable reactions. In Osgood’s theory, meaning was precisely such a mediational process: the meaning of a sign was defined as a fractional response, i.e. as a part of the total behavior associated with the referent of the sign; by dint of repetition, this response was reduced to a kind of abbreviated replica, a “disposition” (\citealt{osgood_nature_1952}). Now, the conditioning of signs to primary dispositions, through generalization, extended to mediators elicited in proportion to their similarity to original reactions, but also thanks to spontaneous cross-domain elicitation, that is, synesthesia (\citealt{karwoski_studies_1942}). But synesthesia was only the most vivid manifestation of a wider phenomenon, the correlation of properties belonging to different dimensions. Osgood and his associates set out to demonstrate that it made sense, even for subjects who were not synesthetes, to correlate scales which belonged to different modalities, for example the soft-loud scale and the large-small scale, the happy-sad scale and the bright-dark scale etc. From there, the next step was to define the semantic profile of a sign as the set of its values determined by subjects on a large number of continua. This set, the “semantic differential”, was thus conceived of as an operational definition of meaning, although, as admitted by Osgood himself, meaning was in effect reduced to connotative associations (\citealt{osgood_nature_1952}: 231).

While Osgood (\citet{osgood_nature_1952}) had alluded to the cross-cultural relevance of correlations associating spatial relations, especially verticality (up-down) with positive-negative values, he had not pursued this idea along localist lines. In his dissertation, Nagy (\citeyear{nagy_figurative_1974}) applied Osgood’s notion of a bipolar organization of meaning by selecting a specific spatial scale, the up-down axis, with the goal of studying the productivity of its application to non-spatial domains across the lexicon. He called \textit{figurative pattern} a mapping from the verticality scale to another domain, noting for instance that many predicates similar to \textit{high} occur in the domain of prices (\textit{prices were above guidelines, one effect of the war was to boost the price of gasoline, stocks prices climbed slowly\slash declined\slash dipped} etc.). Distinctly localist in Nagy’s study was the notion of an asymmetrical dependence of non-spatial domains on spatial axes. This asymmetry, given the then current concern for generative capacity, was tentatively captured by redundancy rules stating, for example, that lexical items referring to vertical position could be used with terms in the domains of prices, pitch, opinion etc. The restrictions constraining the productivity of figurative patterns proved to be very difficult to state: metonymies had to be taken into account (\textit{stocks went down}), and contextual effects had to be factored in (*\textit{A low suggestion} but \textit{His suggestion as to how much we should pay them was much lower than the legal minimum wage).} This concern for stating the limitations on the productivity of figurative patterns remains a hallmark of Nagy’s study. When localist ideas got appropriated by cognitive linguistics, the issue of productivity and of stating rules for limiting it receded into the background, probably as a consequence of an antagonism to generative rules on the part of cognitive linguists. 

\subsection{Localist cognitive semantics: Particles and adpositions}

In the preceding sections, we have enumerated a number of circumstances which may explain why semantic analyses of particles and adpositions would gain so much prominence once linguistics, and especially American linguistics, had rediscovered the importance of semantics. To these circumstances we should add the obvious fact that Indo-European languages possess rich inventories of prepositions, preverbs, locative adverbs and particles, which fulfill functions that other languages entrust to a generic adposition or to other strategies, such as verb-serialization, applicative markers and posture expressions (\citealt{fortis_morpho-syntax_2016}). In addition, in the special case of English, “relators” such as \textit{at, in, out, off, up} etc. had not disappeared from the linguistic horizon for a reason that is related to the complexity of their morphosyntactic behavior and the attendant difficulty of assigning them to clear-cut categories: are they particles? prepositions? adverbs? “adpreps”? Taking a stance on these matters often meant that semantic considerations had to be brought into the discussion. Some authors, for example, correlated the fact a relator had a “literal” or “concrete” meaning with its having an adverbial function, or being susceptible of receiving contrastive stress (see \citealt{lindner_lexico-semantic_1981}, for a review). In a study quite remarkable for treating together morphosyntactic phenomena, semantic aspects and pragmatic intent (such as contrastive stress), Bolinger (\citeyear{bolinger_phrasal_1971}) underlined the interplay between the literalness of the particle, its ability to move to last position, and its being susceptible of emphasis. Since phrasal verbs in which particles retained their “literal” or “concrete” meaning seemed to be less cohesive, it could be argued that this was due to their making an independent semantic contribution to the phrasal verb, hence that the “literal” meaning was the original one and preceded tighter integrations of the particle with the verb. For this reason, Bolinger claimed for example that the primitive meaning of \textit{up} was directional; he further speculated that this directional meaning had got associated with perfectivity (as in \textit{choke up, rev up}) through physical events of completion (e.g. because filling a glass means the level of the liquid goes up), and also via the notion that a gap between the thing viewed and the viewer was thereby closed (cf. \textit{He came up to me}). In conformity to traditional views and in anticipation of cognitive linguistics Bolinger claimed that such semantic extensions were metaphorical: “the phrasal verb”, he said, ”is a floodgate of metaphor” (\citealt{bolinger_phrasal_1971}: xii).

There are therefore circumstantial and more perennial reasons for the importance given to semantic analyses of particles and prepositions in cognitive linguistics: the persistence and revitalization of empiricist views, an interest for the language\slash perception interface, promoted by cognitive science, a long-standing interest in “particles”\footnote{\textrm{ The term “particles” is intended to be neutral, as far as the morphosyntactic category of these forms is concerned (adverbs, prepositions or adpreps). The morphosyntactic behavior and categorization of these forms are not a priority of the studies of cognitive semantics we are considering here. }} on the part of linguists, the rise of semantics, of cognitive linguistics, and the avenues of research in lexical semantics opened up by prototype theory.

As far as I know, the first two studies of “particles” conducted in the spirit of cognitive linguistics were those of \citeauthor{lindner_lexico-semantic_1981} (\citeyear{lindner_lexico-semantic_1981}, her PhD thesis, supervised by Langacker) on \textit{out} and \textit{up}, and \citet{brugman_story_1981} on \textit{over}. The latter, though less documented than the former, was to rise to celebrity especially through its being exploited in Lakoff’s bestseller, \textit{Women, fire and dangerous things} (\citeyear{Lakoff1978})\footnote{According to \citet{dirven_does_2001}, the book sold around 100,000 copies.}, in which the description of polysemous items, like \textit{over}, was presented as an application of prototype theory (or rather, a simplified form of its psychological version). Lindner, Brugman and Lakoff all regarded spatial meanings as primary, a localist bent that was further shored up, in Lakoff’s case, by his empiricist conception of metaphorical thinking.

These first studies and the numerous ones of the same style which followed share a number of characteristics: they are semasiological, taking as their object a single lexical item at a time, rather than a system (contrary to what was done by Bennett and Leech for instance); there is a tendency to neglect pragmatic and contextual factors, i.e. variations of use conditioned by a contrastive emphasis on a specific kind of information, given the system of expressions available in the language; as a consequence of the foregoing, polysemy tends to proliferate, and little is done to reduce it to a minimal set of features (unlike, e.g., in \citealt{pottier_systematique_1962}; for a discussion, \citealt{fortis_probleme_2009}); finally, although cognitive semantics can be expected to live up to its name only if it can lay a serious claim to the psychological validity of its analyses, in the overwhelming majority of the cases, no attempt is made at devising a method other than introspective, e.g. no psychological experimentation is conducted (\citealt{sandra_network_1995} being a rare exception). On the whole, then, and if we abstract away from their philosophical backdrop, more or less explicitly articulated, little seems to distinguish these early studies from \citet{aristotle_physics_1957} on \textit{ἐν}, \citet{leibniz_analysis_1986} on \textit{ad}, \citet{harris_hermes_1773} on \textit{over}, and \citet{condillac_cours_1775} on \textit{de}.

\subsection{ Neolocalist accounts of thematic roles: Gruber and Jackendoff}

We now come to the last strand of neolocalism, the set of theories dealing with thematic roles. There are, among these accounts, commonalities which set them apart from the past localist tradition; this is not to say they are detached from this tradition, but their relation to past localism is quite variable and at times rather obscure. In spite of this difficulty, the observation that they all emerge in a rather narrow period, approximately spanning the years 1965--1975, cannot but incite us to have a closer look at possible lines of influence and transmission.

Chronologically, the first theory with a localist (or better, localistic) bent is the one proposed by \citet{gruber_studies_1965}. In this study (his dissertation) Gruber sets himself the task of providing a \textit{syntactic} (not semantic) representation that should state the kind of complements a verb can occur with, that is, whether it takes a direct object, and/or a prepositional complement, which kind of preposition is compatible with it, and the selectional restrictions that hold of its arguments. This syntactic level of representation or, as Gruber calls it, “prelexical” structure, consists in the marking of “deep” cases incorporated in verbs in the form of prepositions. For example, verbs which, like \textit{obtain}, take a goal as subject are noted as \textsc{TO~V}, the \textsc{TO} subject argument (to the left of the verb) being obligatorily incorporated. Note again that the interpretation of these prelexical structures is left to a semantic component, in conformity with the \textit{Aspects} model; this merely attests to the reluctance of mixing syntax and semantics, in a time when generative semantics has not yet really caught on. It is in circumscribing the number of roles and identifying their semantic import that spatial relations show their usefulness. For Gruber, verbs expressing concrete motion (“positional transition”) incorporate roles which, for some of them, are common to other, “abstract” fields; further, all abstract roles appear to have a concrete counterpart. These generalized roles are: \textit{theme} (the located or moving entity), \textit{source}, \textit{goal}, \textit{location, agent}. Generalized \textit{source} and \textit{goal} are for example instantiated in domains like “activity” (\textit{The climate changed from being rainy to manifesting the dryness of the desert})\textit{, }“possession” (\textit{John obtained a book from Mary, John gave a book to Bill})\textit{ or }“abstract transfer” (\textit{John reported to Mary from Bill that he would like to see her}). Importantly, such analogies do not give rise to any sweeping declaration on the cognitive primacy of space. In fact, Gruber explicity declares that on his view “there is no particular priority intended for the sense of concrete motion” (\citealt{gruber_studies_1965}: 48), that is, “motional” seems to be a substitute for “dynamic”. In the terminology of Anderson, Gruber’s theory should therefore be classified as belonging to the family of localistic analyses.

The conundrum Gruber confronts us with is the following: his text simply contains no reference to any previous study. We seem to be dealing with a rediscovery based on linguistic facts such as the cross-domain analogies cited above. Like Wüllner before him, \citet{gruber_look_1967} analyzes the accusative as a goal-case, yet there is no evidence he was aware that he had predecessors. Nor was \citet{jackendoff_rules_1969}, in all likelihood, when he first borrowed Gruber’s list of “thematic relations”.\footnote{Jackendoff first acknowledged localism had a past in his 1983 book, referring back to Anderson as a source \citep[188]{jackendoff_semantics_1983}. But even at this date, it may be doubted that Jackendoff had delved very deep into historical matters, since he declared, without further ado, that Gruber’s essay offered the best demonstration for localist(ic) ideas to date.} The context in which this borrowing took place is rather puzzling too, since there was no clear motivation, from Jackendoff’s own point of view, for adopting Gruber’s thematic relations rather than Fillmore’s cases. For thematic relations were introduced in Jackendoff’s discussion merely to state a condition on reflexivization and the control of ∅ arguments in complement clauses, and to this purpose the hierarchy of thematic relations proposed by Fillmore would have served as well. Thus, the semantic part of the condition on reflexivization (the \textit{thematic hierarchy condition}) stated that a reflexive should not be higher on the hierarchy \textsc{agent} {\textgreater} \textsc{location}, \textsc{source}, \textsc{goal} {\textgreater} \textsc{theme} than its antecedent, which ruled out *\textit{John was shaved by himself} (with \textit{John} as theme, \textit{himself} as agent) or *\textit{I talked about Thmug to himself} (with \textit{Thmug} as theme and \textit{himself} as goal). The only remote allusion to a cognitive motivation can be found in the very general assertion that “to suppose a universal semantic representation is to make a strong claim about the innateness of semantic structure”, and that “presumably the semantic representation is very closely integrated into the cognitive apparatus of the mind” (\citeyear{jackendoff_rules_1969}: 1).

It should be noted that the thematic hierarchy condition was but one piece in a machinery designed to push back generative semantics, i.e. was aimed at supporting Chomsky’s rival version of TG. The condition was regarded as an interpretative rule filtering out unacceptable interpretations, not as a semantic condition formulated at deep structure. And globally, the book (and its updated version of 1972) was intended as a refutation of the level-mixing infesting GS, and a rehabilitation of the role of surface structure in semantic interpretation.

The cognitive justification for positing local thematic relations is much more elaborated on when Jackendoff, from 1976 on, embarks on the description of autonomous semantic representations, with the aim of providing explicit procedures mapping these representations to syntactic structure. In essence, Gruber’s generalized roles are now reformulated as “conceptual” predicates (first termed “semantic functions”) specified with respect to ontological domains, but with cognitive primacy being granted to the “positional” domain and to “the innate conception of the physical world” (\citeyear{jackendoff_toward_1976}: 149). For example, changes of state will be paraphrased by means of the predicate GO specified in the domain called “identificational” (Gruber’s term), that is, concerning the inner properties of an entity, or “circumstantial”, that is, referring to events. A simple illustration of the latter case is offered in the following example (\citeyear{jackendoff_toward_1976}: 129), in which \textsc{bill} takes on the status of theme by virtue of being the first argument of \textsc{go}, while the other arguments are, respectively, a circumstantial source (unspecified) and a circumstantial goal:

\ea
\ea John caused \textsc{bill} to scream.
\ex \textsc{cause} (\textsc{john}, \textsc{go}\textsubscript{Circ} (\textsc{bill}, y, \textsc{bill scream}))
\z
\z

Just like in Gruber, Anderson and later Lakoff, a central argument in favor of such semantic functions was the observation that inferences valid in the spatial realm analogically carry over to non-spatial domains, with concomitant variations. For example, while \textsc{go} from X to Y generally implies that the theme is no longer at X when it has reached Y, whether in space or when a change of state occurs, the inference does not hold in contexts referring to spatial extents such as \textit{The road extended\slash reached from Altoona to Johnstown}, nor to abstract identificational extents as in \textit{This theory ranges from the sublime to the ridiculous}.

In subsequent texts, this account of “conceptual structure” will not undergo any substantial change. In \textit{Semantics and cognition} (\citeyear{jackendoff_semantics_1983}: 188), it will be claimed to rest on a fundamental hypothesis of basic and presumably universal “semantic functions”: “In any semantic field of [\textsc{events}] and [\textsc{states}]”, says Jackendoff, “the principal event-, state-, path-, and place-functions are a subset of those used for the analysis of spatial location and motion.” However, as localist as this statement may sound, Jackendoff’s final position may be more adequately described as localistic. On a speculative note, he declares himself in favor of the idea that thematic structure is a generalized abstract organization which is not grounded in spatial metaphors, at least in synchrony and during ontogeny (phylogeny being another matter; \citeyear{jackendoff_semantics_1983}: 210). 

\subsection{Neolocalist accounts of thematic roles: Anderson}\largerpage

Anderson’s case grammar may be regarded as the most sophisticated localist account of thematic roles; it is remarkable in another respect: Anderson has always taken care to refer to past localist ideas (notably the doxography contained in Hjemslev’s \textit{Catégorie des Cas}) and he has often presented his theory as an ongoing debate with rival models, two aspects which confer to his texts a historiographical dimension. The complexity of Anderson’s theory prevents us from presenting but a rough sketch.{\interfootnotelinepenalty=10000\footnote{\citet{bohm_investigating_2018} provides an excellent and updated overview of Anderson’s theory. The historical background and the evolution of Anderson’s case grammar is the subject of my paper in the same volume (\citealt{bohm_andersons_2018}).}}

His initial motivations (\citealt{anderson_ergative_1968}) for introducing thematic roles (or, in his terms, \textit{case roles\slash relations}) into the base component of grammar are close to those voiced by Fillmore (\citeyear{fillmore_toward_1966}, \citeyear{bach_case_1968}) around the same time. Like Fillmore, Anderson is unsatisfied with the configurational definition of grammatical relations proposed in generative grammar, a definition which he deems insufficient to capture generalities which hold within English itself as well as in the cross-linguistic comparison of accusative and ergative languages. In a nutshell, grammar should first and foremost be based on subcategorization, and subcategorization is expressed in terms of “deep” cases. At this early stage of Anderson’s thinking, these cases are the Nominative (i.e. ‘absolutive’) and the Ergative. At a deep level, ergative and accusative languages are non-distinct. For English, various rules take care of the accusative realignment of arguments, in other words, and this a leitmotiv of Anderson’s theory, grammatical relations are a superficial phenomenon and should not be expressed at deep structure (\textit{pace} generative grammar).

Although Anderson’s case grammar will undergo some changes with the years, its essential characteristics are presented in his \citeyear{anderson_grammar_1971} book, \textit{The grammar of case}, and will remain largely unaltered (\citealt{bohm_andersons_2018}). His syntactic model consists of a dependency grammar in which predicates govern case roles taking as their dependents nominal expressions. Unlike Fillmore, and like Hjelmslev, he insists on the fact that case roles should exhibit systematicity, i.e. should form a system whose members are differentiated by features belonging to a semantic field (included their ∅ marked counterparts). This is where his “localist hypothesis” enters the stage: syntactic representations are to be “constructed out of predications that are locational or directional or non-locative nondirectional” (\citealt{anderson_essay_1973}: 10).

In his \textit{Grammar of case}, Anderson settles on a list of four fundamental subcategorial features of the functional category ``case role'': Nominative, Ergative, Locative and Ablative (capitalized here, in order to distinguish them from surface cases). This list is of course reminiscent of semi-localist accounts of the past, and in this Anderson may have been inspired by Hjelmslev and by timely remarks made by Lyons. Shortly before Anderson put forth his “localist hypothesis”, \citet{lyons_note_1967} had drawn attention to cross-linguistic parallels between locative, existential and possessive constructions, and more to the point, his linguistic textbook of 1968 included a discussion which laid some importance on the distinction between grammatical and local cases and alluded to a minimal inventory of three essential local cases (\textsc{at}, \textsc{from}, \textsc{to}). Now, Anderson’s own list of fundamental categorial features is unorthodox and perplexity may arise: what about the goal-case? Or the experiencer? And the instrumental case, which can obviously cooccur with an ergative, and which Fillmore had for this reason separated from it? Can a goal-case be conflated with Anderson’s ``Nominative''? But is it plausible to treat the Nominative and what is usually assimilated to ``dative'' uses as instances of one and the same case? On the other hand, since we are not dealing here with surface cases but with a localist description of semantic roles, it would be tempting to reduce agentivity (i.e. the Ergative) to the notion of source (i.e. the Ablative): is the Ergative not superfluous from a localist point of view?

We cannot go here into all the subtleties involved in the discussion of these issues and the solutions proposed by Anderson. Suffice it to say that one solution for capturing fine-grained semantic distinctions among roles is to posit complex subcategorial features: experiencers, for example, will be glossed as both ergative and locative, [\textsc{erg}, \textsc{loc}], and distinguished from simple datives, [\textsc{loc}]. A further solution will be to introduce contextual rules: a goal will be coded as a Locative in the context of a predicate which subcategorizes for a source, i.e. an Ablative. An Instrumental in a circumstantial phrase will then be glossed as a variety of path, i.e. as [\textsc{loc}, \textsc{abl}]. Finally, as regards the Ergative and the Ablative, their semantic kinship will prevail and in his \citeyear{anderson_case_1977} book, Anderson will explicitly mark their commonality through the feature ``source'', restricting the name \textit{Ablative} to locational sources. This modification leads to the following, penultimate version of his system of case roles (note the ``nominative'' has been renamed \textit{Absolutive}).

\begin{table}
\begin{tabular}{cccc}
\lsptoprule
\textsc{abs} & \textsc{loc} & \textsc{erg} & \textsc{abl}\\\midrule
& place & & place\\
& &  source & source \\
\lspbottomrule
\end{tabular}
\caption{Features of cases in \citet{anderson_case_1977}} 
\end{table}

This abstracts away from contextual effects, for instance that a Locative can be a goal in the presence of a locational source. No mention will be made either of the reanalyses prompted by this featural reformulation. Again, expounding the theory in its details, its posterior evolution and its final stage would take us too far afield. Our first purpose was to convey to the reader a sense of Anderson’s approach. However, a few concluding words need to be said about the localist(ic) spirit in which this theory was elaborated.

In Anderson’s discussions, cognitive motivations have never loomed large, although he does occasionally hint at the importance of spatial cognition (e.g. \citealt{anderson_localism_1994}). This relative agnosticism is the reason why he initially described his theory as “localistic”. He is well-informed about past localist accounts, and his initial semi-localist grammar evinces this familiarity, as does his foray into a localist description of aspect and tense, for which he refers back to \citet{darrigol_dissertation_1827} on Basque (\citealt{anderson_essay_1973}). His quest for systematicity, with a minimal list of basic features and a ∅-marked, or neutral, case (the absolutive) reflects a modern, structuralist perspective. Modern too is the fact that his main purpose is not to deliver a semasiological analysis of surface cases; like Fillmore, he envisages cases as roles abstracted from their surface realizations. This separation of levels and the formal, generative-like style of his analyses entail, as for Fillmore and Jackendoff, that rules mapping his deep structures to their surface realizations have to be explicitly formulated. Or in other words, his theory bears the mark of the generative period and has to be contextualized in the debates surrounding the relation of deep structure to semantics and grammatical relations.

\section{Other perspectives}

In the foregoing, I have made no pretense at presenting all the facets of modern localism. I see at least three aspects that could have been examined more closely and that I will briefly mention here. First, we have dealt here with the clearest instances of localism, or what \citet{descles_predication_1991} calls “pure localism”, in which a direct mapping is posited between spatial relations and linguistic meanings. Some other varieties of localism were therefore left out. Desclés’ own view would be more aptly characterized as a \textit{partial localism}, in which spatial reference points (\textit{repères}, a term also found in Culioli) are but a primitive stratum giving rise to a set of more abstract cognitive operations; in particular, states of affairs and events are situated with respect to enunciative reference points which cannot be regarded as purely spatial. Like Wundt, he also insists on the role of intuitive but non-spatial primitive notions, such as intentional control (\citeyear{descles_predication_1991}, \citeyear{descles_interactions_1993}). In a way similar to Desclés’ higher reference points, it might be questioned whether the visuo-spatial diagrams used by Langacker for the sake of representing semantic focus, headhood, tenses, aspects and modalities, the meanings of various relational expressions etc. are to be affiliated to localism. His first version of cognitive grammar went by the name of \textit{Space grammar}, but space was invoked first and foremost because he had adopted a diagrammatic representation of the linguistic strata making up a clause. On the other hand, and from an early date on, he has proposed semantic analyses one would be tempted to describe as localist: in a 1975 paper, for example, possessive structures which can be glossed as ‘x is by/at y’ are declared to issue from a locative metaphor (\citeyear{langacker_functional_1975}: 384--385). However, he denies that his diagrams are to be assimilated to the kind of visual imagery studied by psychologists (\citeyear{langacker_introduction_1986}: 6). The brand of diagrammatic localism advocated by Langacker is not the first of its kind. We feel that examining this family of theories would necessarily take us to territories unexplored here: the import of visual representations in linguistics and science at large; the relation of linguistics to place and space as themes scrutinized in physics, mathematics, philosophy and psychology, in other words, the role of the global epistemic context in the importance attributed to place and space, and the possible impact of changing conceptions of space on linguistic theories with a philosophical background. These issues overlap with those addressed in this volume by Chalozin-Dovrat, who claims that diagrammatic localism can be viewed as a way of furnishing scientific credentials by connecting linguistics to a category of Western science.

A second point we have not mentioned relates to what is designated today as “grounded cognition”, a perspective according to which meanings are based on multimodal representations which are reenacted when these meanings are activated (\citealt{barsalou_grounded_2008}). This trend partly takes its origin in cognitive linguistics and the conceptual metaphor theory of Lakoff and Johnson, but it has now developed into a body of research which extends beyond the boundaries of linguistics. Of relevance for us here is e.g. Boroditsky’s psychological work on facilitation (or interference) caused by visually perceived motion on the processing of temporal expressions (e.g. \citealt{boroditsky_does_2001}). Views of this sort do not only argue for what we designated as morphogenetic localism; they hypothesize that spatial conceptualization is, as it were, constantly active when spatial \textit{and} metaphorically spatial forms are being processed.

Our third omission involves the junction between diachronical linguistics and the more or less explicit neo-empiricist views we have reviewed above. Our retrospective has shown there is no novelty in this diachronic twist. For historians, it may be significant that the later strand was revived by people who were connected with the tradition of historical linguistics, and were in a position to reinstate the notion of grammaticalization.\footnote{On cognitive linguistics as reactivating themes and perspectives of historical linguistics, see \citet{geeraerts_theories_2010}.} Bybee and Traugott illustrate this revival (for a localist application, see \citealt{traugott_spatial_1975}), which got further reinforcement from typological research in the spirit of Givón: we are alluding here to the work of what may be identified as the Cologne school, brought together in the 1970s on the occasion of a project on language universals. Among the linguists involved, Bernd Heine achieved perhaps the greatest notoriety. He would later join forces with cognitive linguistics and instill localist elements in his research on grammaticalization, showing for instance that possessive constructions result from different metaphors, some of them spatial or partly spatial: ‘Y is located at X’, ‘Y is (intended) for X\textsubscript{\textsc{goal}}’, ‘Y exists from X’, ‘X grasps Y’ etc. \citep{heine_cognitive_1997}. Of course, although the localist hypothesis fares well with respect to the semantic evolution of adpositions, it is by no means implied that spatial expressions are the only starting points. Counter-localist evolution may even be observed, but this appears to be exceptional (for a discussion on Romance, cf. \citealt{fagard_espace_2010}).

\section{Conclusion}

In the preceding sections, we have seen that localist ideas were deeply entrenched in the history of linguistics and recurred in different contexts. However, in spite of this historical depth, some neolocalist views cannot be straightforwardly traced back to their antecedents. With the exception of Lyons’ scattered reflections and Anderson’s case grammar, the link which binds neolocalist theories of thematic roles with their past seems to be rather thin. Indeed, the legacy of past localist accounts is at times so dimly visible that one may be tempted to speak of reinvention, as we saw in the case of Talmy, Gruber and, in his wake, Jackendoff. In semasiological studies, especially those devoted to prepositions, the connection is somewhat more apparent. On one hand, conceptual metaphor theory and prototype theory have clear links to a tradition we may describe as empiricist (from Aristotle to Locke and beyond). Empiricist tenets were in the intellectual horizon of linguists and were congenial to some before they could really be supported by extensive research of their own. For example, although it is not clear if empiricist philosophy was on Langacker’s mind, it is rather striking to see him rediscover the Lockian problem of the general triangle, or use the Humean archetype of a ball hitting another to illustrate prototypical causation \citep[13]{Langacker1991}. As for Lakoff and Johnson, philosophically oriented reflection was the backdrop, among other influences, to their conceptual metaphor theory \citep{johnson_introduction:_1981}; however, they seem to have both underestimated the extent to which their views had been anticipated and exaggerated the novelty of their brand of empiricism, sometimes by omitting sources \citep{arigne_prototype_2018}. On the other hand, and beyond this revival of philosophical grammar, localist ideas substantiated by linguistic or psychological evidence collected in the pre-cognitive era had, as it were, percolated into semantics. We can refer the reader to what has been said above about Osgood, Nagy and Lindner. 

Whether the history recounted here concerns very general ideas whose paths of transmission are for this reason difficult to map, or more specific claims with clear antecedents, we may at least hope that our depiction of their intellectual environment goes some way toward understanding why they could be put forward and received favorably from the 1960s on. 

% \section*{Acknowledgements}

{\sloppy\printbibliography[heading=subbibliography,notkeyword=this]}
\end{document}
