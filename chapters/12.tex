\documentclass[output=paper]{../langscibook}
\author{Jacopo D’Alonzo\affiliation{Sapienza Università di Roma}\orcid{}} 

\title{Per una semiologia materialista e dialettica: Trần Đức Thảo critico di Saussure}
\shorttitlerunninghead{Per una semiologia materialista e dialettica}

\abstract{Specialista della fenomenologia e vicino agli esistenzialisti francesi, marxista e militante anticolonialista, Trần Đức Thảo ha dedicato gran parte della propria carriera intellettuale alla descrizione dell’origine ontogenetica e filogenetica della coscienza e del linguaggio. In tale contesto, Thảo ha proposto una semiologia generale che doveva rendere conto dello sviluppo progressivo delle capacità simboliche. Nel nostro lavoro, studieremo le implicazioni teoriche della critica che Thảo indirizza al modello semiologico proposto nel CLG (e più in generale alle letture strutturaliste del CLG) e introdurremo così la nozione di “linguaggio della vita reale”.}


\IfFileExists{../localcommands.tex}{
  \addbibresource{../localbibliography.bib}
  % add all extra packages you need to load to this file  

\usepackage{tabularx,multicol} 
\usepackage{url} 
\urlstyle{same}

\usepackage{listings}
\lstset{basicstyle=\ttfamily,tabsize=2,breaklines=true}

\usepackage{./langsci/styles/langsci-optional}
\usepackage{./langsci/styles/langsci-lgr}
\usepackage{./langsci/styles/langsci-gb4e}

  \newcommand*{\orcid}{}
\newcommand*{\markindex}{}

  %% hyphenation points for line breaks
%% Normally, automatic hyphenation in LaTeX is very good
%% If a word is mis-hyphenated, add it to this file
%%
%% add information to TeX file before \begin{document} with:
%% %% hyphenation points for line breaks
%% Normally, automatic hyphenation in LaTeX is very good
%% If a word is mis-hyphenated, add it to this file
%%
%% add information to TeX file before \begin{document} with:
%% %% hyphenation points for line breaks
%% Normally, automatic hyphenation in LaTeX is very good
%% If a word is mis-hyphenated, add it to this file
%%
%% add information to TeX file before \begin{document} with:
%% \include{localhyphenation}
\hyphenation{
affri-ca-te
affri-ca-tes
dis-ci-plin-ary
}

\hyphenation{
affri-ca-te
affri-ca-tes
dis-ci-plin-ary
}

\hyphenation{
affri-ca-te
affri-ca-tes
dis-ci-plin-ary
}

  \togglepaper[12]%%chapternumber
}{}



\begin{document}
\begin{otherlanguage}{italian}
\maketitle

\section{Introduzione} 

 Negli ultimi anni, l’attenzione della comunità scientifica si è rivolta sempre più frequentemente ai rapporti tra \is{fenomenologia}fenomenologia, \is{strutturalismo}strutturalismo e saussurismo (\citealt{de_palo_saussure_2016}; \citealt{aurora_filosofia_2017}). A questo proposito l’opera del filosofo vietnamita Trần Đức Thảo (1917--1993), specialista della fenomenologia husserliana, merita una menzione speciale. Da un lato la (relativamente) celebre riflessione di Thảo sulla filosofia di Edmund Husserl (1859--1938) non può essere isolata dalle ricerche più generali che Thảo ha dedicato al linguaggio. D’altro canto, la teoria dell’origine del linguaggio proposta da Thảo – che ha solo di rado attirato l’attenzione degli studiosi – si inscrive nel contesto di una polemica nei confronti, sia dello strutturalismo generalizzato degli anni Sessanta, sia della fenomenologia husserliana. In simile quadro, Thảo ha condotto una critica poco nota a una certa lettura del modello semiologico introdotto nel \textit{Cours de linguistique générale} (CLG) di Ferdinand de Saussure (1857--1913). Nel lavoro che segue, proveremo a tracciare le linee direttrici della critica di Thảo al CLG per metterne in evidenza le numerose implicazioni teoriche.

\section{La vita e l’opera di Trần Đức Thảo} 

Trần Đức Thảo può essere considerato uno dei più importanti intellettuali di lingua francese – tra i quali bisogna annoverare Jean-Paul Sartre (1905--1980) e Maurice Merleau-Ponty (1908--1961) – che, negli anni Quaranta, hanno discusso la possibilità di conciliare \is{marxismo}marxismo e \is{fenomenologia}fenomenologia (cf. \citealt{thao_marxisme_1946,thao_existentialisme_1949}). Presto, però, si produce una rottura profonda tra Thảo e l’ambiente parigino (\citealt{thao_les_1950,thao_phenomenologie_1951} 1951).\footnote{{Per maggiori ragguagli bibliografici su Thảo si veda \citealt[1--11]{thao_formation_1991}; \citealt{thao_note_1993,giao_ecrits_1988,hemery_itineraire_2013,thao_les_2004,thao_quelques_2013,feron_qui_2014}}} Dopo aver condotto i suoi studi superiori presso l’ENS di Parigi tra il 1939 e il 1942 ed essere diventato una voce autorevole nel \textit{milieu} filosofico parigino, malgrado la sua origine straniera, nel 1951 Thảo decide di tornare nel suo paese natale, il Vietnam, per prendere parte alla lotta di liberazione nazionale. Nel 1951, poco prima della sua partenza per il Vietnam, dove resterà sino al 1991, Thảo raccoglie e pubblica i risultati delle sue ricerche sulla fenomenologia condotte durante il decennio precedente.

\textit{Phénoménologie et matérialisme dialectique} (PMD) si apre con un’introduzione alla fenomenologia husserliana e una descrizione del metodo fenomenologico “d’un point de vue purement historique” (\citealt[5]{thao_phenomenologie_1951}). \citet[7]{thao_phenomenologie_1951} considera la fenomenologia di Husserl una forma di idealismo. La riduzione trascendentale alla sfera del vissuto come polo costituente dei fenomeni avrebbe dovuto essere molto più radicale: in tal caso, la fenomenologia avrebbe mostrato che:
(1) l’ego trascendentale non è che l’ego storico e concreto;
(2) la soggettività trascendentale è essa stessa costituita dal movimento della storia naturale e sociale che la precede;
(3) il vissuto cosciente è sempre preceduto dall’organismo e dalla sua attività;
(4) l’esperienza ante-predicativa si situa a livello della vita animale e non a quello della vita umana;
(5) il vissuto è l’aspetto astratto della \is{vita reale}vita reale.

Nella seconda parte di PMD appare per la prima volta un progetto che impegnerà Th\textlatin{ả}o per il resto della sua vita, cioè una descrizione delle dinamiche naturali e storiche che hanno favorito lo sviluppo della coscienza dagli organismi unicellulari all’umanità; una descrizione condotta dal punto di vista di una metafisica materialista e secondo una logica dialettica.

In Vietnam, Thảo è presto escluso dalla vita politica e accademica del paese in seguito a un conflitto con il Partito comunista a proposito alle libertà civili a cui fa seguito un periodo di reclusione. Tuttavia, negli anni Sessanta, si impegna in un vasto progetto di ricerca sulle origini della coscienza e del linguaggio, come testimonia una serie di articoli \citep{thao_du_1969,thao_du_1969-1,thao_du_1970} riuniti poi nelle \textit{Recherches sur l’origine du langage et de la conscience} (RLC) nel 1973 e in cui Thảo cerca di rendere ragione della cognizione umana – seguendo le indicazioni di alcuni classici del marxismo – a partire della vita pratica e collettiva dei nostri predecessori ominidi (\citealt{federici_viet_1970,caveing_recherche_1974,haudricourt_tran_1974,frederic_tran_1974}).

La teoria di Thảo s’impernia su tre ipotesi: (1) la coscienza emerge nel e grazie al linguaggio considerato nella sua materialità e nella sua funzione pratica e operativa; (2) il linguaggio non è un oggetto, ma la mediazione tra l’uomo e la realtà, tra uomo e uomo, e tra l’individuo e se stesso; (3) il linguaggio non può essere studiato come una realtà autonoma, ma bisogna osservarlo all’interno della vita sociale e pratica.

A fondamento del linguaggio umano, sia dal punto di vista diacronico che sincronico, ci sarebbero, secondo Thảo, alcuni segni naturali fondamentali (gesti, vocalizzazioni, espressioni fisiognomiche, ecc.) in cui la relazione tra il \is{significante}significante e il \is{significato}significato non è né arbitraria né convenzionale. Tali segni hanno, infatti, uno spiccato carattere corporeo e fanno parte della vita pratica e collettiva. La loro interiorizzazione psichica segue necessariamente il loro uso, nella misura in cui essi sono l’espressione immediata della vita del gruppo umano che li utilizza prima di divenire strumento d’espressione a disposizione degli individui.

\section{Alcuni aspetti della critica di Thảo al CLG}

Thảo pubblica tra il 1974 e il 1975 un articolo dal titolo \textit{De la phénoménologie à la dialectique matérialiste de la conscience} che serviva da introduzione, sia biografica che teorica, alle RLC. Th\textlatin{ả}o prende esplicitamente di mira la teoria del segno proposta nel CLG. Th\textlatin{ả}o reputava la sua ipotesi sull’origine del linguaggio, così come la \is{semiologia}semiologia che ne era il supporto teorico, in disaccordo radicale con alcune ipotesi del CLG, riprese poi anche dallo strutturalismo.

Preliminarmente va detto che Th\textlatin{ả}o ignorava la storia editoriale che ha condotto alla pubblicazione del CLG. Al di là della questione concernente le conoscenze che Th\textlatin{ả}o aveva a disposizione riguardo la redazione del CLG, quello che interessa mettere in luce è il fatto che la critica di Th\textlatin{ả}o aveva di mira una certa lettura del CLG e in particolare quella che difendeva un modello autonomista del linguaggio. Tuttavia, la lettura che egli offre del CLG investe più livelli teorici e va ben al di là della polemica nei confronti dello strutturalismo generalizzato (\citealt{chiss_structuralisme_2015} e \citealt{leon_historiographie_2013}), benché questo sia il suo punto di partenza.

\begin{enumerate}
\renewcommand{\labelenumi}{{\alph{enumi}})}

\item Negli anni Sessanta e Settanta non mancava un certo disaccordo con le tesi dello strutturalismo generalizzato e con una certa recezione, circolazione e interpretazione del CLG e la cui portata trascendeva gli orizzonti della linguistica (\citealt{dosse_histoire_1991,dosse_histoire_1992,puech_lesprit_2013a,puech_lesprit_2013b,lepschy_linguistica_1966}).

Tra le critiche dell’estensione all’insieme degli oggetti delle scienze umane e sociali del modello semiologico incentrato sulla \textit{langue} come forma sopra-individuale composta da elementi negativi e differenziali, si devono menzionare quelle provenienti dal fronte marxista. Nel loro insieme, le critiche di certi marxisti manifestano un posizionamento abbastanza omogeneo dal punto di vista strategico, ideologico e teorico. Per citare solo due esempi, si devono ricordare \citet{seve_structuralisme_1984} e \citet{lefebvre_au-a_1971}. Secondo questi autori, l’autonomia delle strutture conduce a pensare le stesse come entità metafisiche e a identificare l’ideale con il reale, i prodotti della scienza con la realtà effettiva (si tratta della stessa denuncia dello strutturalismo ontologico che si trova anche in \citealt{eco_struttura_1968}).

Th\textlatin{ả}o condivideva le stesse preoccupazioni dei suoi colleghi francesi di orientamento marxista, ma aveva un obiettivo differente: si trattava per lui di far emergere i principi fondamentali di una semiologia generale che gli avrebbe permesso di descrivere lo sviluppo filogenetico del linguaggio e della coscienza. E per questo motivo presenta il suo progetto semiologico come un antidoto per ridurre il campo d’applicazione dei due concetti chiave dello strutturalismo, quello di \is{arbitrarietà}arbitrarietà e quello di \is{valore}valore. Th\textlatin{ả}o decide dunque di volgersi alla fonte di queste due nozioni:

\begin{quote} 
En 1964, je reçus les premiers échos des succès retentissants du structuralisme dans les pays occidentaux. L’étude du \textit{Cours de linguistique générale} de Ferdinand de Saussure s’imposait comme une nécessité urgente (\citealt[39]{thao_phenomenologie_1974}).
\end{quote}

Da questo punto di vista, il CLG perde il suo status di oggetto storico e viene collocato nell’attualità del dibattito teorico.

\item Per Th\textlatin{ả}o il CLG è allo stesso tempo sia l’obiettivo di una critica rivolta a una prospettiva che vede nella \textit{langue} un’entità psichica, interiore, separata dalle pratiche linguistiche (che Th\textlatin{ả}o stesso riconosce confermata solo in parte dal testo del CLG), sia il luogo per una ricerca di un’altra semiologia possibile: “On peut lire ainsi à travers le texte du \textit{Cours de linguistique générale,} en transparence et pour ainsi dire en pointillé, la possibilité et la nécessité d’une autre sémiologie […]” (\citealt[40]{thao_phenomenologie_1974}). Il CLG è così il luogo di un progetto solamente abbozzato che bisogna ristabilire e sviluppare, quello, cioè, di una semiologia che prenda in conto un insieme di sistemi di segni più ampio di quello dei segni arbitrari:

\begin{quote}
Cependant l’auteur [Saussure] avait lui-même reconnu au début de la première partie du livre l’existence de toute une classe de signes présentés comme “signes naturels”, soit entièrement comme la pantomime, soit partiellement comme les signes de politesse, les symboles, etc. (\citealt[39]{thao_phenomenologie_1974})
\end{quote}

\citet[42]{thao_phenomenologie_1974} denomina il suo progetto “\textit{sémiologie dialectique}”, il cui oggetto sarebbe il “\textit{système général des signes intrinsèques, ou esthétiques}” (\citealt[40]{thao_phenomenologie_1974}), cioè il sistema di segni motivati che mostrano direttamente all’intuizione sensibile il loro significato.

\item Il CLG e in particolare le nozioni saussuriane dell’arbitrarietà e del valore sono per Th\textlatin{ả}o l’oggetto di una critica, nel senso kantiano del termine, vale a dire di una delimitazione della loro legittimità teorica. Si tratta di una questione ben nota tra i lettori del CLG (De \citealt{de_mauro_note_2011}[413--416]; \citealt{sofia_petite_2013}) e che, ancora oggi, è al centro del dibattito sull’eredità saussuriana (\citealt{rastier_valeur_2002,paolucci_identite_2012,laks_phonotactique_2012,coursil_valeurs_2015}). Anzitutto, si tratta per Th\textlatin{ả}o di stabilire i limiti della nozione di arbitrarietà rifiutando di prenderla come unico criterio per giudicare ogni tipo di segno e riabilitando così i segni motivati.

Th\textlatin{ả}o prende, poi, anche di mira la riduzione della significazione al valore dato che, secondo lui, la significazione concerne: i) il valore intrinseco di quei segni che non sono totalmente arbitrari; ii) il valore differenziale dei segni arbitrari; iii) la relazione tra i segni e la realtà trascendente (realtà materiale, esperienza pre-linguistica, ecc.).

Per delimitare il campo di validità dell’arbitrarietà e del valore, Th\textlatin{ả}o si impegna così in un’analisi della natura dei segni non totalmente arbitrari così come nello studio della motivazione intrinseca agli atti concreti di linguaggio:

\begin{quote}
    Déjà le langage ordinaire cherche à obtenir par l’intonation, le choix des mots et des tournures, la disposition de la phrase, une certaine \textit{qualité expressive} non réglée en tant que telle par les conventions du code, et qui contribue, parfois de manière décisive, à la signification (\citealt[39--40]{thao_phenomenologie_1974}).
\end{quote}

\citet[39]{thao_phenomenologie_1974} pensa dunque che il sistema dei segni intrinseci sia la condizione di possibilità dei sistemi di segni arbitrari di cui il miglior esempio sarebbe la lingua convenzionale delle discipline scientifiche, “qui vise essentiellement à exprimer distinctement des idées distinctes et, pour ce but, utilise autant que possible une langue conventionnelle”. Per Th\textlatin{ả}o la significazione dei segni intrinseci non dipende né dalla nozione di valore né da quella di \is{arbitrarietà}arbitrarietà, ma dalla produzione dei segni come si presenta nella vita pratica, nel “mouvement sémiotique matériel” (\citealt[39]{thao_phenomenologie_1974}). 

\item Il CLG è anche l’obiettivo di una critica che mira alle fondamenta empiriche della semiologia saussuriana. Saussure aveva spiegato che “le signe linguistique unit non une chose et un nom, mais un concept et une image acoustique” (\citealt[98]{saussure_cours_1995}). E l’immagine acustica (signifiant) “n’est pas le son matériel, chose purement physique, mais l’empreinte psychique de ce son, la représentation que nous en donne le témoignage de nos sens”. Benché Saussure abbia sostenuto che la natura mentale del segno dipende dalle esperienze percettive anteriori, secondo Th\textlatin{ả}o, la sua linguistica della \textit{langue} si rivolge solamente all’aspetto psichico dei fenomeni linguistici. Così Th\textlatin{ả}o fa notare che:

\begin{quote} 
    une telle théorie s’inspirait manifestement d’une psychologie qui n’est plus acceptable de nos jours [...]. En réalité on ne peut pas séparer le langage intérieur, à titre de pure opération idéale, des mouvements réels plus ou moins esquissés, de la voix et du geste. (\citealt[25--26]{thao_phenomenologie_1975})
\end{quote}
Di conseguenza, la semiologia saussuriana tradisce un mentalismo che non può più essere difeso.


\item Allo stesso tempo, la lettura di Th\textlatin{ả}o si rivolge anche ai presupposti filosofici del CLG. Per Saussure il pensiero, prima del linguaggio, sarebbe una “masse amorphe et indistincte” (\citealt[155]{saussure_cours_1995}) e la \textit{langue} ne ordinerebbe il flusso articolando quella massa in segmenti psichici (significanti e significati). In questo passaggio argomentativo, secondo Th\textlatin{ả}o, la semiologia di Saussure cade in errore poiché mostrerebbe l’ipotesi idealista che la sostiene. Quella di Saussure, più che una “théorie linguistique de la signification verbale” (cioè una semantica), è una “théorie gnoséologique du concept [cioè una teoria della conoscenza]” (\citealt[41]{thao_phenomenologie_1974}). I segni linguistici sarebbero la condizione di possibilità dell’articolazione del pensiero e lo articolerebbero in maniera esclusiva. Ne risulta che il pensiero articolato linguisticamente non necessiterebbe di alcun aggancio al mondo sensibile, corporeo e materiale esteriore al soggetto conoscente. In altre parole, Saussure cadrebbe nel medesimo errore dell’idealismo soggettivo che faceva corrispondere l’oggetto del pensiero con l’oggetto reale.

\item Th\textlatin{ả}o critica la nozione di valore secondo lo schema della critica all’economia politica volgare sviluppata da Marx nel \textit{Capitale} \citep{marx_kapital._1867}: “II est clair que la conception de la valeur économique exposée ici par l’auteur [Saussure], est précisément celle de \textit{l’économie politique vulgaire}” (\citealt[42]{thao_phenomenologie_1974}). Tra i lettori del CLG, ci sono stati alcuni che hanno provato a individuare le fonti economiche della nozione di valore (\citealt[68]{koerner_ferdinand_1973}; cf. anche \citealt[541]{sljusareva_notion_1980}; \citealt[2]{ponzio_valeur_2005}; \citealt{ponzio_linguistica_2015}; \citealt{joseph_saussures_2014}). \citet[n. 165]{de_mauro_note_2011} aveva già sottolineato che Saussure era a conoscenza dei dibattiti in economia politica. E come \citet{ponzio_valeur_2005} ha ricordato, Saussure condivideva con i teorici del marginalismo e dell’economia neo-classica numerosi principi metodologici. Ma ci sono stati anche studiosi che hanno contestato questo approccio \citep[235]{godel_les_1957} ed altri che invece si sono interessati alle fonti eminentemente linguistiche della nozione (\citealt[295]{auroux_deux_1985}; cf. anche \citealt[329]{swiggers_girard_1982}; \citealt{hasler_notion_2007}).

Th\textlatin{ả}o è tra quanti riconducono la nozione saussuriana di valore alla sua origine economica. E, come altri autori marxisti, era affascinato dalla comparazione tra linguaggio e economia (\citealt{lefebvre_langage_1966,goux_marx_1968}; cf. anche \citealt[207]{schaff_introduction_1968}; \citealt{baudrillard_pour_1972,latouche_linguistique_1973,bourdieu_leconomie_1977}, \citealt[180--181]{rossi-landi_linguistica_2016};\citealt{rossi-landi_il_2003,rossi-landi_linguistics_1977}). L’obiettivo di Th\textlatin{ả}o è, tuttavia, diverso da quello di altri marxisti; Th\textlatin{ả}o vuole mostrare l’identità del valore e della significazione nel CLG (159--160). Anche altri autori (\citealt[437]{malmkjaer_linguistics_1991}; \citealt[III, 406]; \citealt[91]{bouquet_semiologie_1992}: 91; \citealt[317]{bouquet_introduction_1997}; \citealt{rastier_valeur_2002}) difendono l’idea che ci sarebbe una coincidenza dei due concetti, di significazione e valore. Va, inoltre, sottolineato che il dibattito a questo proposito è stato recentemente riaperto a seguito della pubblicazione degli \textit{Ecrits de linguistique générale} (ELG) nel 2002.

Assimilato al valore economico, il valore linguistico è al punto di convergenza della relazione significato-significante (nei termini del CLG: parola-idea) e della relazione tra i segni (parole). Su questa base Saussure introdusse la distinzione tra il valore e la significazione: il valore di una parola dipende dalla relazione di comparazione con parole differenti, mentre la significazione verte sulla relazione di scambio di una parola con un concetto. Ma \citet[42]{thao_phenomenologie_1974} aggiunge che non sarebbe possibile spiegare come i segni differiscano tra di loro senza fare appello a una significazione che sarebbe già presente.

Tuttavia, per rendere pienamente ragione della natura della significazione, si dovrebbe tenere conto anche del valore di quel segno all’interno del sistema. Th\textlatin{ả}o descrive dunque una concezione \textit{dizionariale} del valore: per definire il valore di un segno, lo si deve definire per mezzo di altri segni del sistema. Il fatto che Saussure abbia preso come esempio la serie \textit{craindre}, \textit{redouter} e \textit{avoir peur} aumenta in maniera esponenziale le difficoltà connesse alla definizione di ciò che è la significazione. A partire da tale concezione, come ha mostrato \citet[74]{eco_semiotica_1984}, ci si ritrova immediatamente in un circolo vizioso: il valore presuppone la significazione e la significazione il valore. Anche i lettori più recenti di Saussure hanno messo in evidenza il medesimo problema \citep[67]{sanders_linguistic_2004}.\footnote{Tuttavia, non bisogna dimenticare che un limite al principio dell’arbitrarietà assoluta, da cui dipende la nozione di valore, era già stato indicato da Saussure stesso attraverso il fenomeno dell’arbitrarietà relativa (cf. \citealt[181--182]{saussure_cours_1995}.} Come si vedrà nella conclusione, la soluzione offerta da Th\textlatin{ả}o sarà quella di introdurre dei segni fondamentali che rendano conto tanto della significazione che del valore dei segni arbitrari.
\end{enumerate}

\section{Conclusioni}

Ciò intorno a cui ruotano le analisi di Th\textlatin{ả}o è l’idea che la significazione che caratterizza i segni di cui si occupa la semiologia dialettica è la condizione \textit{sine qua non} della significazione arbitraria:

\begin{quote}
    il [le système des signes motivés] présente directement dans l’intuition sensible le contenu de signification, auquel le second [le système des signes arbitraires] donne une expression conventionnelle, formellement plus distincte, pour le développer sur le plan discursif (\citealt[40]{thao_phenomenologie_1974}).
\end{quote}

Certamente Th\textlatin{ả}o sembra pensare l’arbitrarietà più in termini di relazione convenzionale tra il significato e il significante che in termini di non-motivazione. La convenzione deve, infatti, presupporre uno strato comunicazionale, cognitivo e sociale preesistente. Una tale osservazione potrà sembrare banale, tenuto conto della riflessione filosofica sul linguaggio dal \textit{Cratilo} di Platone in poi. Tuttavia, l’importanza della tesi di Th\textlatin{ả}o risiede altrove: egli vuole chiarire la relazione genetica che giustifica e fonda i sistemi di segni convenzionali (e le terminologie disciplinari). In altre parole, il valore saussuriano è traccia di un meccanismo cognitivo che supporta pratiche linguistiche molto sofisticate come la capacità di distinguere e definire mutualmente i termini impiegati. 

Esiste dunque un \textit{continuum} di sistemi di segni che si sono sedimentati in seno a una comunità e si sviluppano uno dall’altro. Seguendo in questo Karl Marx (1818--1883) e Friedrich Engels (1820--1895) (cf. \citealt[26, 30--31, 331]{marx_marx-engels-werke_1958}), Th\textlatin{ả}o chiama l’insieme delle pratiche linguistiche che fanno parte delle attività pratiche e sociali \textit{langage de la vie réelle} (Sprache des wirklichen Lebens). Th\textlatin{ả}o rifiuta, o riformula completamente, la nozione husserliana di esperienza ante-predicativa nella misura in cui il linguaggio della vita reale precede e fonda la coscienza individuale: esso è un sistema semiologico che serve a organizzare la produzione e altri aspetti della vita pratica anziché un insieme di espressioni linguistiche arbitrarie.

Allo stesso tempo, il riferimento al mondo reale eviterà di trovarsi rinchiusi nel mondo dei segni – come avverrebbe in una semiologia che non teorizzi la referenza.\footnote{{Bisogna ricordare, tuttavia, che non è mancato chi ha ammesso l’esistenza di una cripto-teoria della referenza in Saussure (\citealt{arrive_quen_2007,bouquet_semiologie_1992}).}} All’interno della vita pratica, è necessario che ogni discorso sia essenzialmente un dire qualche cosa a proposito di qualche cosa. A tal riguardo, Th\textlatin{ả}o afferma che ogni discorso concernente il mondo necessita dell’esecuzione di un gesto di indicazione più o meno abbozzato. Da qui la conclusione secondo la quale i sistemi di segni arbitrari poggino, in ultima istanza, sul gesto d’indicazione.

Th\textlatin{ả}o riconosce la debolezza semantica di questo genere di segni, che \citet[48]{eco_semiotica_1984} aveva già sottolineato. Il gesto di indicazione deve essere rinforzato da altre espressioni aventi funzione di metasemiotica, che aggiungano informazione all’indicazione. Nondimeno Th\textlatin{ả}o giustifica la necessità del gesto di indicazione perché esso radicherebbe il discorso nello spazio-tempo così da determinare la referenza e far funzionare un minimo di differenziazione tra i segni:

\begin{quote}
[…] tous les autres signes qu’on pourra ajouter, fonctionneront non pas comme “interprétants” du signe de l’indication, mais comme moyens de présenter les propriétés particulières de l’objet indiqué, ce qui est tout à fait différent. Le signe de l’indication signifie simplement qu’il s’agit de cet objet même: le “\textit{ceci}” comme réalité objective donnée à l’intuition sensible, et \textit{rien de plus}. (\citealt[62]{thao_recherches_1973})
\end{quote}

Si capisce allora cosa conduca Th\textlatin{ả}o a limitare la validità della nozione saussuriana del valore. La significazione non può essere ridotta alla relazione differenziale tra segni, poiché il valore saussuriano è qualche cosa che si aggiunge a una significazione preesistente nata sul terreno delle interazioni reali con l’ambiente sociale e fisico.

Un bilancio storico-epistemologico della lettura che Th\textlatin{ả}o offre del CLG non può ignorare un paradosso: benché Th\textlatin{ả}o dichiari di essere ormai lontano dalla fenomenologia, egli sembra applicare alla lettura strutturalista del CLG quell’approccio fenomenologico-genetico che era stato al centro del dibattito filosofico francese degli anni Quaranta e Cinquanta (specialmente in Sartre e Merleau-Ponty). Th\textlatin{ả}o non è certo interessato a mettere in campo un’indagine filologica del CLG, bensì a opporre allo strutturalismo generalizzato degli anni Sessanta e al primato della sincronia un’indagine che mostri il primato filogenetico, diacronico e assiologico della genesi sulle strutture. Questo obiettivo incontra le preoccupazioni di una certa tradizione marxista attenta a ricondurre le costruzioni ideali alla loro origine materiale – concezione non lontana dalla \textit{Lebenswelt} husserliana orientata ora però verso le condizioni di produzione e riproduzione sociale. Allo strutturalismo marxista di Louis Althusser (1918--1990), Th\textlatin{ả}o contrappone dunque un marxismo attento alle esigenze metodologiche che erano emerse nel dibattito filosofico sulla fenomenologia nell’immediato dopoguerra.


{\sloppy\printbibliography[heading=subbibliography,notkeyword=this]}
\end{otherlanguage}
\end{document}
