\documentclass[output=paper]{langsci/langscibook} 
\author{Bruno Courbon\affiliation{Université Laval}\orcid{}} 
%\ORCIDs{}

\title{Aux origines de la notion de polysémie en français : la formation du concept\footnote{\textrm{Ce texte s’inscrit dans le prolongement d’un précédent travail (\citealt{courbon_sur_2015}). On y trouvera moins de développements relatifs à ce qui fait la spécificité du point de vue de Joseph Halévy (1827-1917) et de Michel Bréal (1832-1915) sur la \is{polysémie}polysémie ; cependant, l’accent est mis sur deux aspects : 1) d’une part, la continuité entre a) la conception générale de ce qui est intuitivement perçu comme de la \is{polysémie}polysémie et b) la conceptualisation nouvelle que rend possible l’importation du \is{terme}terme \is{polysémie}\textit{polysémie} en sémantique, et, d’autre part – et dans une moindre mesure –, la distribution entre l’usage du terme chez Halévy et l’utilisation conceptuelle spécifique qu’en fait Bréal ; 2) l’appartenance du concept de \is{polysémie}polysémie à une mythologie des fondations (celles de « la sémantique » comme discipline), qui n’aurait sans doute pas déplu à Bréal, mythologue et découvreur de talent.}}}

% \title{\texorpdfstring{Word formation and word history:\\ The case of
%  \textsc{capitalist} and \textsc{capitalism}}{Word formation and word
% history:  CAPITALIST and
% CAPITALISM}}

% \renewcommand{\lsCollectionPaperFooterTitle}{Word formation and word
% history: The case of  \noexpand\textsc{capitalist} and
% \noexpand\textsc{capitalism}}

 

\abstract{For at least 80 years, the name of Michel Bréal has been both associated with the development of the field of \is{semantics}semantics within French linguistics, and with the invention of the \is{term}term and \is{concept}concept of polysémie (polysemy), in the 1880s. The present contribution deals with the origin of the \is{concept}concept of \is{polysemy}polysemy, when Bréal used it for the first time. Although Bréal contributed to define \is{polysemy}polysemy thoroughly, neither the idea, nor the term were new. First, the term was coined by Joseph Halévy; moreover, \is{dictionary-makers}dictionary-makers and 19th century linguists were used to conceive of \is{lexical meaning}\text{lexical meaning} in terms of “\is{polysemy}polysemy”.}

\begin{document}
\maketitle

\begin{quote}
    \begin{small}
[…] la pratique métalinguistique des lexicographes pose le problème de l’émergence des notions avant leur dénomination. Le fait de nommer des concepts permet de circonscrire des portions de réalité, de reconnaître des séquences de problématisation et de questionner des phénomènes que l’acte de dénomination vise à cerner. (\citealt[23]{bisconti_sens_2016})
    \end{small}
\end{quote}
     

\section*{Introduction} 
La \is{polysémie}polysémie peut être considérée comme un cas emblématique dans l’histoire des idées linguistiques. Il s’agit d’abord d’un phénomène intuitif, dont les sujets parlants font consciemment l’expérience lorsqu’ils remarquent qu’une même forme lexicale sert à parler de réalités différentes que relie une impression commune\footnote{ \textrm{Ces éléments appartiennent toujours à la définition du \is{concept}concept de \is{polysémie}polysémie : «~un mot polysémique (un \is{polysème}polysème) est un mot qui rassemble plusieurs sens entre lesquels les usagers peuvent reconnaître un lien~» (\citealt[94]{nyckees_semantique_1998}).}}. Dans un second temps, cette expérience donne lieu à des propos qui indiquent la conscience d’un «~fait \is{sémantique}sémantique~». Les \is{lexicographes}lexicographes usent ainsi depuis longtemps d’une terminologie discrétisante pour parler de la pluralité sémantique des formes lexicales («~ce mot a plusieurs sens~», «~le mot \textit{X} reçoit plusieurs acceptions~»…). Cette vision analytique du \is{sens lexical}\text{sens lexical}sens lexical est clairement exprimée depuis le \textsc{xvii}\textsuperscript{e}~siècle dans la tradition métalinguistique de langue française\footnote{ \textrm{Nous ne voulons pas dire par là que l’idée de ce qui est désormais appelé «~\is{polysémie}polysémie~» n’existerait pas avant le} \textrm{\textsc{xvii}}\textrm{\textsuperscript{e}}\textrm{~siècle, mais seulement souligner le fait qu’au moment où le terme} \textrm{\textit{polysémie}} \textrm{est forgé, cette conception était chose relativement banale dans les pratiques de description du français, notamment en \is{lexicographie}lexicographie. En fait, cette conception remonte sans doute bien avant les premières attestations qu’on en trouve dans l’Antiquité : qu’une forme lexicale à valeur référentielle spécifique puisse servir à parler de choses relativement différentes bien qu’en partie apparentées est en effet un phénomène suffisamment fréquent et saillant pour que les sujets parlants en aient conscience assez tôt dans leur expérience linguistique.}}. Il faut pourtant attendre la fin du \textsc{xix}\textsuperscript{e}~siècle pour que l’idée intuitive qu’à une même forme lexicale puissent correspondre plusieurs \is{significations}significations prenne la forme abstraite du \is{concept}concept de \textit{polysémie}. Ce \is{concept}concept est forgé à une époque où les questions relatives au sens linguistique font l’objet d’un examen attentif, qui s’affranchit de l’habitude consistant à observer d’abord les effets de sens spécifiques à des discours particuliers\footnote{ \textrm{Cette contribution ne présentera pas l’évolution conceptuelle qu’a connue l’étude des questions de sens au cours du} \textrm{\textsc{xix}}\textrm{\textsuperscript{e}}\textrm{~siècle. Il est toutefois évident que la focale change, et que l’on se préoccupe davantage, dans la deuxième moitié du siècle, du sujet sémantiseur, qu’il soit «~cognitif~», «~psychique~» ou social… Explicitement présenté comme intersubjectif, le sens linguistique est de moins en moins conçu comme un pur effet discursif (sens sémiotextuel).}}.

Le fait d’étudier la «~cristallisation~» du \is{concept}concept de \is{polysémie}polysémie dans la période charnière durant laquelle la \is{sémantique}sémantique reçoit son nom moderne présente un double intérêt : d’une part, en ce qui concerne l’histoire des idées relatives aux faits de signification linguistique ; d’autre part, en ce qui concerne la sociologie des sciences linguistiques. Pour ce qui est de l’histoire des idées sur la langue, resituer la genèse du concept de \is{polysémie}polysémie dans le contexte intellectuel de l’époque contribue à éclairer quelques-uns des enjeux liés à son utilisation. De plus, du point de vue de la sociologie des sciences du langage, l’histoire de ce \is{concept}concept participe d’un mythe \is{fondateur}fondateur associé, dans le monde francophone, à la constitution de «~la \is{sémantique}sémantique~» en discipline. Dans ce mythe originel, l’aura symbolique du père \is{fondateur}fondateur se surimpose à la complexité des formes de contribution. Bien que ce soit à Michel Bréal~(1832-1915), alors ténor de la linguistique institutionnelle, que l’on attribue couramment la paternité du \is{terme}terme et de la notion de «~\is{polysémie}polysémie~», nous verrons que la situation dans laquelle le \is{concept}concept s’est développé est loin d’être aussi simple, et l’affaire sans doute plus délicate qu’elle semble à première vue.
 
 Bien qu’elle ne soit pas formellement nommée, l’idée que recouvre le \is{concept}concept de \is{polysémie}polysémie est souvent énoncée dans la linguistique francophone du dernier tiers du \textsc{xix}\textsuperscript{e}~siècle. Nous verrons aussi que c’est à l’orientaliste Joseph Halévy~(1827-1917), et non à Michel Bréal, que l’on doit les premières utilisations du \is{terme}terme \is{polysémie}\textit{polysémie}. Étant donné que Bréal et Halévy se côtoyaient régulièrement, on peut faire l’hypothèse que le contexte théorique général en linguistique, comme, en particulier, la conception halévienne, ont pu exercer une influence sur l’interprétation que Bréal donna à la notion de \is{polysémie}polysémie. Nous reviendrons dans la présente contribution sur cette «~erreur~» d’attribution de la paternité du \is{terme}terme et de la notion de \is{polysémie}polysémie à Michel Bréal, en tâchant de les resituer de façon sommaire dans le contexte de l’époque (\textit{i.e.} à partir des années 1860-1870)\footnote{\textstyleFootnoteSymbol{ }\textrm{L’un des relecteurs d’une version antérieure de ce texte résume, à juste titre, qu’il y est fait état d’une «~querelle de paternité~». À notre avis, cette «~erreur~» d’attribution est moins grave qu’on pourrait le croire. En revanche, il nous semble qu’elle devrait intéresser l’histoire et la sociologie de la linguistique à plus d’un titre : 1)~au plan de la circulation des termes «~consacrés~» : son origine oubliée, le terme} \is{polysémie}\textrm{\textit{polysémie}} \textrm{est devenu, une fois repris en sémantique, un puissant instrument conceptuel ; 2)~au plan de la genèse des concepts métalinguistiques, ici clairement située entre conception commune et intuitive, tradition descriptive et proposition dénominative réussie ; 3)~au plan de l’effet sociologique que peuvent exercer les sphères d’appartenance et les réseaux d’influence sur la fortune de certaines conceptions en linguistique (un autre relecteur anonyme note que les projets respectifs de Bréal et d’Halévy qui sous-tendent l’utilisation qu’ils font du terme} \is{polysémie}\textrm{\textit{polysémie}} \textrm{s’opposent nettement : Bréal, dans une optique laïque, utilise ce \is{concept}concept pour illustrer les progrès de l’esprit humain ; Halévy, au contraire, utilise le \is{concept}concept de \is{polysémie}polysémie pour défendre une thèse d’ordre religieux –~outre l’influence personnelle et institutionnelle de Bréal à l’époque, sa position est sans doute plus adaptée au contexte intellectuel français de la fin du} \textrm{\textsc{xix}}\textrm{\textsuperscript{e}}\textrm{~siècle).}}.
 
\section{En amont de la polysémie comme concept dénommé}
Les sujets parlants ont l’intuition de ce que l’on a pris coutume de nommer «~\is{polysémie}polysémie (d’un mot, d’une expression)~». Assez tôt dans l’histoire des descriptions de la langue française des sujets experts ont exprimé cette conception plurisémantique. Parmi les façons de traduire l’intuition ordinaire qu’une forme particulière peut «~comporter~» plusieurs \is{significations}significations, deux d’entre elles semblent se dégager au fil de l’histoire :\medskip

1)~la décomposition du sens linguistique suivant le modèle analytique qu’a progressivement imposé la \is{lexicographie}lexicographie (cf.~le modèle classique de l’article de \is{dictionnaire}dictionnaire, qui distingue les définitions les unes des autres : I.A.1.a, I.A.1.b…) ;\medskip

2)~la description plus intuitive de la pluralité des usages d’une forme lexicale, qui consiste à relever les \is{significations}significations associées à un mot en dehors d’un contexte particulier (par exemple, «~le mot \textit{commerce} a plusieurs sens~»), ou, de façon plus concrète, qui vise à rendre compte de la pluralité des effets de sens produits par l’inscription d’un mot dans un discours spécifique (par exemple, «~dans ce texte, le mot \textit{barbare} peut être interprété de deux façons différentes~»).

Dans la tradition française, l’idée que l’on puisse donner à un mot une pluralité d’«~acceptions~» est clairement formulée au moins depuis le \textsc{xvii}\textsuperscript{e}~siècle. La sous-entrée \textsc{Acception} de la réédition du \textit{Dictionaire} [\textit{sic}] \textit{universel} de Furetière (1619-1688) (Basnage de Bauval~1701) en témoigne : «~Sens auquel un mot se prend. Ce mot a plusieurs \textit{acceptions}. Dans sa première \& plus naturelle \textit{acception}, il signifie \& c.~» (\textit{ibid.}). À cette même époque, divers auteurs parlent d’un mot ou d’une proposition en disant qu’ils «~ont~» ou «~reçoivent~» «~plusieurs sens~» ou «~plusieurs \is{significations}significations~»\footnote{ \textrm{Un examen systématique montrerait certainement que la question de la pluralité des sens glisse en partie, entre le} \textrm{\textsc{xvii}}\textrm{\textsuperscript{e}}\textrm{~siècle}\textrm{ }\textrm{et le} \textrm{\textsc{xix}}\textrm{\textsuperscript{e}}\textrm{~siècle, du texte vers le mot. Alors que les domaines originellement associés à la question des «~sens divers~» relevaient de l’herméneutique et de l’exégèse (droit, théologie, philosophie), le phénomène est progressivement mis en valeur dans les arts du langage (rhétorique, grammaire, \is{lexicographie}lexicographie, puis histoire de la langue).}}. Voici quelques exemples de ces propos :

\begin{enumerate}
\renewcommand{\labelenumi}{$(\theenumi)$}
\item «~Ce mot au figuré a plusieurs sens.~» (Richelet~1680, sous \textbf{bouche}, \textbf{bourse}, \textbf{bout}, \textbf{bureau}, \textbf{côté}…).

\item «~AFFECTION. Ce mot a plusieurs \is{significations}significations […].~» (\citealt{regis_cours_1691}: n.~p.).

\item  «~S\textsc{wec,} \textsc{Swæcce,} \textit{olfactus}, \textit{odor}, \textit{sapor}, \textit{gustus~}; apparemment de \textit{suavis}, qui approche \& est commun à ces sens divers.~» (\citealt[930-931]{thomassin_methode_1690}:~930-931).

\item «~[…] on ajoûte [\textit{sic}], en François [\textit{sic}], les \is{significations}significations diverses des mots, \& on a soin de les bien distinguer, \& d’en rapporter des exemples à part […].~» (\citealt[172]{leclerc_bibliotheque_1715}; à propos du \textit{Dictionarium} \textit{poeticum} du père Vanier publié en 1710).

\item «~COUCHE. Ce mot a plusieurs \is{significations}significations. On le met pour marquer un lit, mais dans ce sens il ne se place guéres [\textit{sic}] que dans le burlesque, ou le stile [\textit{sic}] familier.~» (\citealt[167]{le_roux_dictionaire_1735}).

\end{enumerate}

Avant l’emploi du \is{terme}terme \is{polysémie}\textit{polysémie} en linguistique, la conception qui consiste à voir dans l’unité lexicale une pluralité de \is{significations}significations est explicitement formulée, notamment dans des contextes didactiques :

\begin{enumerate}\setcounter{enumi}{5}
\renewcommand{\labelenumi}{$(\theenumi)$}

\item «~La recherche des sens divers d’un même mot, complète […] les études nécessaires pour arriver à la connaissance de la \is{signification}signification exacte des mots et de la propriété de l’expression.~» (\citealt[252]{michel_etudes_1858}).

\end{enumerate}

Ces quelques exemples montrent que des francophones expriment l’intuition qu’ils ont de ce que l’on appelle aujourd’hui «~\is{polysémie}polysémie~», et ce bien avant la création du terme\footnote{ \textrm{Il est certain que ce type d’intuition n’est pas propre aux francophones. Une recherche approfondie devrait couvrir une période plus large et inclure d’autres langues (dont des langues sans tradition écrite). En restant sur le plan des constituants morphologiques du terme, on remarque que l’adjectif} \is{polysemous}\textrm{\textit{polysemous}} \textrm{est employé en anglais dès 1853 dans un sens proche de celui qu’on lui connaît encore : [à propos de la gestuelle en Italie] «~[…]~he [un homme napolitain, avec un ami] shakes his head and hands, uttering […] the monosyllabic but polysemous exclamation "Eh !" which, like a Chinese word, receives its meaning from its varying accent.~» (\citealt[534]{wiseman_essays_1853}).}}. Non seulement ces discours révèlent la conscience qu’ont certains auteurs de cette intuition, mais ils indiquent aussi la représentation que des sujets experts se faisaient de la \is{signification}signification linguistique, représentation qui se teinte des idées du temps : avec le développement, au \textsc{xix}\textsuperscript{e}~siècle, des études sur l’histoire de la langue française, l’intérêt pour la diversité sémantique tend à passer de l’analyse de textes (effets de sens et usages) à une conception abstraite de l’unité sémiosémantique. Émerge une conscience aiguë du développement de nouvelles \is{significations}significations (ce qu’on appelle aujourd’hui «~néologie sémantique~»).

La préoccupation pour la «~multiplication des \is{significations}significations~» est présente dans les premières utilisations que Michel Bréal fait du \is{terme}terme \is{polysémie}\textit{polysémie}. Mais elle figure aussi dans les textes de quelques-uns de ses prédécesseurs immédiats, comme Émile Littré~(1801-1881) et Arsène Darmesteter~(1846-1888)\footnote{ \textrm{Auteurs que Nerlich (2001a : 1605) considère comme les prédécesseurs immédiats de Bréal en «~sémantique~» (le terme est alors anachronique).}}. \citet[11-12]{darmesteter_traite_1874} parle de «~transformation~» ou de «~succession des sens dans les mots~» (cf. \citealt{darmesteter_sur_1876}; voir aussi \citealt{bailly_transformation_1874}), \citet[1]{littre_etudes_1880} de «~mutations de \is{signification}signification~». On relève, dans l’\textit{Histoire} \textit{de} \textit{la} \textit{langue} \textit{française} de ce dernier (\citealt{littre_histoire_1863}), les germes de la vision «~polysémiste~» de la \is{signification}signification lexicale. Bréal connaissait suffisamment l’œuvre de Littré pour qu’elle pût inspirer sa réflexion. Dans le premier texte où il présente le concept de polysémie (\citealt{breal_lhistoire_1887}), il fait référence au travail du \is{lexicographe}lexicographe. Bréal est également à l’origine de la réédition posthume de l’opuscule de Littré «~Pathologie verbale ou lésions de certains mots dans le cours de l’usage~» (\citealt{littre_etudes_1880}), auquel il préfère donner le titre de \textit{Comment} \textit{les} \textit{mots} \textit{changent} \textit{de} \textit{sens} (\citealt{littre_comment_1888}).

Les liens entre les deux «~sémantistes~» que sont Darmesteter et Bréal sont plus marqués encore. Arsène Darmesteter engage dès le début des années 1870 une réflexion sur la \is{signification}signification lexicale. On peut souvent lire, en filigrane des analyses de Bréal, la présence de Darmesteter. On remarquera aussi que les premières occurrences du \is{terme}terme \is{polysémie}\textit{polysémie} chez Bréal (au nombre de~3) sont publiées dans un texte intitulé «~L’histoire des mots~», paru en~1887 en réaction aux éléments de substrat organiciste que Bréal relève dans l’ouvrage «~éminemment suggestif~» de Darmesteter (\citealt[469]{meyer_arsene_1888}), \textit{La} \textit{vie} \textit{des} \textit{mots} \textit{étudiée} \textit{dans} \textit{leurs} \textit{significations} (sur ce point, voir \citealt{delesalle_vie_1987}). Dès le titre s’affrontent deux conceptions. L’une historiciste («~histoire des mots~»), l’autre naturaliste («~vie des mots~»). Le point de vue de Darmesteter, cependant, est plus nuancé –~et sensé~– que Bréal ne le laisse entendre. Dans sa thèse sur la néologie, parue en 1877 (soit dix ans avant le texte de \citeauthor{breal_lhistoire_1887}), Darmesteter s’interrogeait déjà, dans les termes suivants, sur les mécanismes à l’œuvre dans l’extension de la \is{signification}signification~:

\begin{quote}
Ce mot [\textit{carré}], compris de tous, a des significations multiples ; pour en faire le nom de l’objet nouveau, le peuple [parlant de l’objet du jardin anglais] sera obligé de faire un travail intellectuel qui, par une extension dans la signification, approprie le mot à la chose […]. (\citealt[33]{darmesteter_creation_1877})
\end{quote}

Écrit dix ans avant que Bréal n’emploie le \is{terme}terme \is{polysémie}\textit{polysémie} pour la première fois, ce texte montre l’existence d’un questionnement sur la multiplicité des \is{significations}significations. L’idée de \is{polysémie}polysémie est en germe.

\section{Autour de l’invention du \is{terme}terme \is{polysémie}\textit{polysémie} dans les années 1870-1880}

\begin{quote}
    \begin{small}
        
Même les théories plus ou moins bien connues et souvent discutées ne sont pas connues dans leurs connexions historiques. Ainsi, par exemple, on attribue presque toujours à Saussure les distinctions entre \textit{langue} et \textit{parole}, entre \textit{signifiant} et \textit{signifié}, entre \textit{synchronie} et \textit{diachronie}, toutes distinctions que Saussure a retrouvées dans la tradition, qu’il a, sans doute, reformulées et auxquelles il a donné en partie une interprétation nouvelle, dans le cadre d’un système cohérent, mais qu’il n’a pas été le premier à formuler. (\citealt[74]{coseriu_georg_1967})
    \end{small}
\end{quote}

Le propos de Coseriu (1921-2002) au sujet de distinctions systématisées par Saussure peut s’appliquer à nombre de notions devenues des \is{concepts}concepts-clés en linguistique, et dont la création est attribuée à des «~fondateurs~» (d’une discipline, d’un domaine, d’un courant…). Ainsi en va-t-il de la notion de polysémie : Bréal, \is{fondateur}fondateur officiel de la \is{sémantique}sémantique en France, reprend des idées anciennes à propos de la multiplicité des sens, idées qu’il développe et systématise d’une façon singulière, et qu’il nomme en (ré)utilisant un \is{terme}terme (\is{polysémie}(\textit{polysémie}) employé depuis plus d’une décennie par un confrère qu’il côtoie, écoute, et dont il commente les travaux présentés dans divers cercles savants du Paris des années 1870-1880\footnote{ \textrm{École pratique, Société de linguistique, Académie des inscriptions, etc. À ce sujet, voir \citealt{courbon_sur_2015}}}.

\subsection{Joseph Halévy : l’inventeur du \is{terme}terme \is{polysémie}\textit{polysémie}}

Contrairement à une croyance répandue, ce n’est pas à Michel Bréal, mais à l’orientaliste Joseph Halévy que l’on doit l’invention du \is{terme}terme \is{polysémie}\textit{polysémie} en français. C’est dans le but de démontrer l’origine sémitique de la langue transcrite au moyen de l’écriture cunéiforme que J.~Halévy a l’habitude d’employer ce \is{terme}terme, qui correspond à un \is{concept}concept central de son œuvre. Halévy présente la \is{polysémie}polysémie comme l’une des «~particularités les plus saillantes du système graphique assyro-babylonien~» (\citealt[298]{halevy_recherches_1876}).

Ce \is{terme}terme renvoie alors à la multiplicité de valeurs sémantiques que revêtent les signes cunéiformes. Celle-ci peut être exprimée par l’intermédiaire de signes linguistiques différents. Le \is{concept}concept de \is{polysémie}polysémie se rapproche ainsi du \is{concept}concept classique de synonymie\footnote{ \textrm{Sur la proximité conceptuelle entre synonymie et \is{polysémie}polysémie à cette époque, voir \citealt{delesalle_statut_1986}}}, mais comporte déjà, chez Halévy, les principaux traits que nous lui connaissons encore : pluralité de \is{significations}significations associées à un signe (ici, un \is{idéogramme}idéogramme) et proximité relative des différentes «~lectures~» de ce signe. Halévy décrit sa conception de la \is{polysémie}polysémie dans une communication donnée en 1878 :

\begin{quote}
    
[C]haque signe envisagé comme \is{idéogramme}idéogramme est en général rendu par plusieurs mots […], ce qui revient à dire que le signe comporte à la fois plusieurs sens, qu’il est polysème. (\citealt[275]{halevy_melanges_1883})

\end{quote}

La \is{polysémie}polysémie, qu’Halévy décrit comme la multiplication de sens équivoques, est consubstantielle au système d’écriture cunéiforme :

\begin{quote}
    [L]’accumulation infinie d’équivoques dans l’accado-sumérien, caractérise celui-ci comme un système \is{idéographique}idéographique, ou [\textit{sic}] la \is{polysémie}polysémie des signes est un principe fondamental et inéluctable. (\citealt[276]{halevy_melanges_1883})
\end{quote}

L’idée de la pluralité des \is{significations}significations d’un signe \is{idéographique}idéographique se trouvait déjà cinquante ans auparavant chez Champollion (1790-1832) (\citealt[311-312]{champollion_precis_1828}), mais elle ne recevait alors pas d’autre nom que celui, classique, de «~signes synonymes~» (\citealt[314]{champollion_precis_1828}).

\subsection{Une «~erreur~» d’attribution : «~Michel Bréal, inventeur du terme \textit{polysémie~}»}

L’attribution à la personne de Bréal de la paternité du \is{terme}terme de \textit{polysémie}  –~ou, quelquefois, du \is{concept}concept correspondant~– est présentée dans différents textes : \citet[15]{firth_technique_1957}, \citet[199]{ullmann_precis_1952}, \citet[147]{ricoeur_metaphore_1975}, \citet[286]{delesalle_linguistique_1986}, \citet[89]{delesalle_statut_1986}; \citeyear[300-305]{delesalle_vie_1987}), \citet[22] {nerlich_avant-propos_1993}; \citeyear[1625]{nerlich_study_2001}), \citet[169-170]{peeters_verbe_1993}), \citet{peeters_compte_1994}, \citet[27]{desmet_grammaire_1995}, \citet[118]{branca-rosoff_polysemie_1996}, \citet[11]{victorri_polysemie._1996}, \citet[16]{surcin_expression_1999}, \citet[215]{auroux_semantique_2000}, \citet[156]{siblot_emission_2000}, \citet[4]{nerlich_polysemy:_2003}, \citet[131]{girardin_polysemie_2004}, \citet[51]{piron_analyse_2006}, \citet[55]{cusimano_essai_2008}, \citet[22]{larrivee_histoire_2008}, \citet[116]{thibault_traitement_2009}, \citet[10]{mazaleyrat_vers_2010}, \citet[23]{pauly_polysemie._2010}, \citet[n. p.]{jakimovska_terminologie_2012}, Fr. Rainer, dans \citet{lieber_oxford_2014}, \citet[59]{derradji_forme_2014}, \citet[6]{sorba_presentation_2014},\citet[23]{bisconti_sens_2016}, \citet[4]{bruns_polysemie_2016}, \citet[103]{de_palo_invention_2016}.

Les \is{dictionnaries}dictionnaires contemporains perpétuent cette croyance. Ainsi, la date de première attestation du \is{terme}terme présentée dans le \textit{Petit} \textit{Robert~2018} et dans le \textit{Trésor} \textit{de} \textit{la} \textit{langue} \textit{française} est 1897, année de parution de l’\textit{Essai} \textit{de} \textit{sémantique} de Bréal, plus de vingt ans après son utilisation courante par Halévy. Cette erreur d’attribution (donc de datation) s’explique par le succès rapide qu’a connu l’\textit{Essai} de Bréal. Halévy, quant à lui, n’est pas associé à l’histoire de la \is{sémantique}sémantique, au contraire de Bréal, dont le nom est associé à la constitution de la \is{sémantique}sémantique en champ disciplinaire, ainsi qu’à l’usage du \is{terme polysémie}terme \textit{polysémie} (le \is{terme}terme, lui, est associé dès 1887 : \citealt{baale_bulletin_1887}).

La valeur générale du \is{terme polysémie}terme \textit{polysémie} envisagée par Halévy est proche de celle que Bréal définit en 1887. Toutefois, son association à un domaine de la linguistique en émergence, et à celui que l’on considère comme son \is{fondateur}fondateur, prévaut sur la chronologie des faits et sur la paternité réelle de la dénomination.

\subsection{La \is{polysémie}polysémie : un \is{concept}concept unificateur dès les débuts de la \is{sémantique}sémantique}

La place (deux chapitres de l’\textit{Essai}) et la valeur particulière que Bréal accorde au \is{concept}concept de \is{polysémie}polysémie indiquent son importance\footnote{ \textrm{Selon Nerlich, le \is{concept}concept de \is{polysémie}polysémie constitue le «~point focal de [l’]œuvre~» de Bréal (\citealt[22]{nerlich_avant-propos_1993}.}}. En utilisant une dénomination simple pour parler d’un ensemble de phénomènes sémantiques relativement complexes, Bréal souligne la puissance significative des formes linguistiques, qui résultent parfois d’une réduction (cf.~le rapport qu’établit Bréal entre \is{polysémie}polysémie et «~ellipse~»). En outre, par ce geste d’objectivation de phénomènes sémantiques intuitifs, Bréal assoit la légitimité de prendre le \is{sens lexical}sens lexical comme objet d’étude à part entière, contribuant ainsi à consolider le projet de fonder la «~\is{sémantique}sémantique~» en un champ de réflexion et de connaissances spécifique\footnote{ \textrm{«~[L]’avènement de la Sémantique et le surgissement de la polysémie~» vont de pair, écrit \citet[83]{delesalle_vie_1987}). Sur les relations étroites, pendant cette période, entre la lexicographie et la sémantique «~naissante~», voir \citealt{bisconti_sens_2016}.}}.

\section{Conclusion}

Créé en français par J.~Halévy dans les années~1870, le \is{terme}terme \is{polysémie}\textit{polysémie} est réemployé par M.~Bréal une dizaine d’années plus tard. L’audace terminologique de J.~Halévy, en dehors des préoccupations propres à la linguistique historique, permet d’appréhender de façon plus stable le problème de la pluralité sémantique. La solution dénominative qu’offre Halévy à la communauté des linguistes ne comporte qu’une faible valeur conceptuelle. Cette proposition ne s’insère pas directement dans le champ de réflexion contemporain sur le sens linguistique. Mais, la création du \is{terme}terme simple et relativement transparent de \textit{polysémie} contribue à attirer le regard sur une série de problèmes qui, tout au long du \textsc{xx}\textsuperscript{e}~siècle, n’ont eu de cesse d’être (re)posés. Il faut reconnaître à Bréal d’avoir insufflé à ce \is{terme}terme, en le plaçant parmi les éléments cardinaux de sa \is{sémantique}sémantique, une valeur particulière.

L’histoire primitive de la «~\is{polysémie}polysémie~» –~entre idée générale, dénomination simple, et stabilisation conceptuelle~– illustre l’interdépendance constitutive entre histoire des idées linguistiques (plan conceptuel) et histoire des réseaux de circulation des idées linguistiques (dimension sociale). Se dessine, à travers cette histoire singulière, l’effet, sur le long terme, du sociosymbolique sur le cognitif, à travers notamment la forte influence d’un sémantiste de renom, Michel Bréal, considéré dans le monde francophone comme le «~père \is{fondateur}fondateur~» de la \is{sémantique}sémantique. L’histoire primitive du concept de \is{polysémie}polysémie, telle qu’on la rencontre depuis près d’un siècle, appartient au mythe originel de la discipline.

Bréal avait pour ambition de faire de la \is{sémantique}sémantique une «~science nouvelle~». Il a insisté sur les nouveautés de sa démarche (fonction promotionnelle caractéristique des discours de \is{fondation}fondation). Le concept de \is{polysémie}polysémie fait partie des fondements conceptuels de ce mouvement épistémologique. Dans l’entreprise \is{fondationelle}fondationnelle, la référence au néologue Halévy, créateur du \is{terme polysémie}terme \textit{polysémie}, aurait sans doute compliqué le tableau. On l’a oublié. Dans les faits, la nouveauté qu’apporte Bréal tient plus à l’articulation d’ensemble qu’aux phénomènes discutés ou aux termes employés et parfois présentés comme neufs. Contrairement à Halévy, dont les préoccupations intellectuelles sont différentes, Bréal présente dans son \textit{Essai} une somme théorique, dont la force réside dans la cohérence des développements et dans la consistance des exemples. Bien qu’il pût avoir l’intuition de l’effet que produirait l’aboutissement de ses réflexions de sémantiste, Bréal ne pouvait, dans les années 1880-1890, connaître l’avenir de cette discipline dont il jetait explicitement les bases en lui donnant un nom en français. Il n’imaginait sans doute pas l’effet, sur les générations suivantes, de ses propos relatifs à la nouveauté du \is{terme polysémie}terme \textit{polysémie}. Une autre histoire reste à écrire, qui consisterait à suivre les voies par lesquelles le mythe fondateur d’un Bréal inventeur du \is{terme}terme et/ou de la notion de \is{polysémie}polysémie s’est installé durablement dans le paysage de la linguistique française\footnote{ \textrm{Je tiens ici à remercier les deux relecteurs anonymes pour leurs commentaires constructifs sur une précédente version de ce texte.}}.

\section*{Acknowledgements}

{\sloppy\printbibliography[heading=subbibliography,notkeyword=this]}
\end{document}
