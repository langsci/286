\renewcommand{\lsSeries}{hpls}
\renewcommand{\lsSeriesNumber}{3}

\renewcommand{\lsID}{286}

\title{Essays in the history of linguistics}
\renewcommand{\lsCoverTitleFont}[1]{\sffamily\addfontfeatures{Scale=MatchUppercase}\fontsize{46pt}{15.25mm}\selectfont #1}

\subtitle{Descriptive concepts and case studies}
\author{Emilie Aussant\lastand Jean-Michel Fortis}

\BackBody{This volume offers a selection of 12 papers presented during the 14th international conference on the history of the language sciences (ICHoLS XIV, Paris, 2017). It is divided thematically into two parts: (I) \textit{Metalinguistic concepts and representations}, and (II) \textit{Fields, authors and disciplinary commitments}. 

Part I brings together studies dealing with linguistic categories and descriptive concepts, such as the notion of “accidens” in Latin grammar and its Greek counterparts (Mazhuga). Three  papers address issues which have a strong echo in contemporary linguistics: the revival of localist perspectives in recent lexico-grammatical semantics and in analyses of thematic roles (Fortis), the origin of the term “polysemy” and the backdrop to its coinage by Halévy and its adoption through Bréal (Courbon), and the theoretical difficulties attending the categorization and description of prefabs, idioms and other “fixed expressions”, and the new resources available to research (Christy). This first part also includes studies dealing with ways of representing linguistic phenomena, whether these concern the treatment of local spoken varieties (so-called \textit{patois}) in the \textit{Revue des Patois Gallo-Romans} (Bergounioux), or the import of spatial representations in description of linguistic time, beyond their mere descriptive potential (Chalozin-Dovrat). Or again, now taking the word “representation” quite literally, the creation of conventional means for the visual display of grammatical relations, in the form of the first syntactic diagrams (Mazziotta). 
Part II embraces the history of language description from various perspectives, but always with a consideration of wider concerns, in particular those of a social nature: the “from below” approach to the history of Chinese Pidgin English underlines the social roles of speakers and the various speech situations they participate in (Li), while the scrutiny of Lhomond’s Latin and French textbooks demonstrates the interplay of pedagogical practice, cross-linguistic comparison and descriptive innovation (Piron). On a level we may describe as ideological, an overview of early descriptions of Central Australian languages reveals a whole spectrum of humanist to positivist and anti-humanist stances during the missionary and colonial age (Moore). In another guise, an overarching framework is also at play in the anthropological perspective championed by Meillet, whose socially and culturally oriented semantics is shown to live on in some of Benveniste’s analyses (Frigeni). And finally, in the work of Trần Đức Thảo is exhibited an original synthesis between phenomenology and Marxist semiology, whose combined resources are wielded against the “idealistic” doctrine of Saussure (D’Alonzo).}
