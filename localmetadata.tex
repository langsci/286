\renewcommand{\lsSeries}{hpls}
\renewcommand{\lsSeriesNumber}{3}

\renewcommand{\lsID}{286}

\title{Essays in the history of linguistics}
\renewcommand{\lsCoverTitleFont}[1]{\sffamily\addfontfeatures{Scale=MatchUppercase}\fontsize{46pt}{15.25mm}\selectfont #1}

\subtitle{Descriptive concepts and case studies}
\author{Emilie Aussant\lastand Jean-Michel Fortis}

\BackBody{This volume offers a selection of papers presented during the 14th international conference on the history of the language sciences (ICHoLS XIV, Paris, 2017).
 
Part I brings together studies dealing with descriptive concepts. First examined is the notion of “accidens” in Latin grammar and its Greek counterparts. Other papers address questions with a strong echo in today’s linguistics: localism and its revival in recent semantics and syntax, the origin of the term “polysemy” and its adoption through Bréal, and the difficulties attending the description of prefabs, idioms and other “fixed expressions”. This first part also includes studies dealing with representations of linguistic phenomena, whether these concern the treatment of local varieties (so-called \textit{patois}) in French research, or the import and epistemological function of spatial representations in descriptions of linguistic time. Or again, now taking the word “representation” literally, the visual display of grammatical relations, in the form of the first syntactic diagrams. 

Part II presents case studies which involve wider concerns, of a social nature: the “from below” approach to the history of Chinese Pidgin English underlines the social roles of speakers and the diversity of speech situations, while the scrutiny of Lhomond’s Latin and French textbooks demonstrates the interplay of pedagogical practice, cross-linguistic comparison and descriptive innovation. An overview of early descriptions of Central Australian languages reveals a whole spectrum of humanist to positivist and antihumanist stances during the colonial age. An overarching framework is also at play in the anthropological perspective championed by Meillet, whose socially and culturally oriented semantics is shown to live on in Benveniste. The volume ends with a paper on Trần Đức Thảo, whose work is an original synthesis between phenomenology and Marxist semiology, wielded against the “idealistic” doctrine of Saussure.}

\typesetter{Katharina Berking, Felix Kopecky, Sebastian Nordhoff}
